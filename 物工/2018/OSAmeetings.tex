%%%%%%%%%%%%%%%%%%%%%%%%%%%%%%%%%%%%%%%%%%%%%%%%%%%%%%%
%                   File: OSAmeetings.tex             %
%                  Date: 29 Novemver 2018              %
%                                                     %
%     For preparing LaTeX manuscripts for submission  %
%       submission to OSA meetings and conferences    %
%                                                     %
%       (c) 2018 Optical Society of America           %
%%%%%%%%%%%%%%%%%%%%%%%%%%%%%%%%%%%%%%%%%%%%%%%%%%%%%%%

\documentclass[12pt,dvipdfmx]{jsarticle}
%% if A4 paper needed, change letterpaper to A4
\usepackage[dvipdfmx]{graphicx}
\usepackage[dvipdfmx]{color}
\usepackage{osameet3} %% use version 3 for proper copyright statement
\usepackage{ascmac}
%% provide authormark
\newcommand\authormark[1]{\textsuperscript{#1}}

%% standard packages and arguments should be modified as needed
\usepackage{amsmath,amssymb}
\usepackage[colorlinks=true,bookmarks=false,citecolor=blue,urlcolor=blue]{hyperref} %pdflatex
%\usepackage[breaklinks,colorlinks=true,bookmarks=false,citecolor=blue,urlcolor=blue]{hyperref} %latex w/dvipdf
\usepackage{mathtools}
\usepackage{amsmath}
\usepackage{empheq}
\usepackage{physics}
\usepackage[scr=rsfs]{mathalpha}
\usepackage[svgnames]{xcolor}% tikzより前に読み込む必要あり
\usepackage{tikz}
\usepackage{bm}
\usepackage{here}
\usepackage{braket}
\usepackage{framed,color}
\usepackage{dcolumn}
\definecolor{shadecolor}{gray}{0.80}
\usetikzlibrary{perspective}
\tikzset
{%
  my ball/.style={draw,circle,minimum size=2*\r cm,inner sep=0,shading=ball,ball color=cyan!50!blue,opacity=#1},
  my ball/.default=1,
  hidden line/.style={black!60}
}
\begin{document}

\title{東大物理工学科 2018}

\author{21B00817 鈴木泰雅,\authormark{1}}

\email{\authormark{*}suzuki.t.ec@m.titech.ac.jp} %% email address is required

\section*{\Large{第一問}}
\subsection*{\large{[1.1]}}
万有引力と遠心力が釣り合う点なので,
\begin{eqnarray}
  G \frac{mM}{R^2} = m\frac{v_1^2}{R}
\end{eqnarray}
よって,
\begin{eqnarray}
  v_1 = \sqrt{G\frac{M}{R}}
\end{eqnarray}
である.
\subsection*{\large{[1.2]}}
\begin{eqnarray}
  \frac{1}{2}mv_2^2 - G\frac{mM}{R}>0 
\end{eqnarray}
を満たせばよいので
\begin{eqnarray}
  v_2 = \sqrt{G\frac{2M}{R}}
\end{eqnarray}
である.
\subsection*{\large{[1.3]}}
エネルギー保存則より
\begin{eqnarray}
  \frac{1}{2}mv^2 - G\frac{mM}{R} =: E 
\end{eqnarray}
であり,
\begin{eqnarray}
  E = \frac{1}{2}m \left( \dot{r} + r^2 \dot{\theta}^2 \right)- G\frac{Mm}{r}
\end{eqnarray}
が成立する.また,面積速度が一定である:
\begin{eqnarray}
  h = r^2 \dot{\theta} = rv
\end{eqnarray}
であるため,
\begin{eqnarray}
  \dot{\theta} = \frac{v}{r}
\end{eqnarray}
であるから,
\begin{eqnarray}
  E = \frac{1}{2}m\dot{r} + \frac{1}{2}mr^2 \left( \frac{v}{r} \right)^2 - G\frac{Mm}{r}
\end{eqnarray}
であり,長軸の位置にいるとき,$\dot{r}=0$であるため,
\begin{eqnarray}
  E = \frac{1}{2}mr^2 \left( \frac{v}{r} \right)^2 - G\frac{Mm}{r}
\end{eqnarray}
であり,それぞれの位置を$r_{\pm}$とすると,
\begin{eqnarray}
  r_{\pm} = -G\frac{mM}{2E} \pm \frac{1}{2}\sqrt{ \frac{G^2 m^2 M^2}{E^2}+ 2\frac{mv^2}{E} }
\end{eqnarray}
であり,長軸の長さは
\begin{eqnarray}
  a = r_+ + r_- = -G\frac{mM}{E}=  G \frac{M}{G\frac{M}{R}- \frac{1}{2}v^2}
\end{eqnarray}
である.
\subsection*{\large{[2.1]}}
円運動をしているため
\begin{eqnarray}
  2mR_0 \omega_0^2 = G\frac{2mM}{R_0^2}, \quad\therefore \omega_0 = \sqrt{ G\frac{M}{R_0^3} }
\end{eqnarray}
である.
\subsection*{\large{[2.2]}}
$z$軸周りの慣性モーメントは
\begin{eqnarray}
  I_z = \int_0^{l}r^2\left( m\delta(r-1/2) + m\delta(m+l/2) \right)dr =\frac{ml^2}{2}
\end{eqnarray}
\subsection*{\large{[2.3]}}

回転系で考える.それぞれの地球の中心からの距離は
\begin{eqnarray}
  r_{\pm} = \sqrt{ R_0^2 + (l/2)^2 \pm 2R_0 (l/2)\cos\phi } \approx R_0 \left( 1 \pm \frac{l}{2R_0}\cos\phi \right)
\end{eqnarray}
である.よって,トルクは
\begin{eqnarray}
  \bm{r}\times \bm{F} &&= -\left[ \frac{l}{2}mR_0 \left( 1 + \frac{l}{2R_0}\cos\phi \right)\omega_0^2 \sin\phi  \right]+\left[ \frac{l}{2}mR_0 \left( 1 - \frac{l}{2R_0}\cos\phi \right)\omega_0^2 \sin\phi  \right]\\
  &&= -\frac{1}{4}\sin(2\phi)m l^2\omega_0^2
\end{eqnarray}
である.
\subsection*{\large{[2.4]}}
トルクが$0$の時
\begin{eqnarray}
  \sin(2\phi)= 0\quad\therefore \phi=0,\frac{\pi}{2},\pi,\frac{3\pi}{2},\pi
\end{eqnarray}
である.また,運動方程式は
\begin{eqnarray}
  I_z \dot{\omega}_z = \bm{r}\times \bm{F}
\end{eqnarray}
ゆえ
\begin{eqnarray}
  \frac{ml^2}{2} \dot{\phi} = -\frac{1}{4}\sin(2\phi) l^2\omega_0^2 \approx -\frac{1}{4}2\phi l^2\omega_0^2
\end{eqnarray}
である.よって,
\begin{eqnarray}
  \ddot{\phi} = -\omega_0^2 \phi
\end{eqnarray}
となるため微小振動の角振動数は
\begin{eqnarray}
  \omega_0
\end{eqnarray}
である.
\section*{\Large{第二問}}
\subsection*{\large{[1.1]}}
ファラデーの法則より
\begin{eqnarray}
  \oint_C \bm{E}\cdot d\bm{r} = -\frac{d}{dt}\int_S \bm{B}\cdot d\bm{S}
\end{eqnarray}
であるため,
\begin{eqnarray}
  E 2\pi r= -\mu_0H_0 \cos(2\pi ft)\cdot 2\pi f \cdot \pi r^2
\end{eqnarray}
であるため,
\begin{eqnarray}
  E =  -\mu_0H_0 \cos(2\pi ft)\cdot f\pi r
\end{eqnarray}
である.
\subsection*{\large{[1.2]}}
全体の起電力は
\begin{eqnarray}
  E = \int_0^R dr\left(-\mu_0H_0 \cos(2\pi ft)\cdot f\pi r\right)
\end{eqnarray}
であり,
微小円環で発生する電力$P$は
\begin{eqnarray}
  P &&= EI =  \frac{E^2}{\rho} = \left(\mu_0H_0 \cos(2\pi ft)\cdot f\pi r\right)^2/\rho
\end{eqnarray}
である.よって,答えは
\begin{eqnarray}
  \int_0^R  PL 2\pi r dr = \left(\mu_0H_0 \cos(2\pi ft)\cdot f\pi \right)^2\frac{1}{\rho}\frac{1}{4}R^4 L 2\pi
\end{eqnarray}
\subsection*{\large{[1.3]}}
\begin{eqnarray}
  \int_0^t \left( \left(\mu_0H_0 \cos(2\pi ft)\cdot f\pi \right)^2\frac{1}{\rho}\frac{1}{4}R^4 L 2\pi\right) dt \frac{1}{C}
\end{eqnarray}

\subsection*{\large{[2.1]}}
ビオザバールの法則より
\begin{eqnarray}
  \bm{B} = \frac{I a^2}{2\mu_0(a^2+x^2)^{3/2}}\bm{e}_x
\end{eqnarray}
である.詳しくはカステラのp232
\subsection*{\large{[2.2]}}
それぞれの磁束密度を足し合わせて
\begin{eqnarray}
  \bm{B} = \frac{I a^2}{2\mu_0(a^2+(x+b)^2)^{3/2}}\bm{e}_x -\frac{I a^2}{2\mu_0(a^2+(x-b)^2)^{3/2}}\bm{e}_x
\end{eqnarray}
である.ここで$x\sim 0$の時,
\begin{eqnarray}
  (a^2 +(x\pm b)^2)^{-3/2} &&= ( a^2 + b^2 + x^2 \pm 2bx) ^{-3/2} = (a^2 + b^2 )^{-3/2} \left( 1 + \frac{x^2 \pm 2bx}{a^2+b^2} \right)^{-3/2}\\
  &&= (a^2 + b^2 )^{-3/2}\left( 1 - \frac{3}{2} \frac{x^2 \pm 2bx}{a^2+b^2} + \frac{15}{8} \left(\frac{x^2 \pm 2bx}{a^2+b^2}\right)^{2} +\cdots \right)
\end{eqnarray}
となる.ここで,$x^2$の項は無視するため
\begin{eqnarray}
  (a^2 +(x\pm b)^2)^{-3/2} \approx (a^2 + b^2 )^{-3/2} \left( 1\mp \frac{3}{2}\frac{2bx}{a^2+b^2} \right)
\end{eqnarray}
である.よって,
\begin{eqnarray}
  \bm{B} = \frac{I a^2}{2\mu_0}(a^2 + b^2 )^{-3/2} \left( -\frac{3bx}{a^2+b^2} \right)\bm{e}_x = -3 \frac{Ia^2}{\mu_0}(a^2+b^2)^{-1/2}bx\bm{e}_x
\end{eqnarray}
\subsection*{\large{[2.3]}}
\begin{eqnarray}
  \bm{B} = \frac{I a^3}{2\mu_0(a^2+(x+b)^2)^{3/2}}\bm{e}_x +\frac{I a^3}{2\mu_0(a^2+(x-b)^2)^{3/2}}\bm{e}_x
\end{eqnarray}
となる.よって,二次の項まで考えると
\begin{eqnarray}
  \bm{B} = \frac{Ia^3}{2\mu_0}\left[ 2 + x^2\left( -\frac{3}{2(a^2+b^2)} + \frac{15}{2}\frac{b^2}{(a^2+b^2)^2} \right) \right]
\end{eqnarray}
である.よって,この二次の項を$0$にするような$a,b$の関係式は
\begin{eqnarray}
  -\frac{3}{2(a^2+b^2)} + \frac{15}{2}\frac{b^2}{(a^2+b^2)^2}=0 \quad\therefore a =2b
\end{eqnarray}
である.
\newpage
\section*{\Large{第三問}}
\subsection*{\large{[1.1]}}
$x=0$での境界条件により
\begin{eqnarray}
  1+r=t
\end{eqnarray}
であり,微小区間$-\epsilon,\epsilon$でシュレディンガー方程式を積分すると
\begin{eqnarray}
  -\frac{\hbar^2}{2m}\left( \psi'(\epsilon)-\psi'(-\epsilon) \right) + \alpha \psi(0) = 0
\end{eqnarray}
となる.よって,
\begin{eqnarray}
  -\frac{\hbar^2}{2m}ik\left( t-(1-r) \right) + \alpha t = 0, \quad\therefore t = \frac{i(-1+r)}{2C-i}
\end{eqnarray}
\subsection*{\large{[1.2]}}
これらを解くと
\begin{eqnarray}
  r = \frac{-C^2-iC}{1+C^2}, \quad t =\frac{1-iC}{1+C^2}
\end{eqnarray}
となる.
\subsection*{\large{[2.1]}}
対称性を満たすようなポテンシャルに衝突する前の波動関数は
\begin{eqnarray}
  \psi(x) = \frac{1}{\sqrt{2}}\left( \psi_+(x)\psi_-(x)- \psi_-(x)\psi_+(x)\right)
\end{eqnarray}
となる.そして,それぞれ衝突した後は
\begin{eqnarray}
  \psi(x) &&\to \frac{1}{\sqrt{2}}\left[ \left( t\psi_+(x) + r\psi_-(x) \right)\left( r\psi_+(x) + t\psi_-(x) \right)- \left( r\psi_+(x) + t\psi_-(x) \right)\left( t\psi_+(x) + r\psi_-(x) \right)\right]\\
  &&= \frac{1}{\sqrt{2}}\left[ \left( t^2-r^2 \right)\psi_+\psi_- + \left( r^2-t^2 \right)\psi_-\psi_+\right]
\end{eqnarray}
となり,$\pm$の項のみであるため必ず反対方向に散乱される.


\subsection*{\large{[2.2]}}
対称な波動関数は
\begin{eqnarray}
  \psi(x) = \frac{1}{\sqrt{2}}\left( \psi_+(x)\psi_-(x)+ \psi_-(x)\psi_+(x)\right)
\end{eqnarray}
となる.よって衝突後は
\begin{eqnarray}
  \psi(x) &&\to \frac{1}{\sqrt{2}}\left[ \left( t\psi_+(x) + r\psi_-(x) \right)\left( r\psi_+(x) + t\psi_-(x) \right)+ \left( r\psi_+(x) + t\psi_-(x) \right)\left( t\psi_+(x) + r\psi_-(x) \right)\right]\\
  &&= \frac{1}{\sqrt{2}}\left[ 2tr\left( \psi_+ \psi_+ + \psi_-\psi_- \right) + \left( t^2+r^2 \right)\psi_+\psi_- + \left( r^2+t^2 \right)\psi_-\psi_+\right]
\end{eqnarray}
となる.ここで,
\begin{eqnarray}
  tr = \frac{-2C^2 + i(C^3-C)}{1+C^4+2C^2}, \quad t^2 + r^2 = \frac{1+C^4-2C^2}{1+C^4+2C^2}
\end{eqnarray}
となる.それぞれ極限をとると
\begin{eqnarray}
  &&\alpha\to 0 \quad\therefore C\to 0の時:tr\to 0,\quad t^2 + r^2 \to 1\\
  &&\alpha\to \infty \quad\therefore C\to \infty の時:tr\to 0,\quad t^2 + r^2 \to 1
\end{eqnarray}
である.よって,それぞれの極限にて,これは反対称の波動関数になる.また,$t^2 + r^2=0$の時,それぞれ同じ方向に散乱される.よって,
\begin{eqnarray}
  t^2 + r^2 =0,\quad\therefore 1+C^4-2C^2 =0 \quad\therefore C =\pm  1
\end{eqnarray}
であり,
\begin{eqnarray}
  \alpha_0 = \pm \frac{\hbar^2}{m}\frac{\sqrt{2mE}}{\hbar}
\end{eqnarray}
となる.

\subsection*{\large{[3]}}
$x<0$の領域で反射する波動関数が$0$になるためにはすべての反射する波動関数が干渉して相殺する必要がある.そこで,それぞれ$0,1,2,\cdots$回だけポテンシャルに反射されるときの反射する波動関数の重ね合わせは
\begin{eqnarray}
  re^{-ikx} + t^2re^{-ik(x+2L)} + t^2r^3e^{-ik(x+4L)} + \cdots &&= e^{-ikx}\left[ r+ rt^2 e^{-2ikL}\left( 1 + r^2 e^{-ik2L} + \cdots \right) \right]\\
  &&=e^{-ikx}\left[ r+\frac{rt^2e^{-2ikL}}{1-r^2e^{-2ikL}} \right]
\end{eqnarray}
である.これが$0$になるということは
\begin{eqnarray}
  \exp\left( -2ikL+2in\pi \right) = \frac{1}{r^2-t^2} = \frac{1+C^4+2C^2}{C^4+(2C^3+2C)i-1}
\end{eqnarray}
であることから,
\begin{eqnarray}
  L = \frac{n\pi}{k} -\frac{1}{2ik}\log\left( \frac{1+C^4+2C^2}{C^4+(2C^3+2C)i-1} \right), \quad n\in\mathbb{Z}_+
\end{eqnarray}
である.ここで,
\begin{eqnarray}
  |r^2-t^2|=1
\end{eqnarray}
であるため,
\begin{eqnarray}
  \log\left( \frac{1+C^4+2C^2}{C^4+(2C^3+2C)i-1} \right) = i\arg \left( \frac{1+C^4+2C^2}{C^4+(2C^3+2C)i-1} \right) 
\end{eqnarray}
より,
\begin{eqnarray}
  L = \frac{n\pi}{k} -\frac{1}{2k}\arg \left( \frac{1+C^4+2C^2}{C^4+(2C^3+2C)i-1} \right) \in \Re
\end{eqnarray}
である.

\newpage
\section*{\Large{第四問}}
\subsection*{\large{[1]}}
\begin{eqnarray}
  \left( \frac{\partial P}{\partial n} \right)&&= -\frac{nk_BT}{(1-bn)^2}(-b)-2an = \frac{bnk_BT}{(1-bn)^2}-2an\\
  \left( \frac{\partial^2 P}{\partial n^2} \right)&&= 2b^2\frac{nk_BT}{(1-bn)^3}-2a
\end{eqnarray}
である.よって,
\begin{eqnarray}
  n_C =\frac{1}{3b}, \quad T_C = \frac{8a}{9k_B b}
\end{eqnarray}
となる.また,圧力は
\begin{eqnarray}
  P_C = \frac{a}{3b^2}
\end{eqnarray}
である.
\subsection*{\large{[2]}}
多変数関数の逆関数の積分がこれでいいのか分からないけど...
\begin{eqnarray}
  \frac{\partial V}{\partial P} = -\frac{N}{n^2}\left( \frac{\partial P}{\partial n} \right)^{-1} = -\frac{N}{n^2}\left( \frac{bnk_BT}{(1-bn)^2}-2an \right)^{-1}
\end{eqnarray}
であり,$n_C=n, T>T_C$の時,
\begin{eqnarray}
  \frac{\partial V}{\partial P}=-\frac{N}{n_C^2} \frac{4}{3k_B(T-T_C)}
\end{eqnarray}
である.ここで,
\begin{eqnarray}
  -\frac{1}{V}\left( \frac{\partial V}{\partial P} \right)=-\frac{n_C}{N}\frac{\partial V}{\partial P} = \frac{1}{n_C} \frac{4}{3k_B(T-T_C)}
\end{eqnarray}
である.
\subsection*{\large{[3]}}
まず,$Z_0$を求めるとガウス積分より
\begin{eqnarray}
  Z_0 = \frac{1}{N!}\frac{V^{N}}{h^{3N}}\left( \frac{2m\pi}{k_B T} \right)^{3N/2}
\end{eqnarray}
となる.よって自由エネルギーは
\begin{eqnarray}
  F_0 = -k_BT \left( -N\log N +N + N\log V -3N\log h + \frac{3N}{2}\log\left( \frac{2m\pi}{k_B T} \right) \right)
\end{eqnarray}
である.また,
\begin{eqnarray}
  P_0 = -\frac{\partial F}{\partial V}, \quad S_0=-\frac{\partial F}{\partial T},\quad \mu_0 = \frac{\partial F_0}{\partial N}, \quad U_0 = F_0 + T S_0
\end{eqnarray}
である.よって,
\begin{eqnarray}
  &&P_0 = \frac{Nk_B T}{V}, \quad S_0 = k_B \left( -N\log N + \frac{5}{2}Nk_B + N\log V-3N\log h + \frac{3}{2}N\log\left( \frac{2m\pi}{k_B T} \right) \right)\\
  &&\mu_0 = -k_B T\left( -\log N + \log V -2\log h + \frac{3}{2}\log\left( \frac{2m\pi}{k_B T} \right) \right), \quad U_0 = \frac{3}{2}Nk_B T
\end{eqnarray}
である.
\subsection*{\large{[4]}}
$A$は
\begin{eqnarray}
  A = \left( \frac{V-Nv}{V} \right)^N \left[ 1+\frac{4\pi\epsilon}{V}\frac{l^3}{k_B T}\frac{1}{3}\right]^{N(N-1)/2}
\end{eqnarray}
である.ここで,
\begin{eqnarray}
  \log A &&= N \log \left( 1- \frac{N}{V}v \right) + \frac{N^2-N}{2}\log \left[ 1+\frac{4\pi\epsilon}{V}\frac{l^3}{k_B T}\frac{1}{3}\right]\\
  &&\approx N \log \left( 1- \frac{N}{V}v \right) + \frac{N^2-N}{2} \left( \frac{4\pi\epsilon}{V}\frac{l^3}{k_B T}\frac{1}{3} \right), \quad\because \epsilon/V \ll 1
\end{eqnarray}
となる.よって,
\begin{eqnarray}
  \frac{\partial \log A}{\partial V} &&= +N \frac{1}{1-\frac{N}{V}v}\frac{N}{V^2}v -\frac{N^2-N}{2}\frac{4\pi\epsilon}{V^2}\frac{l^3}{k_B T}\frac{1}{3}\\
  &&= +N \frac{1}{1-\frac{N}{V}v}\frac{N}{V^2}v -\frac{N^2-N}{2}\frac{v}{V^2}2\epsilon\frac{1}{k_BT}
\end{eqnarray}
となる.ここで,
\begin{eqnarray}
  Z = Z_0 A
\end{eqnarray}
でああるため
\begin{eqnarray}
  F = F_0 -k_BT \log A
\end{eqnarray}
となり.
\begin{eqnarray}
  F &&= -k_BT \left( -N\log N +N + N\log V -3N\log h + \frac{3N}{2}\log\left( \frac{2m\pi}{k_B T} \right) \right)\\
  &&\quad -k_BT N \log \left( 1- \frac{N}{V}v \right) + \frac{N^2-N}{2} \left( \frac{4\pi\epsilon}{V}l^3\frac{1}{3} \right)
\end{eqnarray}
となる.
\subsection*{\large{[5]}}

\begin{eqnarray}
  P = P_0 - k_BT \frac{n^2 v}{1-nv} - n^2 v\epsilon + \frac{n\epsilon}{2}\frac{N}{V^2}
\end{eqnarray}
であるが,低密度を考えてるため
\begin{eqnarray}
  P \approx nk_B T - k_BT \frac{n^2 v}{1-nv} - n^2 v\epsilon = \frac{nk_BT}{1-nv} -n^2 v\epsilon
\end{eqnarray}
となる.
\subsection*{\large{[6]}}
以上より
\begin{eqnarray}
  b = v, a = v\epsilon
\end{eqnarray}
である.
\newpage
\section*{\Large{第五問}}

\subsection*{\large{[1.1]}}
ファラデーの法則より
\begin{eqnarray}
  -\partial_z E^{(i)}=  \mu \frac{\partial }{\partial t} H^{(i)}
\end{eqnarray}
であるため,
\begin{eqnarray}
  \frac{\omega n_0}{c} E_0^{(i)} = -\mu\omega H_0^{(i)},\quad\therefore H_0^{(i)} = -\frac{n_0 }{c\mu}E_0^{(i)}
\end{eqnarray}
である.
\subsection*{\large{[1.2]}}
表面電流が流れないので,
\begin{eqnarray}
  E_0^{(i)}+E_0^{(r)} = E_0^{(t)}, \quad H_0^{(i)}+H_0^{(r)} =H_0^{(t)}
\end{eqnarray}
である.よって,それぞれ,
\begin{eqnarray}
  H_0^{(i)} = -\frac{n_0 }{c\mu}E_0^{(i)},\quad H_0^{(r)} = \frac{n_0 }{c\mu}E_0^{(r)}, \quad H_0^{(t)} = -\frac{n_g }{c\mu}E_0^{(t)}
\end{eqnarray}
を代入すると,
\begin{eqnarray}
  1+ \frac{E_0^{(r)}}{E_0^{(i)}} = \frac{n_0}{n_g}\left( 1 - \frac{E_0^{(r)}}{E_0^{(i)}} \right)
\end{eqnarray}
であり,
\begin{eqnarray}
  \frac{E_0^{(r)}}{E_0^{(i)}} = \frac{n_0-n_g}{n_0+n_g}
\end{eqnarray}
となる.
\subsection*{\large{[2.1]}}
ファラデーの法則より,
\begin{eqnarray}
  -\partial_z E_l=  \mu \frac{\partial }{\partial t} H_l
\end{eqnarray}
であり,
\begin{eqnarray}
  &&- \left[ E_l^{(-)}\left( -i\frac{\omega n_l}{c} \right)\exp\left( -i\frac{\omega n_l}{c}(z-z_l) \right) + E_l^{(+)}\left( i\frac{\omega n_l}{c} \right)\exp\left( i\frac{\omega n_l}{c}(z-z_l) \right) \right] \\
  &&=-i\mu\omega \left[ H_l^{(-)}\exp\left( -i\frac{\omega n_l}{c}(z-z_l) \right)  + H_l^{(+)}\exp\left(i\frac{\omega n_l}{c}(z-z_l) \right)  \right]
\end{eqnarray}
となる.これが任意の$z$で成立するためには,
\begin{eqnarray}
  -E_l^{(-)}\left( -i\frac{\omega n_l}{c} \right) = -i\mu\omega H_l^{(-)}, \quad  -E_l^{(+)}\left( i\frac{\omega n_l}{c} \right) = -i\mu\omega H_l^{(+)}
\end{eqnarray}
が成立する必要がある.
\subsection*{\large{[2.2]}}
境界条件より,
\begin{eqnarray}
  E_l(z=z_{l-1},t) = E_{l-1}(z=z_{l-1},t), \quad H_l(z=z_{l-1},t) = H_{l-1}(z=z_{l-1},t)
\end{eqnarray}
となる.よって,
\begin{eqnarray}
  &&\left[ E_l^{(-)}\exp\left( -i\frac{\omega n_l}{c}(z_{l-1}-z_{l}) \right) + E_l^{(+)}\exp\left( i\frac{\omega n_l}{c}(z_{l-1}-z_{l}) \right) \right]\\
  && = \left[ E_{l-1}^{(-)}\exp\left( -i\frac{\omega n_{l-1}}{c}(0) \right) + E_{l-1}^{(+)}\exp\left( i\frac{\omega n_{l-1}}{c}(0) \right) \right]
\end{eqnarray}
であるため,
\begin{eqnarray}
  E_l^{(-)}\exp\left( -i\Delta_l \right) + E_l^{(+)}\exp\left( i\Delta_l \right) = E_{l-1}^{(-)} +  E_{l-1}^{(+)}
\end{eqnarray}
となる.また,$H$の境界条件から,
\begin{eqnarray}
  H_l^{(-)}\exp\left( -i\Delta_l \right) + H_l^{(+)}\exp\left( i\Delta_l \right) = H_{l-1}^{(-)} +  H_{l-1}^{(+)}
\end{eqnarray}
である.ここで,それぞれの係数を代入すると
\begin{eqnarray}
  -\frac{1}{n_l}E_l^{(-)}\exp\left( -i\Delta_l \right) +\frac{1}{n_l} E_l^{(+)}\exp\left( i\Delta_l \right) = -\frac{1}{n_{l-1}}E_{l-1}^{(-)} +\frac{1}{n_{l-1}}  E_{l-1}^{(+)}
\end{eqnarray}
である.よって,
\begin{eqnarray}
  \frac{E_{l-1}^{(-)}-E_{l-1}^{(+)}}{E_{l-1}^{(-)}+E_{l-1}^{(+)}} = \frac{n_{l-1}}{n_l}\frac{E_l^{(-)}\exp\left( -i\Delta_l \right)-E_l^{(+)}\exp\left( i\Delta_l \right)}{E_l^{(-)}\exp\left( -i\Delta_l \right)+E_l^{(+)}\exp\left( i\Delta_l \right)}
\end{eqnarray}
となる.よって,$\alpha_l = n_{l-1}/n_l$となる.
\subsection*{\large{[2.3]}}
$\Delta_{l}=\pi/2$の時,
上式は
\begin{eqnarray}
  \underbrace{\frac{E_{l-1}^{(-)}-E_{l-1}^{(+)}}{E_{l-1}^{(-)}+E_{l-1}^{(+)}}}_{b_{l-1}} = \underbrace{\frac{n_{l-1}}{n_l}}_{\alpha_l}\underbrace{\frac{E_l^{(-)}+E_l^{(+)}}{E_l^{(-)}-E_l^{(+)}}}_{1/b_l}, \quad\therefore \alpha_l = b_{l-1}\cdot b_l
\end{eqnarray}
となり,それぞれ$b_{l-1},b_l$を設定すると
\begin{eqnarray}
  \alpha_1\cdot \frac{\alpha_3}{\alpha_2}\cdot\frac{\alpha_5}{\alpha_4}\cdots \frac{\alpha_{2N+1}}{\alpha_{2N}} = (b_0\cdot b_1)\frac{b_2 \cdot b_3}{b_1 \cdot b_2}\cdots \frac{b_{2N}\cdot b_{2N+1}}{b_{2N-1}\cdot b_{2N}} = b_0 \cdot b_{2N+1}
\end{eqnarray}
である.ここで,
\begin{eqnarray}
  \alpha_1\cdot \frac{\alpha_3}{\alpha_2}\cdot\frac{\alpha_5}{\alpha_4}\cdots \frac{\alpha_{2N+1}}{\alpha_{2N}}  = \frac{n_0}{n_L}\underbrace{\frac{n_H/n_L}{n_L/n_H}\cdots \frac{n_H/n_L}{n_L/n_H}}_{2N-1個} \cdot \frac{n_H/n_g}{n_L/n_H} = \frac{n_0}{n_L}\frac{n_H}{n_g}\left( \frac{n_H}{n_L} \right)^{2N}
\end{eqnarray}
であり,ガラスの層では反射はないため,
\begin{eqnarray}
  b_0 = \frac{1-r_1}{1+r_1}, \quad b_{2N+1} = \frac{E_{2N+1}^{(-)}-E_{2N+1}^{(+)}}{E_{2N+1}^{(-)}+E_{2N+1}^{(+)}} =1 \quad\because E_{2N+1}^{(+)} =0
\end{eqnarray}
であることから,
\begin{eqnarray}
  \frac{n_0}{n_g}\left( \frac{n_H}{n_L} \right)^{2N+1} = \frac{1-r_1}{1+r_1}, \quad\therefore r_1 = \frac{1-\frac{n_0}{n_g}\left( \frac{n_H}{n_L} \right)^{2N+1}}{1+\frac{n_0}{n_g}\left( \frac{n_H}{n_L} \right)^{2N+1}}
\end{eqnarray}
となる.よって反射しない時は
\begin{eqnarray}
  1 = \frac{n_0}{n_g}\left( \frac{n_H}{n_L} \right)^{2N+1}
\end{eqnarray}
となる.

\section*{\Large{第六問}}
\subsection*{\large{[1.1]}}


\end{document}