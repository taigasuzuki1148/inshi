%%%%%%%%%%%%%%%%%%%%%%%%%%%%%%%%%%%%%%%%%%%%%%%%%%%%%%%
%                   File: OSAmeetings.tex             %
%                  Date: 29 Novemver 2018              %
%                                                     %
%     For preparing LaTeX manuscripts for submission  %
%       submission to OSA meetings and conferences    %
%                                                     %
%       (c) 2018 Optical Society of America           %
%%%%%%%%%%%%%%%%%%%%%%%%%%%%%%%%%%%%%%%%%%%%%%%%%%%%%%%

\documentclass[12pt,dvipdfmx]{jsarticle}
%% if A4 paper needed, change letterpaper to A4
\usepackage[dvipdfmx]{graphicx}
\usepackage[dvipdfmx]{color}
\usepackage{osameet3} %% use version 3 for proper copyright statement
\usepackage{ascmac}
%% provide authormark
\newcommand\authormark[1]{\textsuperscript{#1}}

%% standard packages and arguments should be modified as needed
\usepackage{amsmath,amssymb}
\usepackage[colorlinks=true,bookmarks=false,citecolor=blue,urlcolor=blue]{hyperref} %pdflatex
%\usepackage[breaklinks,colorlinks=true,bookmarks=false,citecolor=blue,urlcolor=blue]{hyperref} %latex w/dvipdf
\usepackage{mathtools}
\usepackage{amsmath}
\usepackage{empheq}
\usepackage{physics}
\usepackage[scr=rsfs]{mathalpha}
\usepackage[svgnames]{xcolor}% tikzより前に読み込む必要あり
\usepackage{tikz}
\usepackage{bm}
\usepackage{here}
\usepackage{braket}
\usepackage{framed,color}
\usepackage{dcolumn}
\definecolor{shadecolor}{gray}{0.80}
\usetikzlibrary{perspective}
\tikzset
{%
  my ball/.style={draw,circle,minimum size=2*\r cm,inner sep=0,shading=ball,ball color=cyan!50!blue,opacity=#1},
  my ball/.default=1,
  hidden line/.style={black!60}
}
\begin{document}

\title{東大物理工学科 2024}

\author{21B00817 鈴木泰雅,\authormark{1}}

\email{\authormark{*}suzuki.t.ec@m.titech.ac.jp} %% email address is required

\section*{\Large{第二問}}
\subsection*{\large{[1.1]}}
\begin{eqnarray}
  m\ddot{x}_n = k x_{n+1}(t)- 2k x_n(t) + k x_{n-1}(t)
\end{eqnarray}
\subsection*{\large{[1.2]}}
計算すると
\begin{eqnarray}
  m\ddot{c}_q(t)= \left(2k\cos(q)-2k\right) c_q(t),\quad\therefore \ddot{c}_q(t) = -\frac{4k}{m}\sin^2(q/2)c_q(t)
\end{eqnarray}
\subsection*{\large{[1.3]}}
前の問題より
\begin{eqnarray}
  \omega_q = 2\sqrt{\frac{k}{m}}\left| \sin\left(\frac{q}{2}\right) \right|
\end{eqnarray}
\subsection*{\large{[1.4]}}
$q=0$のとき,
\begin{eqnarray}
  \ddot{c}_q(t) =0\quad\therefore c_q(t) = At + B
\end{eqnarray}
\subsection*{\large{[2.1]}}
\subsubsection*{(i): $n\leq -1,n\geq 1$の時}
前問と同様にして
\begin{eqnarray}
  m\ddot{x}_n = k x_{n+1}(t)- 2k x_n(t) + k x_{n-1}(t)
\end{eqnarray}

\subsubsection*{(ii): $n=0$の時}
$M$に気を付けて
\begin{eqnarray}
  M\ddot{x}_0 = k x_{1}(t)- 2k x_0(t) + k x_{-1}(t)
\end{eqnarray}
\subsection*{\large{[2.2]}}
それぞれ代入して整理すると
\begin{eqnarray}
  -\omega_q^2m \left( e^{iqn}+ R_q e^{-iqn} \right) =k\left( e^{iqn}+ R_q e^{-iqn} \right) \left( 2cos(q)-2 \right)
\end{eqnarray}
また,
\begin{eqnarray}
  -\omega_q^2m = k\left( 2cos(q)-2 \right)
\end{eqnarray}
よりそれぞれ成立する.
\subsection*{\large{[2.3]}}
\subsubsection*{(i):$n=0$のとき}
\begin{eqnarray}
  M\left( -\omega_q^2\left( T_q \right) \right) = k\left( T_q e^{iq}-2T_q+e^{-iq} + R_q e^{iq} \right) 
\end{eqnarray}

\subsubsection*{(ii):$n=-1$のとき}
\begin{eqnarray}
  -\omega_q^2m \left( e^{-iq} + R_q e^{iq} \right) = k\left( T_q-2 \left( e^{-iq} + R_q e^{iq} \right) + e^{-2iq} + R_q e^{2iq} \right)
\end{eqnarray}
よって,これを行列で表すと
\begin{eqnarray}
  \begin{bmatrix}
    -\omega_q^2M -ke^{iq}+2k & -ke^{-iq} \\
    -k & -\omega_q^2 m e^{iq} + 2ke^{iq}-k e^{2iq}
  \end{bmatrix}
  \begin{bmatrix}
    T_q \\
    R_q
  \end{bmatrix}
  =
  \begin{bmatrix}
    ke^{iq} \\
    \omega_q^2 m e^{-iq} -2ke^{-iq}+k e^{-2iq}
  \end{bmatrix}
\end{eqnarray}
これに$\omega_q$を代入すると
\begin{eqnarray}
  \begin{bmatrix}
    \frac{M}{m}k(e^{iq}+e^{-iq}-1)-ke^{iq}+2k & -ke^{iq}\\
    -k & k 
  \end{bmatrix}
  \begin{bmatrix}
    T_q \\
    R_q
  \end{bmatrix} =
  \begin{bmatrix}
    ke^{-iq} \\
    -k
  \end{bmatrix}
\end{eqnarray}
\subsection*{\large{[2.4]}}
下の式より
\begin{eqnarray}
  -k T_q + kR_q = -k \quad\therefore R_q = T_q-1
\end{eqnarray}
となる.また,上の方の式より
\begin{eqnarray}
  T_q = \frac{e^{-iq}-e^{iq}}{M/m(e^{iq}+e^{-iq}-1)-2e^{iq}+1}
\end{eqnarray}
となる.
\subsection*{\large{[2.5]}}
\subsubsection*{$M=m$のとき}
\begin{eqnarray}
  T_q =1
\end{eqnarray}
であり,加えた撃力がそのまま減衰することなく伝わることを示している.
\subsubsection*{$M=\infty$のとき}
\begin{eqnarray}
  T_q =0
\end{eqnarray}
となり,これは撃力が伝わらないことを意味している.

\subsection*{\large{[3.1]}}


\end{document}