%%%%%%%%%%%%%%%%%%%%%%%%%%%%%%%%%%%%%%%%%%%%%%%%%%%%%%%
%                   File: OSAmeetings.tex             %
%                  Date: 29 Novemver 2018              %
%                                                     %
%     For preparing LaTeX manuscripts for submission  %
%       submission to OSA meetings and conferences    %
%                                                     %
%       (c) 2018 Optical Society of America           %
%%%%%%%%%%%%%%%%%%%%%%%%%%%%%%%%%%%%%%%%%%%%%%%%%%%%%%%

\documentclass[12pt,dvipdfmx]{jsarticle}
%% if A4 paper needed, change letterpaper to A4
\usepackage[dvipdfmx]{graphicx}
\usepackage[dvipdfmx]{color}
\usepackage{osameet3} %% use version 3 for proper copyright statement
\usepackage{ascmac}
%% provide authormark
\newcommand\authormark[1]{\textsuperscript{#1}}

%% standard packages and arguments should be modified as needed
\usepackage{amsmath,amssymb}
\usepackage[colorlinks=true,bookmarks=false,citecolor=blue,urlcolor=blue]{hyperref} %pdflatex
%\usepackage[breaklinks,colorlinks=true,bookmarks=false,citecolor=blue,urlcolor=blue]{hyperref} %latex w/dvipdf
\usepackage{mathtools}
\usepackage{amsmath}
\usepackage{empheq}
\usepackage{physics}
\usepackage[scr=rsfs]{mathalpha}
\usepackage[svgnames]{xcolor}% tikzより前に読み込む必要あり
\usepackage{tikz}
\usepackage{bm}
\usepackage{here}
\usepackage{braket}
\usepackage{framed,color}
\usepackage{dcolumn}
\definecolor{shadecolor}{gray}{0.80}
\usetikzlibrary{perspective}
\tikzset
{%
  my ball/.style={draw,circle,minimum size=2*\r cm,inner sep=0,shading=ball,ball color=cyan!50!blue,opacity=#1},
  my ball/.default=1,
  hidden line/.style={black!60}
}
\begin{document}
\title{東大物理工学科 2017}

\author{21B00817 鈴木泰雅,\authormark{1}}
\section*{\Large{第一問}}
\subsection*{\large{[1.1]}}
エネルギー保存則より
\begin{eqnarray}
  mgR = mgR\cos\theta + \frac{1}{2}mv^2, \quad\therefore v = \sqrt{2gR(1-\cos\theta)}
\end{eqnarray}
\subsection*{\large{[1.2]}}
回転系で見ると遠心力がかかるため
\begin{eqnarray}
  m\frac{v^2}{R} + N = mg\cos\theta, \quad\therefore N = mg\cos\theta-m\frac{v^2}{R} = mg(3\cos\theta-2) =0
\end{eqnarray}
の時に質点が離れるため
\begin{eqnarray}
  \cos\theta = \frac{2}{3}
\end{eqnarray}
よって,速度は
\begin{eqnarray}
  v_{c1} = \sqrt{ \frac{2}{3}gR}
\end{eqnarray}
\subsection*{\large{[2.1]}}
\begin{eqnarray}
  I &&= \int dm r'^2 = \frac{3m}{4\pi r^3}\int r'^3 dr' d\theta dz , \quad 0 \leq r' \leq \sqrt{r^2-z^2}\\
  &&=   \frac{3m}{4\pi r^3}\cdot 2\pi \int_{-r}^{r} \frac{1}{4} (r^2-z^2)^2 dz\\
  &&= \frac{2}{5}mr^2
\end{eqnarray}
\subsection*{\large{[2.2]}}
図より
\begin{eqnarray}
  R\theta = r(\phi-\theta), \quad\therefore (R+r)\dot{\theta} = r\dot{\phi}
\end{eqnarray}
よって
\begin{eqnarray}
  v = \frac{\dot{\theta}}{R+r}, \omega = \dot{\phi}
\end{eqnarray}
より,
\begin{eqnarray}
  (R+r)^2 v = r\omega
\end{eqnarray}
\subsection*{\large{[2.3]}}
束縛条件は
\begin{eqnarray}
  f_r(r') = r' - (r+R) = 0, \quad f_{\theta,\phi}(\theta,\phi) = (R+r)\theta -r\phi =0
\end{eqnarray}
であり,束縛条件がない時のラグランジアンは
\begin{eqnarray}
  L_0 = \frac{1}{2}m (\dot{r}'^2 + r'^2 \dot{\theta}^2) + \frac{1}{2}\left( \frac{2}{5}mr^2 \right)\dot{\phi}^2 - mgr' \cos\theta
\end{eqnarray}
よって,束縛条件がある場合のラグランジュ方程式は
\begin{eqnarray}
  m\ddot{r}' - ( mr'\dot{\theta}^2 -mg\cos\theta ) &&= \lambda_r \frac{\partial f_r}{\partial r} = \lambda_r\\
  mr'^2\ddot{\theta}-mgr'\sin\theta &&= \lambda_{\theta,\phi}(R+r)\\
  \frac{1}{5}mr'^2 \ddot{\phi} = \lambda_{\theta,\phi}(-r)
\end{eqnarray}
よって,$\lambda_{\theta,\phi}$を消去して,また,$r$の束縛条件より
\begin{eqnarray}
  mr'^2\ddot{\theta}-mgr'\sin\theta = -\frac{m}{5}(R+r)^2 \ddot{\theta}, \quad\therefore \frac{6}{5}mr'\ddot{\theta}-mg\sin\theta =0
\end{eqnarray}
ここで,
\begin{eqnarray}
  \frac{6}{5}mr' \dot{\theta}\ddot{\theta} - \dot{\theta}mg\sin\theta = 0\quad\therefore \frac{3}{5}mr' \dot{\theta}^2 + mg\cos\theta =0
\end{eqnarray}
よって,束縛条件より$\dot{r}', \ddot{r}' =0$であるため,
\begin{eqnarray}
  -mr' \frac{5}{3m}g(1-\cos\theta) + mg\cos\theta = \lambda_r, \quad\therefore \cos\theta = \frac{2}{5}
\end{eqnarray}
また,
\begin{eqnarray}
  \dot{\theta} = \sqrt{ \frac{5}{3(R+r)}\frac{3}{5} } = \frac{\sqrt{g}}{\sqrt{ R+r }}
\end{eqnarray}
より,
\begin{eqnarray}
  v = \frac{\sqrt{g}}{(R+r)^{3/2}}
\end{eqnarray}
\subsection*{\large{[2.4]}}
回転により寄与がかかるから.

\end{document}