%%%%%%%%%%%%%%%%%%%%%%%%%%%%%%%%%%%%%%%%%%%%%%%%%%%%%%%
%                   File: OSAmeetings.tex             %
%                  Date: 29 Novemver 2018              %
%                                                     %
%     For preparing LaTeX manuscripts for submission  %
%       submission to OSA meetings and conferences    %
%                                                     %
%       (c) 2018 Optical Society of America           %
%%%%%%%%%%%%%%%%%%%%%%%%%%%%%%%%%%%%%%%%%%%%%%%%%%%%%%%

\documentclass[12pt,dvipdfmx]{jsarticle}
%% if A4 paper needed, change letterpaper to A4
\usepackage[dvipdfmx]{graphicx}
\usepackage[dvipdfmx]{color}
\usepackage{osameet3} %% use version 3 for proper copyright statement
\usepackage{ascmac}
%% provide authormark
\newcommand\authormark[1]{\textsuperscript{#1}}

%% standard packages and arguments should be modified as needed
\usepackage{amsmath,amssymb}
\usepackage[colorlinks=true,bookmarks=false,citecolor=blue,urlcolor=blue]{hyperref} %pdflatex
%\usepackage[breaklinks,colorlinks=true,bookmarks=false,citecolor=blue,urlcolor=blue]{hyperref} %latex w/dvipdf
\usepackage{mathtools}
\usepackage{amsmath}
\usepackage{empheq}
\usepackage{physics}
\usepackage[scr=rsfs]{mathalpha}
\usepackage[svgnames]{xcolor}% tikzより前に読み込む必要あり
\usepackage{tikz}
\usepackage{bm}
\usepackage{here}
\usepackage{braket}
\usepackage{framed,color}
\usepackage{dcolumn}
\definecolor{shadecolor}{gray}{0.80}
\usetikzlibrary{perspective}
\tikzset
{%
  my ball/.style={draw,circle,minimum size=2*\r cm,inner sep=0,shading=ball,ball color=cyan!50!blue,opacity=#1},
  my ball/.default=1,
  hidden line/.style={black!60}
}
\begin{document}
\title{東大物理工学科 2013}

\author{21B00817 鈴木泰雅,\authormark{1}}
\section*{\Large{第四問}}
\subsection*{\large{[1]}}
それぞれ周期境界条件があるため,
\begin{eqnarray}
  p_{x,1} = \frac{2\pi\hbar}{L}n_1,\cdots
\end{eqnarray}
のように$p_{x,i},p_{y,i}$それぞれに関して周期的境界条件が成立する.つまり運動量空間において,$\frac{L}{2\pi\hbar}$倍をすると量子状態の数が求められるため
$E$よりも小さいエネルギーの量子状態は
\begin{eqnarray}
  \Omega(E)=\frac{L^{4}}{(2\pi\hbar)^4}\cdot (半径\sqrt{2mE}の4次元球の体積)
\end{eqnarray}
である.よって,$E\sim E+dE$にある量子状態の数は
\begin{eqnarray}
  W(E) = \frac{d\Omega(E)}{dE}dE = \frac{L^{4}}{(2\pi\hbar)^4}\cdot (半径\sqrt{2mE}の4次元球の表面積)\cdot 2mE \frac{dE}{E}
\end{eqnarray}
である.よって示せた.
\subsection*{\large{[2]}}
2次元球(円)の体積は図より
\begin{eqnarray}
  \int_{-r}^{r} dq_1 2\sqrt{r^2-q_1^2}
\end{eqnarray}
で求まる.ここで,2次元球の表面積はこの体積を$r$で微分したものだから
\begin{eqnarray}
  S_2(r) = \int_{-r}^{r} dq_1 \frac{2r}{\sqrt{r^2-q_1^2}}
\end{eqnarray}
である.同様にして3次元球の体積は$q_1=x,q_2=y$として見立てることによって
\begin{eqnarray}
  \int_{-r}^{r}dq_1 \int_{-\sqrt{r^2-q_1^2}}^{\sqrt{r^2-q_1^2}}dq_2 2\sqrt{r^2-q_1^2-q_2^2}
\end{eqnarray}
である.これを$r$で微分することによって$S_3(r)$を導ける.同様にして$S_4(r)$も導ける.
\subsection*{\large{[3]}}
$v_1$を満たす量子状態の個数を$W'$とすると等確率の原理から
\begin{eqnarray}
  P(v_1)dv_1 = \frac{W'}{W(E)}
\end{eqnarray}
である.ただし今回は規格化をしなくても良いため
\begin{eqnarray}
  P(v_1)dv_1 = W'
\end{eqnarray}
として良い.全体のエネルギーは
\begin{eqnarray}
  E = \frac{1}{2}mv_1^2 + E_2
\end{eqnarray}
であり,全体の量子状態は$E_2$に関してのみ数え上げすればよい.これは全体のエネルギー
\begin{eqnarray}
  \epsilon = E-\frac{1}{2}mv_1^2
\end{eqnarray}
として1粒子の2次元状態であるため
\begin{eqnarray}
  \Omega'(\epsilon) = \frac{L^2}{(2\pi\hbar)^2}\pi(2m\epsilon)
\end{eqnarray}
よって,
\begin{eqnarray}
  W' = \frac{d\Omega'}{d\epsilon}d\epsilon = (定数)\cdot (-mv_1)dv_1
\end{eqnarray}
である.よって,$P(v_1)= v_1$?

\end{document}