%%%%%%%%%%%%%%%%%%%%%%%%%%%%%%%%%%%%%%%%%%%%%%%%%%%%%%%
%                   File: OSAmeetings.tex             %
%                  Date: 29 Novemver 2018              %
%                                                     %
%     For preparing LaTeX manuscripts for submission  %
%       submission to OSA meetings and conferences    %
%                                                     %
%       (c) 2018 Optical Society of America           %
%%%%%%%%%%%%%%%%%%%%%%%%%%%%%%%%%%%%%%%%%%%%%%%%%%%%%%%

\documentclass[12pt,dvipdfmx]{jsarticle}
%% if A4 paper needed, change letterpaper to A4
\usepackage[dvipdfmx]{graphicx}
\usepackage[dvipdfmx]{color}
\usepackage{osameet3} %% use version 3 for proper copyright statement
\usepackage{ascmac}
%% provide authormark
\newcommand\authormark[1]{\textsuperscript{#1}}

%% standard packages and arguments should be modified as needed
\usepackage{amsmath,amssymb}
\usepackage[colorlinks=true,bookmarks=false,citecolor=blue,urlcolor=blue]{hyperref} %pdflatex
%\usepackage[breaklinks,colorlinks=true,bookmarks=false,citecolor=blue,urlcolor=blue]{hyperref} %latex w/dvipdf
\usepackage{mathtools}
\usepackage{amsmath}
\usepackage{empheq}
\usepackage{physics}
\usepackage[scr=rsfs]{mathalpha}
\usepackage[svgnames]{xcolor}% tikzより前に読み込む必要あり
\usepackage{tikz}
\usepackage{bm}
\usepackage{here}
\usepackage{braket}
\usepackage{framed,color}
\usepackage{dcolumn}
\definecolor{shadecolor}{gray}{0.80}
\usetikzlibrary{perspective}
\tikzset
{%
  my ball/.style={draw,circle,minimum size=2*\r cm,inner sep=0,shading=ball,ball color=cyan!50!blue,opacity=#1},
  my ball/.default=1,
  hidden line/.style={black!60}
}
\begin{document}
\title{東大物理工学科 2013}

\author{21B00817 鈴木泰雅,\authormark{1}}
\section*{\Large{第一問}}
\subsection*{\large{[1]}}
連立方程式を解くことによって
\begin{eqnarray}
  \bm{r}_2 = \bm{R} + \frac{m_1}{m_1+m_2}\bm{r}, \quad \bm{r}_1 = \bm{R} - \frac{m_2}{m_1+m_2}\bm{r}
\end{eqnarray}
よって,これらを代入することによって,
\begin{eqnarray}
  \mathcal{L} = \frac{1}{2}(m_1+m_2)\dot{\bm{R}}^2 + \frac{1}{2}\left( \frac{m_1m_2}{m_1+m_2} \right)\bm{\dot{r}}^2 - U(\bm{r})
\end{eqnarray}
となる.よって示せた.
\subsection*{\large{[2]}}
\begin{eqnarray}
  \bm{\dot{r}}^2= \left( \dot{r}\cos\phi - r\sin\phi\dot{\phi} \right)^2 + \left( \dot{r}\cos\phi +r\cos\phi\dot{\phi} \right)^2 = \dot{r}^2 + r^2\dot{\phi}^2
\end{eqnarray}
であるため示せた.
\subsection*{\large{[3]}}
$r$成分のラグランジュ方程式より 
\begin{eqnarray}
  \mu \ddot{r} = \frac{\partial}{\partial r} \left( \mu r \dot{\phi}^2 -U(r) \right)
\end{eqnarray}
であり,
\begin{eqnarray}
  \frac{d}{dr}\left( \mu r^2\dot{\phi}^2 \right) &&= \frac{\partial}{\partial r}\left(\mu r^2\dot{\phi}^2\right) + \frac{\partial}{\partial \dot{\phi}}\left(\mu r^2\dot{\phi}^2\right) \frac{d\dot{\phi}}{dr}\\
  &&= 2\mu r \dot{\phi}^2 - 4 \mu r \dot{\phi}^2 = - 2\mu r \dot{\phi}^2\\
  &&= - \frac{\partial}{\partial r}\left(\mu r^2\dot{\phi}^2\right)
\end{eqnarray}
であるため,
\begin{eqnarray}
  \mu \ddot{r} = -\frac{d}{d r} \left( \mu r \dot{\phi}^2 +U(r) \right) =  -\frac{d}{d r} \left( \frac{1}{2}\frac{l^2}{\mu r^2} +U(r) \right) 
\end{eqnarray}
この物理的な意味としては大きなポテンシャル(一般化したポテンシャル)と見なすことができ,第一項は遠心力によるポテンシャル,第二項は外力によるポテンシャルである.

\subsection*{\large{[4]}}
両辺に$\dot{r}$を書けると
\begin{eqnarray}
  \mu\dot{r}\ddot{r} = -\frac{d}{dr}\frac{dr}{dt} \left( \frac{1}{2}\frac{l^2}{\mu r^2} +U(r) \right)  = - \frac{d}{dt}\left( \frac{1}{2}\frac{l^2}{\mu r^2} +U(r) \right) 
\end{eqnarray}
であり,左辺は
\begin{eqnarray}
  \frac{d}{dt} \left( \frac{1}{2}\mu r^2 \right) =  - \frac{d}{dt}\left( \frac{1}{2}\frac{l^2}{\mu r^2} +U(r) \right) 
\end{eqnarray}
より確かにエネルギー保存が実現している.
\subsection*{\large{[5]}}
まず$\bm{l}$と垂直であることを示す.そもそも$\bm{r},\dot{\bm{r}}$は$xy$平面上での運動であったため$z$成分も持たない.よって自明に垂直である.
また,
\begin{eqnarray}
  \bm{\dot{r}} =\left( \dot{r}\cos\phi-r\dot{\phi}\sin\phi \right)\bm{e}_x + \left( \dot{r}\sin\phi+r\dot{\phi}\cos\phi \right)\bm{e}_y
\end{eqnarray}
であり,$\bm{l}=l\bm{e}_z$であるため,
\begin{eqnarray}
  \bm{\dot{r}}\times\bm{l}= l\left( \dot{r}\cos\phi-r\dot{\phi}\sin\phi \right)(-\bm{e}_y)+ l \left( \dot{r}\sin\phi+r\dot{\phi}\cos\phi \right)\bm{e}_x
\end{eqnarray}
であるため,
\begin{eqnarray}
  \mu\left( \bm{\dot{r}}\times\bm{l} \right)\cdot\bm{r} = \mu l r^2 \dot{\phi}
\end{eqnarray}
であるため,
\begin{eqnarray}
  Ar\cos\alpha =  \mu l r^2 \dot{\phi}-\mu k r = l^2 - \mu k r
\end{eqnarray}
であるため,
\begin{eqnarray}
  \frac{1}{r} = \frac{\mu k}{l^2}\left( 1+\frac{A}{\mu k}\cos\alpha \right) 
\end{eqnarray}
が成立する.

\newpage
\section*{\Large{第二問}}
\subsection*{\large{[1.1]}}
ビオザバールの公式を使う.
\begin{eqnarray}
  d\bm{H} = \frac{I}{4\pi} \frac{d\bm{s}\times \bm{r}}{r^2}
\end{eqnarray}
より
\begin{eqnarray}
  dH = \frac{Ids}{4\pi(a^2+z^2)}\frac{a}{r} = \frac{Iads}{4\pi(a^2+z^2)^{3/2}},\quad\therefore H = \frac{Ia}{4\pi(a^2+z^2)^{3/2}}\int_0^{2\pi a}ds = \frac{Ia^2}{2(a^2+z^2)^{3/2}}
\end{eqnarray}
\subsection*{\large{[1.2]}}
単位長さあたり巻き数は$N/l$であるため,
\begin{eqnarray}
  H = \frac{N}{l}I,\quad\therefore B= \mu_0 \frac{N}{l}I
\end{eqnarray}
\subsection*{\large{[1.3]}}
Maxwell 方程式より
\begin{eqnarray}
  \nabla\times\bm{E} = -\frac{\partial \bm{B}}{\partial t},\quad\therefore \oint_C \bm{E}\cdot d\bm{l} = -\frac{\partial}{\partial t}\left( \bm{B}\cdot \bm{S} \right) = -\mu_0 \frac{N}{l}S \frac{dI}{dt}
\end{eqnarray}
であり,$\oint_C \bm{E}\cdot d\bm{l} $はちょうどコイル内の電圧であるため,
\begin{eqnarray}
  L = \mu_0 \frac{N}{l}S
\end{eqnarray}
である.
\subsection*{\large{[2]}}
電場に関して考える.定義より下向きを正として,また,$r=d/2+x$より,
\begin{eqnarray}
  V = E(x,t) x,\quad\therefore v_0\sin\omega t = E(r,t)\left( r-\frac{d}{2} \right)
\end{eqnarray}
よって
\begin{eqnarray}
  E(r,t) = \frac{v_0\sin\omega t}{r-d/2}
\end{eqnarray}
となる.

Maxwell 方程式より
\begin{eqnarray}
  \nabla\times \bm{B} = \mu_0 \bm{j} + \mu_0\epsilon_0 \frac{\partial \bm{E}}{\partial t}
\end{eqnarray}
であり,空間内に発生する電流は$0$であるため,
\begin{eqnarray}
  \nabla\times \bm{B} =   \mu_0\epsilon_0 \frac{\partial \bm{E}}{\partial t}
\end{eqnarray}
となり,対称性から,図のように経路を取ると
\begin{eqnarray}
  B(t)\cdot 2\pi r = \mu_0 \epsilon_0 \frac{\partial}{\partial t}\int_S \bm{E}\cdot d\bm{S} = \mu_0 \epsilon_0 E'(r,t) \pi r^2 =  \mu_0 \epsilon_0\pi r^2 \frac{v_0 \omega \cos\omega t}{r-d/2}
\end{eqnarray}
となる.よって,
\begin{eqnarray}
  B(r) = \frac{r}{2}\mu_0\epsilon_0 \frac{v_0\omega\cos\omega t}{r-d/2}
\end{eqnarray}
となる.
\subsection*{\large{[3.1]}}
回路の方程式より半時計周りを正とすると
\begin{eqnarray}
  0 = \frac{Q(t)}{C} + L \frac{dI}{dt}
\end{eqnarray}
であり,これを$t$でもう一度微分すると
\begin{eqnarray}
  0 = \frac{I}{C} + L \frac{d^2 I}{dt^2}
\end{eqnarray}
となるため,角周波数$\omega = \frac{1}{\sqrt{LC}}$の単振動になる.実際に初期条件から解くと
\begin{eqnarray}
  I = -\omega CV sin(\omega t),\quad Q = CV \cos(\omega t)
\end{eqnarray}
となる.
\subsection*{\large{[3.2]}}
それぞれのエネルギーは
\begin{eqnarray}
  E_C = \frac{1}{2}\frac{Q^2}{C} = \frac{1}{2}CV^2 \cos^2 (\omega t),\quad E_L = \frac{1}{2}L\omega^2 C^2 V^2 \sin^2 (\omega t) =  \frac{1}{2}CV^2 \sin^2 (\omega t)
\end{eqnarray}
となる.よってそれぞれの和は初期状態のエネルギーに一致する.
\subsection*{\large{[3.3]}}
\begin{eqnarray}
  C = \epsilon_0 \frac{S}{d}
\end{eqnarray}
より,$C$が2倍になる.
\newpage
\section*{\Large{第三問}}
\subsection*{\large{[1]}}
それぞれ固有方程式を解いて
\begin{eqnarray}
  &&\sigma_x:\quad  \lambda=1: 
  \frac{1}{\sqrt{2}}(\alpha + \beta),\quad \lambda=-1:\frac{1}{\sqrt{2}}(\alpha - \beta)\\
  &&\sigma_y:\quad  \lambda=1: 
  \frac{1}{\sqrt{2}}(\alpha +i \beta),\quad \lambda=-1:\frac{1}{\sqrt{2}}(\alpha -i \beta)
\end{eqnarray}
\subsection*{\large{[2.1]}}
代入すると$x$軸方向しかないことに注意して,
\begin{eqnarray}
  H_{SO}= -\left(  \frac{\gamma E}{\hbar}\right)p_x \sigma_y
\end{eqnarray}
である.
\subsection*{\large{[2.2]}}
$\chi = \uparrow,\downarrow$であり,$\sigma_y$の固有状態は求まっているため
\begin{eqnarray}
  \psi_1 = e^{ikx}\frac{1}{\sqrt{2}}(\alpha +i \beta), \quad \psi_2 = e^{ikx}\frac{1}{\sqrt{2}}(\alpha -i \beta)
\end{eqnarray}
それぞれの固有値は
\begin{eqnarray}
  E_1= \frac{\hbar^2k^2}{2m} - \left(  \frac{\gamma E}{\hbar}\right)\hbar k , \quad E_2 = \frac{\hbar^2k^2}{2m} + \left(  \frac{\gamma E}{\hbar}\right)\hbar k 
\end{eqnarray}
\subsection*{\large{[2.3]}}
ハミルトニアンは
\begin{eqnarray}
  \frac{\hbar^2 k^2}{2m} - \left(  \frac{\gamma E}{\hbar}\right)\hbar\sigma_y + \frac{1}{2}g\mu_B B\sigma_x=
  \begin{bmatrix}
    \frac{\hbar^2 k^2}{2m} & - i\left(  \frac{\gamma E}{\hbar}\right)\hbar + \frac{1}{2}g\mu_B B\\
     i\left(  \frac{\gamma E}{\hbar}\right)\hbar + \frac{1}{2}g\mu_B B & \frac{\hbar^2 k^2}{2m}
  \end{bmatrix}
\end{eqnarray}
これを対角化して固有エネルギーを求めると
\begin{eqnarray}
  E_{\pm} = \frac{\hbar^2 k^2 }{2m} \pm \sqrt{\left(\frac{1}{2}g\mu_B B\right)^2 + \left(\gamma Ek \right)^2 }
\end{eqnarray}

\newpage

\section*{\Large{第四問}}
\subsection*{\large{[1]}}
エネルギーが一定であるため
\begin{eqnarray}
  E = \frac{1}{2}m\left( v_{x,1}^2 + v_{y,1}^2 + v_{x,2}^2 + v_{y,2}^2 \right)
\end{eqnarray}
であり,このミクロカノニカル分布はこれらの4つの自由度を持ち,上記を満たすような分布である.よって,その状態数$W$は
\begin{eqnarray}
  W &&= \# \left[ v_{x,1},v_{y,1},v_{x,2},v_{y,2} : E = \frac{1}{2}m\left( v_{x,1}^2 + v_{y,1}^2 + v_{x,2}^2 + v_{y,2}^2 \right) \right]\\
  &&=\left( 半径 \sqrt{\frac{2E}{m}}の4次元球の表面積 \right)
\end{eqnarray}

\subsection*{\large{[2]}}
2次元球(円)の体積は図より
\begin{eqnarray}
  \int_{-r}^{r} dq_1 2\sqrt{r^2-q_1^2}
\end{eqnarray}
で求まる.ここで,2次元球の表面積はこの体積を$r$で微分したものだから
\begin{eqnarray}
  S_2(r) = \int_{-r}^{r} dq_1 \frac{2r}{\sqrt{r^2-q_1^2}}
\end{eqnarray}
である.同様にして3次元球の体積は$q_1=x,q_2=y$として見立てることによって
\begin{eqnarray}
  \int_{-r}^{r}dq_1 \int_{-\sqrt{r^2-q_1^2}}^{\sqrt{r^2-q_1^2}}dq_2 2\sqrt{r^2-q_1^2-q_2^2}
\end{eqnarray}
である.これを$r$で微分することによって$S_3(r)$を導ける.同様にして$S_4(r)$も導ける.
\subsection*{\large{[3]}}
$v_1$を固定すると全体の系は
\begin{eqnarray}
  E = \frac{1}{2}m \left( v_1^2 + v_{x,2}^2 + v_{y,2}^2 \right)
\end{eqnarray}
であり,幅$dv_1$を持つときの状態数は
\begin{eqnarray}
  W'  &&= \# \left[  v_{x,2},v_{y,2}: \frac{2E}{m}-v_1^2 = v_{x,2}^2 + v_{y,2}^2 \right]\\
  &&=\left( 半径\sqrt{ \frac{2E}{m}-v_1^2 }の2次元球の表面積 \right)\cdot dv_1
\end{eqnarray}
であるため,等重率の原理から
\begin{eqnarray}
  P(v_1) = \frac{S_2\left( \sqrt{ \frac{2E}{m}-v_1^2 } \right)}{S_4\left( \sqrt{\frac{2E}{m} } \right)}= \frac{2\pi \sqrt{ \frac{2E}{m}-v_1^2 } }{ \pi \left( \sqrt{\frac{2E}{m}} \right)^3 }
\end{eqnarray}
である.
\subsection*{\large{[4]}}
同様にして,この系の速度を固定しなかったときの全体の状態数は
\begin{eqnarray}
  W_{N} &&= \# \left[ v_{x,1},v_{x,2},\cdots: N\epsilon = \frac{1}{2}m \left( v_{x,1}^2 +v_{x,2}^2 + \cdots  \right) \right] \\
  &&= \left( 半径 \sqrt{\frac{2N\epsilon}{m}}の2N次元球の表面積 \right) = S_{2N}\left( \sqrt{\frac{2N\epsilon}{m}} \right)
\end{eqnarray}
である.一方,固定した時は
\begin{eqnarray}
  W'_{N} &&= \# \left[ v_{y,1},v_{y,2},\cdots: N\epsilon = \frac{2N\epsilon}{m}-v_1^2= \left( v_{y,1}^2 +v_{y,2}^2 + \cdots  \right) \right] \\
  &&= \left( 半径 \sqrt{\frac{2N\epsilon}{m}-v_1^2}の2N-2次元球の表面積 \right) = S_{2N-2}\left( \sqrt{\frac{2N\epsilon}{m}-v_1^2} \right)
\end{eqnarray}
よって同様にして考えることによって
\begin{eqnarray}
  P_N(v_1) = \frac{W'_{N}}{W_{N}}= \frac{S_{2N-2}\left( \sqrt{\frac{2N\epsilon}{m}-v_1^2} \right)}{S_{2N}\left( \sqrt{\frac{2N\epsilon}{m}} \right)}
\end{eqnarray}
となる.ここで,規格化因子を無視するとこれらの項は
\begin{eqnarray}
  P_N(v_1) \propto \frac{\left( \sqrt{\frac{2N\epsilon}{m}-v_1^2} \right)^{2N-3}}{\left( \sqrt{\frac{2N\epsilon}{m}} \right)^{2N-1}} = \left( 1- \frac{mv_1^2}{2N\epsilon} \right)^N \frac{m}{\left( 1-\frac{mv_1^2}{2N\epsilon} \right)^{3/2}2N\epsilon}\to \exp\left( -\frac{mv_1^2}{2\epsilon} \right)\frac{m}{2N\epsilon}
\end{eqnarray} 
である.よってmaxwellボルツマン分布になることが確かめられた.
\subsection*{\large{[4]}}
よって,
\begin{eqnarray}
  \frac{N\epsilon}{k_B T} = \frac{mv_1^2}{2}\frac{1}{\epsilon}
\end{eqnarray}
であるため,
\begin{eqnarray}
  \epsilon = \frac{N\epsilon}{k_B T} \frac{2}{mv_1^2}
\end{eqnarray}
\newpage
\section*{\Large{第五問}}
\subsection*{\large{[1]}}
ガウスの法則より
\begin{eqnarray}
  4\pi r^2 D(r) = 4\pi a^2 \sigma ,\quad\therefore D(r) = \sigma\frac{a^2}{r^2}
\end{eqnarray}
\subsection*{\large{[2]}}
この場合は空間に電流が流れないため
\begin{eqnarray}
  \nabla\times \bm{H} = \frac{\partial\bm{D}}{\partial t}
\end{eqnarray}
\subsection*{\large{[3]}}
上式を図中の$C$を囲む平面に関して積分をして,Stokesの定理より
\begin{eqnarray}
  \oint_C \bm{H}\cdot d\bm{l} = \frac{\partial}{\partial t} \int_S \bm{D}\cdot d\bm{S}
\end{eqnarray}
\subsection*{\large{[4]}}
$\bm{D}\cdot d\bm{S}$を求める.立体角をうまく使ったからできるのか?しっかり調べておく必要があるが今回は愚直に計算していく.
\begin{eqnarray}
  \int_S \bm{D}d\bm{S} = \int_0^{2\pi}d\psi \int_{0}^{d}dr' D(r\cos\phi\cos\theta)\cdot \cos\phi r'
\end{eqnarray}
である.ここで,
\begin{eqnarray}
  r' = r\cos\theta \tan\phi,\quad dr' = r\cos\theta \frac{1}{\cos\phi^2}d\phi
\end{eqnarray}
より,
\begin{eqnarray}
  \int_S \bm{D}d\bm{S} &&= 2\pi \int_{0}^{\phi=\theta}  r\cos\theta \frac{1}{\cos\phi^2}d\phi \sigma\frac{a^2}{(r\cos\phi\cos\theta)^2} \cos\phi  r\cos\theta \tan\phi\\
  &&=2\pi\int_0^{\theta} a^2\sigma \frac{1}{\cos^4\phi}\sin\phi d\phi  = \Psi
\end{eqnarray}
となる.よって,
\begin{eqnarray}
  d\Psi = \frac{2\pi a^2\sigma}{\cos^4\theta}d\theta
\end{eqnarray}
また,
\begin{eqnarray}
  x-vt = \frac{d}{\tan\theta}
\end{eqnarray}
であるため,
\begin{eqnarray}
  -vdt = -\frac{d}{\sin^2\theta}d\theta,\quad\therefore d\theta = -\frac{v\sin^2\theta}{d}dt
\end{eqnarray}
となることから.
\begin{eqnarray}
  \frac{d\Psi}{d t} = \frac{2\pi a^2 v\sigma}{d\cos^4\theta}\sin^2\theta = \frac{2\pi a^2 v\sigma}{r\cos^4\theta}\sin\theta, \quad\because d = r\sin\theta
\end{eqnarray}
\subsection*{\large{[5]}}
よって,
\begin{eqnarray}
  H \cdot 2\pi d = \frac{d\Psi}{d t} = \frac{2\pi a^2 v\sigma}{r\cos^4\theta}\sin\theta
\end{eqnarray}
より,
\begin{eqnarray}
  H = \frac{a^2 v\sigma}{r\cos^4\theta}\sin\theta
\end{eqnarray}
\subsection*{\large{[6]}}
\begin{eqnarray}
  u(r.t) = \frac{\mu_0}{2}H^2
\end{eqnarray}
\subsection*{\large{[7]}}
全空間で積分すると
\begin{eqnarray}
  U &&= 2\pi \int_a^{\infty}dr \int_0^{\pi} \frac{\mu_0}{2}H^2 = \pi \mu_0 a^4v^2\sigma^2 \int_a^{\infty}dr\frac{1}{r^2} \int_0^{\pi} \frac{1}{\cos^8\theta}\sin\theta^2 \sin\theta d\theta\\
  &&=\pi \mu_0 a^4v^2\sigma^2 \frac{1}{a}\frac{4}{35} = \frac{4}{35} \mu_0 a^3 v^2\sigma^2
\end{eqnarray}
となる.
\subsection*{\large{[8]}}
\begin{eqnarray}
  \frac{1}{2}mv^2 + \frac{4}{35} \mu_0 a^3 v^2\sigma^2 = \frac{1}{2}\left( m+\frac{8}{35} \mu_0 a^3 \sigma^2 \right)v^2
\end{eqnarray}
となるため,質量が重くなると考えられる.この理由は電磁誘導が起きて帯電球が動きにくくなるから.

\end{document}