%%%%%%%%%%%%%%%%%%%%%%%%%%%%%%%%%%%%%%%%%%%%%%%%%%%%%%%
%                   File: OSAmeetings.tex             %
%                  Date: 29 Novemver 2018              %
%                                                     %
%     For preparing LaTeX manuscripts for submission  %
%       submission to OSA meetings and conferences    %
%                                                     %
%       (c) 2018 Optical Society of America           %
%%%%%%%%%%%%%%%%%%%%%%%%%%%%%%%%%%%%%%%%%%%%%%%%%%%%%%%

\documentclass[12pt,dvipdfmx]{jsarticle}
%% if A4 paper needed, change letterpaper to A4
\usepackage[dvipdfmx]{graphicx}
\usepackage[dvipdfmx]{color}
\usepackage{osameet3} %% use version 3 for proper copyright statement
\usepackage{ascmac}
%% provide authormark
\newcommand\authormark[1]{\textsuperscript{#1}}

%% standard packages and arguments should be modified as needed
\usepackage{amsmath,amssymb}
\usepackage[colorlinks=true,bookmarks=false,citecolor=blue,urlcolor=blue]{hyperref} %pdflatex
%\usepackage[breaklinks,colorlinks=true,bookmarks=false,citecolor=blue,urlcolor=blue]{hyperref} %latex w/dvipdf
\usepackage{mathtools}
\usepackage{amsmath}
\usepackage{empheq}
\usepackage{physics}
\usepackage[scr=rsfs]{mathalpha}
\usepackage[svgnames]{xcolor}% tikzより前に読み込む必要あり
\usepackage{tikz}
\usepackage{bm}
\usepackage{here}
\usepackage{braket}
\usepackage{framed,color}
\usepackage{dcolumn}
\definecolor{shadecolor}{gray}{0.80}
\usetikzlibrary{perspective}
\tikzset
{%
  my ball/.style={draw,circle,minimum size=2*\r cm,inner sep=0,shading=ball,ball color=cyan!50!blue,opacity=#1},
  my ball/.default=1,
  hidden line/.style={black!60}
}
\begin{document}
\title{東大物理工学科 2016}

\author{21B00817 鈴木泰雅,\authormark{1}}
\section*{\Large{第一問}}
\subsection*{\large{[1.1]}}
運動方程式より
\begin{eqnarray}
  0 = kv_0t_0 -\mu mg, \quad t_0 = \frac{\mu mg}{kv_0}
\end{eqnarray}
である.
\subsection*{\large{[1.2]}}
物体Bが右方向に動くという仮説を立てる.($\dot{x}_B>$)ここで運動方程式は
\begin{eqnarray}
  m\ddot{x}_B = k(v_0(t+t_0)-x_B)-\frac{2}{3}\mu mg -k x_B
\end{eqnarray}
となる.
\subsection*{\large{[1.3]}}
\begin{eqnarray}
  m\ddot{x}_B = -2k \left( x_B -\frac{v_0 (t+t_0)}{2}+ \frac{\mu mg}{3k} \right)
\end{eqnarray}
であり,
\begin{eqnarray}
  X_B =  x_B -\frac{v_0 (t+t_0)}{2}+ \frac{\mu mg}{3k} = x_B -\frac{v_0 t}{2} -\frac{\mu mg}{6k}
\end{eqnarray}
とすると
\begin{eqnarray}
  m\ddot{X}_B = -2k X_B,\quad X_B = A\cos\left( \sqrt{\frac{2k}{m}}t \right) + B\sin\left( \sqrt{\frac{2k}{m}}t \right)
\end{eqnarray}
であり,
\begin{eqnarray}
  &&X_B(0)= 0 - \frac{v_0 t_0}{2} + \frac{\mu mg}{3k}= -\frac{\mu mg}{6k}\\
  &&\dot{X}_B(0) = 0 -\frac{v_0}{2} = -\frac{v_0}{2}
\end{eqnarray}
であるため,
\begin{eqnarray}
  &&X_B = -\frac{\mu mg}{6k}\cos\left( \sqrt{\frac{2k}{m}}t \right) -\frac{v_0}{2}\sqrt{\frac{m}{2k}}\sin\left( \sqrt{\frac{2k}{m}}t \right)\\
  &&\therefore x_B = \frac{v_0}{2}\left( t-\sqrt{\frac{m}{2k}}\sin\left( \sqrt{\frac{2k}{m}}t \right) \right) + \frac{\mu mg}{6k}\left( 1- \cos\left( \sqrt{\frac{2k}{m}}t \right)\right)
\end{eqnarray}
である.
\subsection*{\large{[1.4]}}
\begin{eqnarray}
  \sqrt{\frac{2k}{m}}t_0 \ll 1
\end{eqnarray}
であるため,
\begin{eqnarray}
  \sin\left( \sqrt{\frac{2k}{m}}t_0 \right) \approx \sqrt{\frac{2k}{m}}t_0 + \mathcal{O}\left( \sqrt{\frac{2k}{m}}t_0 \right)^3, \quad \cos\left( \sqrt{\frac{2k}{m}}t_0 \right) \approx 1 -\frac{1}{2}\left( \sqrt{\frac{2k}{m}}t_0 \right)^2  + \mathcal{O}\left( \sqrt{\frac{2k}{m}}t_0 \right)^4
\end{eqnarray}
よって二次までの近似をすると
\begin{eqnarray}
  x_B(t=t_0) = \frac{v_0}{2}\left( t_0 - \sqrt{\frac{m}{2k}}\sqrt{\frac{2k}{m}}t_0 \right) + \frac{\mu mg}{6k}\left( 1 -1 + \frac{1}{2}\left( \sqrt{\frac{2k}{m}}t_0 \right)^2 \right) =\frac{t_0^3 kv_0}{6m}
\end{eqnarray}
となり,これは物体Aが静止するために
\begin{eqnarray}
  k\frac{t_0^3 kv_0}{6m} < \mu mg = t_0 k v_0 ,\quad\therefore \frac{t_0^2}{6m}k<1
\end{eqnarray}
であることを示せばよい.解けない??
\subsection*{\large{[2.1]}}
ラグランジュ方程式を解く.ラグランジアンは
\begin{eqnarray*}
  L &&= \frac{1}{2}m\left[ \dot{q}_A^2+\dot{q}_B^2+\dot{q}_C^2 \right] \\
  &&\quad -k_1l^2 \left[ -\frac{1}{2}\left( \frac{l+q_C}{l} \right)^2 - \frac{1}{2}\left( \frac{q_B}{l} \right)^2- \frac{1}{2}\left( \frac{-l+q_A}{l} \right)^2 + \frac{1}{4}\left( \frac{l+q_C}{l} \right)^4+ \frac{1}{4}\left( \frac{q_B}{l} \right)^4 + \frac{1}{4}\left( \frac{-l+q_A}{l} \right)^4 \right]\\
  &&\quad -\frac{1}{2}k_0\left[ (q_C-q_B)^2 + (q_B-q_A)^2 \right]
\end{eqnarray*}
であるためそれぞれの方程式は一次の形まで書くと
\begin{eqnarray*}
  m\ddot{q}_A &&= -k_1l^2 \left[ -\left( \frac{-l+q_A}{l} \right)\frac{1}{l}+ \left( \frac{-l+q_A}{l} \right)^3 \frac{1}{l} \right]+k_0(q_B-q_A)\\
  &&\approx -k_1l^2 \left[ \frac{1}{l}+ \frac{2q_A}{l^2} \right] + k_0\left( q_B-q_A \right)\\
  m\ddot{q}_B &&= -k_1l^2 \left[ -\left( \frac{q_B}{l} \right)\frac{1}{l}+ \left( \frac{q_B}{l} \right)^3 \frac{1}{l}\right]-k_0\left\{ -(q_C-q_B) + (q_B-q_A) \right\}\\
  &&\approx k_1 q_B -2k_0 q_B + k_0 q_C + k_0q_A\\
  m\ddot{q}_C &&\approx -k_1l^2 \left[ \frac{1}{l}+ \frac{2q_C}{l} \right] - k_0(q_C-q_B)
\end{eqnarray*}
である.よってこられを行列で表現すると
\begin{eqnarray}
  m
  \begin{bmatrix}
   \ddot{q}_A\\
   \ddot{q}_B\\ 
   \ddot{q}_C
  \end{bmatrix}
  =
  \begin{bmatrix}
    -2k_1-k_0 & k_0 & 0\\
    k_0 & k_1-2k_0 & k_0\\
    0 & k_0 & -2k_1-k_0
  \end{bmatrix}
  \begin{bmatrix}
    q_A\\
    q_B\\
    q_C
  \end{bmatrix}
  + 
  \begin{bmatrix}
    -k_1l\\
    0\\
    k_1l
  \end{bmatrix}
\end{eqnarray}
となる.ここで,一般的に
\begin{eqnarray}
  m\ddot{\bm{q}} = A\bm{q} + B
\end{eqnarray}
があり対角化して
\begin{eqnarray}
  \bm{q}' = U\bm{q}
\end{eqnarray}
とすると
\begin{eqnarray}
  m\ddot{\bm{q}}' = U^{-1}AU \bm{q}' + U^{-1}B
\end{eqnarray}
であり,例えば一つの成分を取り出して
\begin{eqnarray}
  m\ddot{q}_i' = a q_i' + b = a\left( q_i' + \frac{b}{a} \right)
\end{eqnarray}
であり,この基準振動$\omega$が満たす方程式は
\begin{eqnarray}
  -\omega^2 = \frac{a}{m}
\end{eqnarray}
となるためこれは$b$に依存しない.よって,行列$A$のみの対角化をすればよい.よって,この行列の固有方程式は
\begin{eqnarray*}
  \det
  \begin{bmatrix}
    -2k_1-k_0-\lambda & k_0 & 0\\
    k_0 & k_1-2k_0-\lambda & k_0\\
    0 & k_0 & -2k_1-k_0-\lambda
  \end{bmatrix}
  &&= \left( -2k_1-k_0-\lambda \right)\left( \lambda^2+(k_1+3k_0)\lambda + 3k_0k_1 -2k_1^2 \right)\\
  &&=0
\end{eqnarray*}
であるため固有値は
\begin{eqnarray}
  \lambda = -2k_1-k_0, \frac{1}{2}\left[ -(k_1+3k_0) \pm \sqrt{ 9k_1^2-6k_1k_0 + 9k_0^2 } \right]
\end{eqnarray}
である.よって固有振動数はこれに$1/m$倍して$-1$をかけて平方したものであるため
\begin{eqnarray}
  \omega = \sqrt{ \frac{2k_1+k_0}{m} }, \sqrt{ \frac{1}{2m}\left( k_1+3k_0 \mp \sqrt{ 9k_1^2-6k_1k_0 + 9k_0^2 } \right) }
\end{eqnarray}
である.
\subsection*{\large{[2.2]}}
不安定な解はこの固有振動が虚数数の時であり,時刻に対して指数関数的に増大する.
\begin{eqnarray}
  9k_1^2-6k_1k_0 + 9k_0^2 = 9k_0^2\left[ \left( \frac{k_1}{k_0} \right)^2 -\frac{2}{3}\left( \frac{k_1}{k_0} \right)+1 \right]
\end{eqnarray}
は任意の$k_1/k_0$で正の値であるが,
\begin{eqnarray}
  k_1+3k_0 - \sqrt{ 9k_1^2-6k_1k_0 + 9k_0^2 } <0
\end{eqnarray}
の時この固有振動は不安定になる.よって,
\begin{eqnarray}
  9k_1^2-6k_1k_0 + 9k_0^2 > \left( k_1+3k_0\right)^2,\quad\therefore 2k_1^2-3k_0k_1>0
\end{eqnarray}
である.これは
\begin{eqnarray}
  \frac{k_1}{k_0}>\frac{3}{2}
\end{eqnarray}
よって,$k_C$の値は
\begin{eqnarray}
  k_C = \frac{3}{2}k_0
\end{eqnarray}
である.
\subsection*{\large{[2.3]}}
\begin{eqnarray}
  k_1 = \frac{2}{3}k_0
\end{eqnarray}
であるためそれぞれ固有振動数に代入すると
\begin{eqnarray}
  \sqrt{ \frac{7k_0}{3m} }, \sqrt{ \frac{k_0}{3m} }, \sqrt{ \frac{10k_0}{3m} }
\end{eqnarray}
である.それぞれの固有ベクトルは
\begin{eqnarray}
  -\frac{7k_0}{3}の時:
  \begin{bmatrix}
    1\\
    0\\
    -1
  \end{bmatrix},\quad -\frac{k_0}{3}の時:
  \begin{bmatrix}
    1\\
    2\\
    1
  \end{bmatrix},\quad -\frac{10k_0}{3}の時:
  \begin{bmatrix}
    -1\\
    1\\
    -1
  \end{bmatrix}
\end{eqnarray}
であり,行列$U$は
\begin{eqnarray}
  \begin{bmatrix}
    1 & 1 & -1\\
    0 & 2 & 1\\
    -1 & 1 & -1
  \end{bmatrix},\quad U^{-1}AU = 
  \begin{bmatrix}
    -\frac{7k_0}{3} & 0 & 0\\
    0 & -\frac{k_0}{3} & 0\\
    0 & 0 & -\frac{10k_0}{3}
  \end{bmatrix}
\end{eqnarray}
であり,
\begin{eqnarray}
  U^{-1}B = 
  \begin{bmatrix}
    1/2 & 0& -1/2\\
    1/6 & 1/3 & 1/6\\
    -1/3 & 1/3 & -1/3
  \end{bmatrix}
  \begin{bmatrix}
    -k_1 l\\
    0 \\
    k_1l
  \end{bmatrix}
  =
  \begin{bmatrix}
    -k_1l\\
    0\\
    0
  \end{bmatrix}
\end{eqnarray}
となるため,$\bm{q}'$は$\bm{q}$の成分の線形結合であり,初期値は$\bm{q}(0)=\dot{\bm{q}}(0)=0$であるため,$\bm{q}'(0)=\dot{\bm{q}}'(0)=0$である.よって解いて
\begin{eqnarray}
  \bm{q}'=
  \begin{bmatrix}
    k_1l\left( 1-\cos\left( \sqrt{\frac{7k_0}{3m}}t \right) \right)\\
    0\\
    0
  \end{bmatrix}
\end{eqnarray}
となる.よって,元の座標に戻ると
\begin{eqnarray}
  \bm{q} =
  \begin{bmatrix}
    q_A \\
    q_B \\
    q_C
  \end{bmatrix}
  =U^{-1} \bm{q}'=
  \begin{bmatrix}
    1/2\\
    1/6\\
    -1/3
  \end{bmatrix}
  k_1l\left( 1-\cos\left( \sqrt{\frac{7k_0}{3m}}t \right) \right)
\end{eqnarray}
であるため,
\begin{eqnarray}
  q_A:q_B:q_C = 1/2:1/6:-1/3
\end{eqnarray}
である.
\newpage
\section*{\Large{第二問}}

\subsection*{\large{[1]}}
\begin{eqnarray}
  \phi(P) = \frac{-q}{4\pi\epsilon_0}\frac{1}{r_-} + \frac{q}{4\pi\epsilon_0}\frac{1}{r_+}
\end{eqnarray}
\subsection*{\large{[2]}}
\begin{eqnarray}
  \phi(P)&&= \frac{-q}{4\pi\epsilon}\frac{1}{r+d/2\cos\theta} + \frac{-q}{4\pi\epsilon}\frac{1}{r-d/2\cos\theta} \\
  &&\approx \frac{q}{4\pi\epsilon_0}\frac{1}{r}\left\{ -1 + \frac{d}{2r}\cos\theta + 1 + \frac{d}{2r}\cos\theta \right\} = \frac{p}{4\pi\epsilon_0}\frac{1}{r^2}\cos\theta
\end{eqnarray}
となる.
\subsection*{\large{[3]}}
\begin{eqnarray}
  \bm{E} &&= -\nabla\phi(P) = \left( \bm{e}_r\frac{\partial}{\partial r} + \bm{e}_{\theta}\frac{1}{r}\frac{\partial}{\partial\theta} \right)\frac{p}{4\pi\epsilon_0}\frac{1}{r^2}\cos\theta\\
  &&= \frac{p}{4\pi\epsilon_0r^3}\left( 2\cos\theta\bm{e}_r + \sin\theta\bm{e}_\theta \right)
\end{eqnarray}
である.
\subsection*{\large{[4]}}
$\theta=\pi/2$の時,
\begin{eqnarray}
  \bm{E} = \frac{p}{4\pi\epsilon_0 r^3}\bm{e}_\theta
\end{eqnarray}
であるため,$r$成分は持たない.$\theta_2 = 0$の時,それぞれ$\mp p$の電荷は
\begin{eqnarray}
  -q : \bm{F}_- &&= \frac{p}{4\pi\epsilon_0 (l-d/2)^3}\bm{e}_\theta\\
  q : \bm{F}_+ &&= \frac{p}{4\pi\epsilon_0 (l+d/2)^3}\bm{e}_\theta
\end{eqnarray}
であるため,$F_->F_+$であるから,反時計回りに回転し始める.一方.$\theta=-\pi/2$の時はそれぞれの$p_1$からの距離は等しく,($l$が十分に大きいため),$\theta$方向にしか力がかからないため,回転しない.
\subsection*{\large{[5]}}
$\bm{p}_2$に作る電場$\bm{E}$は
\begin{eqnarray}
  \bm{E}_1 = \frac{p_1}{4\pi\epsilon_0l^3} \left( 2\cos\theta_1\bm{e}_r + \sin\theta_1\bm{e}_\theta \right)
\end{eqnarray}
であり,書き込んだ図より
\begin{eqnarray}
  \bm{p}_2\cdot\bm{e}_r =p_2 \cos\theta_2, \quad \bm{p}_2\cdot\bm{e}_\theta = -p_2\sin\theta_2
\end{eqnarray}
である.よって
\begin{eqnarray}
  U = -\bm{p}_2\cdot\bm{E}_1 = -\frac{p_1p_2}{4\pi\epsilon_0l^3} \left( 2\cos\theta_1\cos\theta_2- \sin\theta_1\sin\theta_2\right)
\end{eqnarray}
となる.
\newpage
\section*{\Large{第三問}}
\subsection*{\large{[1]}}

シュレディンガー方程式は
\begin{eqnarray}
  -\frac{\hbar^2}{2m}\psi''(x) = E\psi(x)
\end{eqnarray}
であり,この解は
\begin{eqnarray}
  \psi(x) = A\sin(\lambda x) + B\cos(\lambda x),\quad \lambda = \frac{\sqrt{2mE}}{\hbar}
\end{eqnarray}
である.
ここで,$x=\pm a$で$\psi$の時,
\begin{eqnarray}
  A = 0 , \text{or} , B=0
\end{eqnarray}
である.よって,
\begin{eqnarray}
  \psi(x) = A\cos(\lambda x),\quad B\sin(\lambda x)
\end{eqnarray}
となり,$\psi(\lambda a)=0$より,
\begin{eqnarray}
  \lambda a = \frac{\pi n_{\text{odd}}}{2}, \quad \lambda a= \frac{\pi n_{\text{even}}}{2}
\end{eqnarray}
である.よって,
\begin{eqnarray}
  \psi(x) = A\cos\left( \frac{\pi n_{\text{odd}}}{2a} x \right), \quad B\sin\left( \frac{\pi n_{\text{even}}}{2a} x \right)
\end{eqnarray}
である.ここで,
\begin{eqnarray}
  \int_{-a}^{a} \cos^2\left( \frac{\pi n_{\text{odd}}}{2a} x \right)dx = \int_{-a}^{a} \sin^2\left( \frac{\pi n_{\text{odd}}}{2a} x \right)dx= a
\end{eqnarray}
であるため,
\begin{eqnarray}
  \psi(x) = \sqrt{a}\cos\left( \frac{\pi n_{\text{odd}}}{2a} x \right), \quad \sqrt{a}\sin\left( \frac{\pi n_{\text{even}}}{2a} x \right)
\end{eqnarray}
であり,
\begin{eqnarray}
  \frac{\sqrt{2mE}}{\hbar}= \frac{\pi n}{2a}, \quad E = \frac{\hbar^2 \pi^2 n^2}{8ma^2}
\end{eqnarray}
である.
\subsection*{\large{[2]}}
\begin{eqnarray}
  \Psi(x_1,x_2) &&= \frac{1}{\sqrt{2}}\left( g(x_1)e(x_2)-g(x_2)e(x_1) \right),\\
  &&g(x_1)g(x_2), \quad e(x_1)e(x_2), \quad \frac{1}{\sqrt{2}}\left( g(x_1)e(x_2)+g(x_2)e(x_1) \right)
\end{eqnarray}
\subsection*{\large{[3]}}
$s=1/2$より,$m=\pm 1/2$である.よって,
\begin{eqnarray}
  S=1, \quad M = \pm 1, 0,\qquad S=0, M=0
\end{eqnarray}
であるため,同時固有状態と全スピンは
\begin{eqnarray}
  S=1, &&\quad |\uparrow\uparrow\rangle, \frac{1}{\sqrt{2}}\left( |\uparrow\downarrow\rangle+ |\downarrow\uparrow\rangle\right), |\downarrow\downarrow\rangle\\
  S=0, &&\quad  \frac{1}{\sqrt{2}}\left( |\uparrow\downarrow\rangle- |\downarrow\uparrow\rangle\right)
\end{eqnarray}
である.
\subsection*{\large{[4]}}

\newpage
\section*{\Large{第四問}}
\subsection*{\large{[1]}}

\begin{eqnarray}
  X = a\left( n^{+}-n^{-} \right)
\end{eqnarray}

であり,$N=n^{+}+n^{-}$であるため,
\begin{eqnarray}
  X = a\left( N-2n^{-} \right)
\end{eqnarray}
であり,$n^{-}$と$X$は一対一対応する.よって,$n^{-}$の選び方が全通りであり,状態数は
\begin{eqnarray}
  W = \frac{N!}{(N-n^{-})!n^{-}!}
\end{eqnarray}
であり,
\begin{eqnarray}
  N-2n^{-}=n, \quad\therefore n^{-}=\frac{N-n}{2}
\end{eqnarray}
の時の状態数は
\begin{eqnarray}
  W = \frac{N!}{(\frac{N-n}{2})!(\frac{N+n}{2})!},\quad\therefore P(n) = \frac{(\frac{N-n}{2})!(\frac{N+n}{2})!}{N!}
\end{eqnarray}
エントロピーは
\begin{eqnarray}
  S = k_B\log W \approx k_B\left( N\log N - \frac{N-n}{2}\log\left( \frac{N-n}{2}\right)-\frac{N+n}{2}\log\left( \frac{N+n}{2}\right) \right)
\end{eqnarray}
となる.

\subsection*{\large{[2]}}
$X$を固定することは$n$を固定することである.また全体のエネルギーは$0$であるため自由エネルギーは
\begin{eqnarray}
  F = -TS = -k_BT\left( N\log N - \frac{N-X/a}{2}\log\left( \frac{N-X/a}{2}\right)-\frac{N+X/a}{2}\log\left( \frac{N+X/a}{2}\right) \right)
\end{eqnarray}
であるため,
\begin{eqnarray}
  \tau = -k_BT\frac{1}{a}\left( \frac{NX/a}{2(N^2-(X/a)^2)} + \frac{1}{2}\log \left( \frac{N-X/a}{N+X/a} \right)\right)
\end{eqnarray}
である.ここで,$X\ll Na$の時,二次以上の項を無視するとすると
\begin{eqnarray}
  &&\log \left( \frac{N-X/a}{N+X/a} \right) = \log \left( \frac{1-X/(Na)}{1+X/(Na)} \right) \approx -\frac{X}{Na} - \frac{X}{Na} = -2\frac{X}{Na}\\
  &&\frac{NX/a}{2(N^2-(X/a)^2)} = \frac{X/(Na)}{2(1-(X/(Na))^2)}\approx \frac{X}{2Na}
\end{eqnarray}
であるため,
\begin{eqnarray}
  \tau \approx -k_B T\frac{1}{a}\left( \frac{X}{2Na}-\frac{X}{Na} \right) = k_B T \frac{X}{2Na^2}
\end{eqnarray}
となる.
\subsection*{\large{[3]}}
分配関数はそれぞれ独立な粒子からできているので
\begin{eqnarray}
  Z = z^N  = \left( \exp(-\beta\kappa)+\exp(\beta\kappa) \right)^N = (2\cosh(\beta\kappa))^N
\end{eqnarray}
であり,エネルギーの期待値は
\begin{eqnarray}
  E = -\frac{1}{Z}\frac{dZ}{d\beta} = -N\tanh(\beta\kappa)\kappa
\end{eqnarray}
また,
\begin{eqnarray}
  E = \kappa(-n^{+}+n^{-}), \quad X = a(n^{+}-n^{-}),\quad\therefore X = -\frac{a}{\kappa}E
\end{eqnarray}
よって,
\begin{eqnarray}
  X = Na \tanh(\beta\kappa)
\end{eqnarray}
である.また,$\kappa\beta \ll 1$の時,
\begin{eqnarray}
  E\approx - N\beta\kappa^2, \quad X\approx Na \beta \kappa
\end{eqnarray}
である.
\subsection*{\large{[4]}}
\begin{eqnarray}
  \frac{dE}{d\beta} = - \langle E^2 \rangle + (\langle E \rangle )^2 = -\left( \langle E^2 \rangle- (\langle E \rangle )^2 \right)
\end{eqnarray}
ゆえ,$E$の分散は
\begin{eqnarray}
  -\frac{dE}{d\beta} = N\kappa \frac{1}{\cosh^2(\beta\kappa)}
\end{eqnarray}
であり,$X$の分散は
\begin{eqnarray}
  -\frac{a}{\kappa}N\kappa \frac{1}{\cosh^2(\beta\kappa)} 
\end{eqnarray}
である.

\end{document}