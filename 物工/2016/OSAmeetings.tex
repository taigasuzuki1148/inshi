%%%%%%%%%%%%%%%%%%%%%%%%%%%%%%%%%%%%%%%%%%%%%%%%%%%%%%%
%                   File: OSAmeetings.tex             %
%                  Date: 29 Novemver 2018              %
%                                                     %
%     For preparing LaTeX manuscripts for submission  %
%       submission to OSA meetings and conferences    %
%                                                     %
%       (c) 2018 Optical Society of America           %
%%%%%%%%%%%%%%%%%%%%%%%%%%%%%%%%%%%%%%%%%%%%%%%%%%%%%%%

\documentclass[12pt,dvipdfmx]{jsarticle}
%% if A4 paper needed, change letterpaper to A4
\usepackage[dvipdfmx]{graphicx}
\usepackage[dvipdfmx]{color}
\usepackage{osameet3} %% use version 3 for proper copyright statement
\usepackage{ascmac}
%% provide authormark
\newcommand\authormark[1]{\textsuperscript{#1}}

%% standard packages and arguments should be modified as needed
\usepackage{amsmath,amssymb}
\usepackage[colorlinks=true,bookmarks=false,citecolor=blue,urlcolor=blue]{hyperref} %pdflatex
%\usepackage[breaklinks,colorlinks=true,bookmarks=false,citecolor=blue,urlcolor=blue]{hyperref} %latex w/dvipdf
\usepackage{mathtools}
\usepackage{amsmath}
\usepackage{empheq}
\usepackage{physics}
\usepackage[scr=rsfs]{mathalpha}
\usepackage[svgnames]{xcolor}% tikzより前に読み込む必要あり
\usepackage{tikz}
\usepackage{bm}
\usepackage{here}
\usepackage{braket}
\usepackage{framed,color}
\usepackage{dcolumn}
\definecolor{shadecolor}{gray}{0.80}
\usetikzlibrary{perspective}
\tikzset
{%
  my ball/.style={draw,circle,minimum size=2*\r cm,inner sep=0,shading=ball,ball color=cyan!50!blue,opacity=#1},
  my ball/.default=1,
  hidden line/.style={black!60}
}
\begin{document}

\title{東大物理工学科 2016}

\author{21B00817 鈴木泰雅,\authormark{1}}

\email{\authormark{*}suzuki.t.ec@m.titech.ac.jp} %% email address is required

\section*{\Large{第一問}}
\subsection*{\large{[1.1]}}
運動方程式より
\begin{eqnarray}
  0 = kv_0t_0 -\mu mg, \quad t_0 = \frac{\mu mg}{kv_0}
\end{eqnarray}
である.
\subsection*{\large{[1.2]}}
物体Bが右方向に動くという仮説を立てる.($\dot{x}_B>$)ここで運動方程式は
\begin{eqnarray}
  m\ddot{x}_B = k(v_0t-x_B)-\frac{2}{3}\mu mg -k x_B
\end{eqnarray}
となる.
\subsection*{\large{[1.3]}}
\begin{eqnarray}
  m\ddot{x}_B = -2k \left( x_B -\frac{v_0 t}{2}+ \frac{\mu mg}{3k} \right)
\end{eqnarray}
であり,
\begin{eqnarray}
  X_B =  x_B -\frac{v_0 t}{2}+ \frac{\mu mg}{3k} 
\end{eqnarray}
とすると
\begin{eqnarray}
  m\ddot{X}_B = -2k X_B,\quad X_B = A\cos\left( \sqrt{\frac{m}{2k}}t \right) + B\sin\left( \sqrt{\frac{m}{2k}}t \right)
\end{eqnarray}
であり,
\begin{eqnarray}
  &&X_B(0)= 0 - \frac{v_0 t_0}{2} + \frac{\mu mg}{3k}= -\frac{\mu mg}{6k}\\
  &&\dot{X}_B(0) = 0 -\frac{v_0}{2} = -\frac{v_0}{2}
\end{eqnarray}
であるため,
\begin{eqnarray}
  &&X_B = -\frac{\mu mg}{6k}\cos\left( \sqrt{\frac{m}{2k}}t \right) -\frac{v_0}{2}\sqrt{\frac{2k}{m}}\sin\left( \sqrt{\frac{m}{2k}}t \right)\\
  &&\therefore x_B = \frac{v_0}{2}\left( t-\sqrt{\frac{2k}{m}}\sin\left( \sqrt{\frac{m}{2k}}t \right) \right) + \frac{\mu mg}{6k}\left( 2- \cos\left( \sqrt{\frac{m}{2k}}t \right)\right)
\end{eqnarray}
である.

\end{document}