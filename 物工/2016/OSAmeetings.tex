%%%%%%%%%%%%%%%%%%%%%%%%%%%%%%%%%%%%%%%%%%%%%%%%%%%%%%%
%                   File: OSAmeetings.tex             %
%                  Date: 29 Novemver 2018              %
%                                                     %
%     For preparing LaTeX manuscripts for submission  %
%       submission to OSA meetings and conferences    %
%                                                     %
%       (c) 2018 Optical Society of America           %
%%%%%%%%%%%%%%%%%%%%%%%%%%%%%%%%%%%%%%%%%%%%%%%%%%%%%%%

\documentclass[12pt,dvipdfmx]{jsarticle}
%% if A4 paper needed, change letterpaper to A4
\usepackage[dvipdfmx]{graphicx}
\usepackage[dvipdfmx]{color}
\usepackage{osameet3} %% use version 3 for proper copyright statement
\usepackage{ascmac}
%% provide authormark
\newcommand\authormark[1]{\textsuperscript{#1}}

%% standard packages and arguments should be modified as needed
\usepackage{amsmath,amssymb}
\usepackage[colorlinks=true,bookmarks=false,citecolor=blue,urlcolor=blue]{hyperref} %pdflatex
%\usepackage[breaklinks,colorlinks=true,bookmarks=false,citecolor=blue,urlcolor=blue]{hyperref} %latex w/dvipdf
\usepackage{mathtools}
\usepackage{amsmath}
\usepackage{empheq}
\usepackage{physics}
\usepackage[scr=rsfs]{mathalpha}
\usepackage[svgnames]{xcolor}% tikzより前に読み込む必要あり
\usepackage{tikz}
\usepackage{bm}
\usepackage{here}
\usepackage{braket}
\usepackage{framed,color}
\usepackage{dcolumn}
\definecolor{shadecolor}{gray}{0.80}
\usetikzlibrary{perspective}
\tikzset
{%
  my ball/.style={draw,circle,minimum size=2*\r cm,inner sep=0,shading=ball,ball color=cyan!50!blue,opacity=#1},
  my ball/.default=1,
  hidden line/.style={black!60}
}
\begin{document}

\title{東大物理工学科 2016}

\author{21B00817 鈴木泰雅,\authormark{1}}

\email{\authormark{*}suzuki.t.ec@m.titech.ac.jp} %% email address is required

\section*{\Large{第一問}}
\subsection*{\large{[1.1]}}
運動方程式より
\begin{eqnarray}
  0 = kv_0t_0 -\mu mg, \quad t_0 = \frac{\mu mg}{kv_0}
\end{eqnarray}
である.
\subsection*{\large{[1.2]}}
物体Bが右方向に動くという仮説を立てる.($\dot{x}_B>$)ここで運動方程式は
\begin{eqnarray}
  m\ddot{x}_B = k(v_0(t+t_0)-x_B)-\frac{2}{3}\mu mg -k x_B
\end{eqnarray}
となる.
\subsection*{\large{[1.3]}}
\begin{eqnarray}
  m\ddot{x}_B = -2k \left( x_B -\frac{v_0 (t+t_0)}{2}+ \frac{\mu mg}{3k} \right)
\end{eqnarray}
であり,
\begin{eqnarray}
  X_B =  x_B -\frac{v_0 (t+t_0)}{2}+ \frac{\mu mg}{3k} = x_B -\frac{v_0 t}{2} -\frac{\mu mg}{6k}
\end{eqnarray}
とすると
\begin{eqnarray}
  m\ddot{X}_B = -2k X_B,\quad X_B = A\cos\left( \sqrt{\frac{2k}{m}}t \right) + B\sin\left( \sqrt{\frac{2k}{m}}t \right)
\end{eqnarray}
であり,
\begin{eqnarray}
  &&X_B(0)= 0 - \frac{v_0 t_0}{2} + \frac{\mu mg}{3k}= -\frac{\mu mg}{6k}\\
  &&\dot{X}_B(0) = 0 -\frac{v_0}{2} = -\frac{v_0}{2}
\end{eqnarray}
であるため,
\begin{eqnarray}
  &&X_B = -\frac{\mu mg}{6k}\cos\left( \sqrt{\frac{2k}{m}}t \right) -\frac{v_0}{2}\sqrt{\frac{m}{2k}}\sin\left( \sqrt{\frac{2k}{m}}t \right)\\
  &&\therefore x_B = \frac{v_0}{2}\left( t-\sqrt{\frac{m}{2k}}\sin\left( \sqrt{\frac{2k}{m}}t \right) \right) + \frac{\mu mg}{6k}\left( 1- \cos\left( \sqrt{\frac{2k}{m}}t \right)\right)
\end{eqnarray}
である.
\subsection*{\large{[1.4]}}
\begin{eqnarray}
  \sqrt{\frac{2k}{m}}t_0 \ll 1
\end{eqnarray}
であるため,
\begin{eqnarray}
  \sin\left( \sqrt{\frac{2k}{m}}t_0 \right) \approx \sqrt{\frac{2k}{m}}t_0 + \mathcal{O}\left( \sqrt{\frac{2k}{m}}t_0 \right)^3, \quad \cos\left( \sqrt{\frac{2k}{m}}t_0 \right) \approx 1 -\frac{1}{2}\left( \sqrt{\frac{2k}{m}}t_0 \right)^2  + \mathcal{O}\left( \sqrt{\frac{2k}{m}}t_0 \right)^4
\end{eqnarray}
よって二次までの近似をすると
\begin{eqnarray}
  x_B(t=t_0) = \frac{v_0}{2}\left( t_0 - \sqrt{\frac{m}{2k}}\sqrt{\frac{2k}{m}}t_0 \right) + \frac{\mu mg}{6k}\left( 1 -1 + \frac{1}{2}\left( \sqrt{\frac{2k}{m}}t_0 \right)^2 \right) =\frac{t_0^3 kv_0}{6m}
\end{eqnarray}
となり,これは物体Aが静止するために
\begin{eqnarray}
  k\frac{t_0^3 kv_0}{6m} < \mu mg = t_0 k v_0 ,\quad\therefore \frac{t_0^2}{6m}k<1
\end{eqnarray}
であることを示せばよい.解けない??
\subsection*{\large{[2.1]}}
ラグランジュ方程式を解く.ラグランジアンは
\begin{eqnarray*}
  L &&= \frac{1}{2}m\left[ \dot{q}_A^2+\dot{q}_B^2+\dot{q}_C^2 \right] \\
  &&\quad -k_1l^2 \left[ -\frac{1}{2}\left( \frac{l+q_C}{l} \right)^2 - \frac{1}{2}\left( \frac{q_B}{l} \right)^2- \frac{1}{2}\left( \frac{-l+q_A}{l} \right)^2 + \frac{1}{4}\left( \frac{l+q_C}{l} \right)^4+ \frac{1}{4}\left( \frac{q_B}{l} \right)^4 + \frac{1}{4}\left( \frac{-l+q_A}{l} \right)^4 \right]\\
  &&\quad -\frac{1}{2}k_0\left[ (q_C-q_B)^2 + (q_B-q_A)^2 \right]
\end{eqnarray*}
であるためそれぞれの方程式は一次の形まで書くと
\begin{eqnarray*}
  m\ddot{q}_A &&= -k_1l^2 \left[ -\left( \frac{-l+q_A}{l} \right)\frac{1}{l}+ \left( \frac{-l+q_A}{l} \right)^3 \frac{1}{l} \right]+k_0(q_B-q_A)\\
  &&\approx -k_1l^2 \left[ \frac{1}{l}+ \frac{2q_A}{l^2} \right] + k_0\left( q_B-q_A \right)\\
  m\ddot{q}_B &&= -k_1l^2 \left[ -\left( \frac{q_B}{l} \right)\frac{1}{l}+ \left( \frac{q_B}{l} \right)^3 \frac{1}{l}\right]-k_0\left\{ -(q_C-q_B) + (q_B-q_A) \right\}\\
  &&\approx k_1 q_B -2k_0 q_B + k_0 q_C + k_0q_A\\
  m\ddot{q}_C &&\approx -k_1l^2 \left[ \frac{1}{l}+ \frac{2q_C}{l} \right] - k_0(q_C-q_B)
\end{eqnarray*}
である.よってこられを行列で表現すると
\begin{eqnarray}
  m
  \begin{bmatrix}
   \ddot{q}_A\\
   \ddot{q}_B\\ 
   \ddot{q}_C
  \end{bmatrix}
  =
  \begin{bmatrix}
    -2k_1-k_0 & k_0 & 0\\
    k_0 & k_1-2k_0 & k_0\\
    0 & k_0 & -2k_1-k_0
  \end{bmatrix}
  \begin{bmatrix}
    q_A\\
    q_B\\
    q_C
  \end{bmatrix}
  + 
  \begin{bmatrix}
    -k_1l\\
    0\\
    k_1l
  \end{bmatrix}
\end{eqnarray}
となる.ここで,一般的に
\begin{eqnarray}
  m\ddot{\bm{q}} = A\bm{q} + B
\end{eqnarray}
があり対角化して
\begin{eqnarray}
  \bm{q}' = U\bm{q}
\end{eqnarray}
とすると
\begin{eqnarray}
  m\ddot{\bm{q}}' = U^{-1}AU \bm{q}' + U^{-1}BU 
\end{eqnarray}
であり,例えば一つの成分を取り出して
\begin{eqnarray}
  m\ddot{q}_i' = a q_i' + b = a\left( q_i' + \frac{b}{a} \right)
\end{eqnarray}
であり,この基準振動$\omega$が満たす方程式は
\begin{eqnarray}
  -\omega^2 = \frac{a}{m}
\end{eqnarray}
となるためこれは$b$に依存しない.よって,行列$A$のみの対角化をすればよい.よって,この行列の固有方程式は
\begin{eqnarray*}
  \det
  \begin{bmatrix}
    -2k_1-k_0-\lambda & k_0 & 0\\
    k_0 & k_1-2k_0-\lambda & k_0\\
    0 & k_0 & -2k_1-k_0-\lambda
  \end{bmatrix}
  &&= \left( -2k_1-k_0-\lambda \right)\left( \lambda^2+(k_1+3k_0)\lambda + 3k_0k_1 -2k_1^2 \right)\\
  &&=0
\end{eqnarray*}
であるため固有値は
\begin{eqnarray}
  \lambda = -2k_1-k_0, \frac{1}{2}\left[ -(k_1+3k_0) \pm \sqrt{ 9k_1^2-6k_1k_2 + 9k_0^2 } \right]
\end{eqnarray}
である.よって固有振動数はこれに$1/m$倍して$-1$をかけて平方したものであるため
\begin{eqnarray}
  \omega = \sqrt{ \frac{2k_1+k_0}{m} }, \sqrt{ \frac{1}{2m}\left( k_1+3k_0 \mp \sqrt{ 9k_1^2-6k_1k_2 + 9k_0^2 } \right) }
\end{eqnarray}
である.
\subsection*{\large{[2.2]}}
不安定な解はこの固有振動が虚数数の時であり,時刻に対して指数関数的に増大する.
\begin{eqnarray}
  9k_1^2-6k_1k_2 + 9k_0^2 = 9k_0^2\left[ \left( \frac{k_1}{k_0} \right)^2 -\frac{2}{3}\left( \frac{k_1}{k_0} \right)+1 \right]
\end{eqnarray}
は任意の$k_1/k_0$で正の値であるが,
\begin{eqnarray}
  k_1+3k_0 - \sqrt{ 9k_1^2-6k_1k_2 + 9k_0^2 } <0
\end{eqnarray}
の時この固有振動は不安定になる.よって,
\begin{eqnarray}
  9k_1^2-6k_1k_2 + 9k_0^2 > \left( k_1+3k_0\right)^2,\quad\therefore 4\left( \frac{k_1}{k_0} \right)^2-6\left( \frac{k_1}{k_0} \right) +1 >0
\end{eqnarray}
である.これは
\begin{eqnarray}
  0<\frac{k_1}{k_0}<3-\sqrt{5}, または\quad \frac{k_1}{k_0}>3+\sqrt{5}
\end{eqnarray}
よって,$k_C$の値は
\begin{eqnarray}
  k_C = (3+\sqrt{5})k_0
\end{eqnarray}
である.

\end{document}