%%%%%%%%%%%%%%%%%%%%%%%%%%%%%%%%%%%%%%%%%%%%%%%%%%%%%%%
%                   File: OSAmeetings.tex             %
%                  Date: 29 Novemver 2018              %
%                                                     %
%     For preparing LaTeX manuscripts for submission  %
%       submission to OSA meetings and conferences    %
%                                                     %
%       (c) 2018 Optical Society of America           %
%%%%%%%%%%%%%%%%%%%%%%%%%%%%%%%%%%%%%%%%%%%%%%%%%%%%%%%

\documentclass[12pt,dvipdfmx]{jsarticle}
%% if A4 paper needed, change letterpaper to A4
\usepackage[dvipdfmx]{graphicx}
\usepackage[dvipdfmx]{color}
\usepackage{osameet3} %% use version 3 for proper copyright statement
\usepackage{ascmac}
%% provide authormark
\newcommand\authormark[1]{\textsuperscript{#1}}

%% standard packages and arguments should be modified as needed
\usepackage{amsmath,amssymb}
\usepackage[colorlinks=true,bookmarks=false,citecolor=blue,urlcolor=blue]{hyperref} %pdflatex
%\usepackage[breaklinks,colorlinks=true,bookmarks=false,citecolor=blue,urlcolor=blue]{hyperref} %latex w/dvipdf
\usepackage{mathtools}
\usepackage{amsmath}
\usepackage{empheq}
\usepackage{physics}
\usepackage[scr=rsfs]{mathalpha}
\usepackage[svgnames]{xcolor}% tikzより前に読み込む必要あり
\usepackage{tikz}
\usepackage{bm}
\usepackage{here}
\usepackage{braket}
\usepackage{framed,color}
\usepackage{dcolumn}
\definecolor{shadecolor}{gray}{0.80}
\usetikzlibrary{perspective}
\tikzset
{%
  my ball/.style={draw,circle,minimum size=2*\r cm,inner sep=0,shading=ball,ball color=cyan!50!blue,opacity=#1},
  my ball/.default=1,
  hidden line/.style={black!60}
}
\begin{document}

\title{東工大理物 2020}

\author{21B00817 鈴木泰雅,\authormark{1}}

\email{\authormark{*}suzuki.t.ec@m.titech.ac.jp} %% email address is required

\section*{\Large{第一問}}
この問題は中心力の作用する場合の運動の解析である.実は角運動量が保存する条件から運動方程式の自由度を削減することができ,自由度を削減したラウシアンという修正したラグランジアンを得る.これを下に考察していく.
まず,極座標における運動エネルギーは
\begin{eqnarray}
  T = \frac{1}{2}m\dot{\bm{r}}^2 = \frac{1}{2}m\left( \dot{r}^2 + \left( r\dot{\theta} \right)^2 \right)
\end{eqnarray}
であり,ポテンシャルは
\begin{eqnarray}
  U = -G\frac{mM}{r}
\end{eqnarray}
である.
ラグランジアンは
\begin{eqnarray}
  L = \frac{1}{2}m\left( \dot{r}^2 + \left( r\dot{\theta} \right)^2 \right) + G\frac{mM}{r}
\end{eqnarray}
よってラグランジュ方程式は
\begin{eqnarray}
  \frac{d}{dt}\left( \frac{\partial L}{\partial \dot{r}} \right) - \frac{\partial L}{\partial r} &&= m\ddot{r}- mr \dot{\theta}^2 + G\frac{mM}{r^2} =0\\
  \frac{d}{dt}\left( \frac{\partial L}{\partial \dot{\theta}} \right) - \frac{\partial L}{\partial \theta} &&= mr^2 \ddot{\theta} =0
\end{eqnarray}
である.また,ラグランジアンには$\theta$が陽に含まれていないため,$\theta$に対応する一般化運動量である角運動量が保存する:
\begin{eqnarray}
  \frac{\partial L}{\partial \dot{\theta}} = mr^2 \dot{\theta} = \text{Const} =: M_\theta
\end{eqnarray}
また,この保存量(第一積分とも呼ぶ)から,速度に対応する成分である$\dot{\theta}$を逆算すると
\begin{eqnarray}
  \dot{\theta} = \frac{M_\theta}{mr^2}
\end{eqnarray}
であり,これによってラグランジアンの自由度が減り以下の修正したラグランジアンが得られる.なお,ラウシアンは,元のラグランジアンから$M_\theta \dot{\theta}$差し引く必要があるため(詳しくは山本解析参照)
\begin{eqnarray}
  R &&:= \frac{1}{2}m\dot{r}^2 + \frac{1}{2}\frac{M_\theta^2}{mr^2} + G\frac{mM}{r} -\frac{M_\theta^2}{mr^2}\\
  &&=\frac{1}{2}m\dot{r}^2 - \frac{1}{2}\frac{M_\theta^2}{mr^2} + G\frac{mM}{r}
\end{eqnarray}
となる.
このラグランジアンには$\theta,\dot{\theta}$が含まれておらず,もともと$r,\dot{r},\theta,\dot{\theta}$という4つの自由度があったが,この式には2つの自由度しか存在しない.一般的には$q,\dot{q}$の$2n$個の自由度
があった時,第一積分によって,$2n-2$個の自由度に削減されるのである.なお,この修正したラグランジアンのことをラウシアンという.また,
\begin{eqnarray}
  \frac{1}{2}m\dot{r}^2
\end{eqnarray}
を運動エネルギーの項と見立てるとこの系のポテンシャルは
\begin{eqnarray}
  \frac{1}{2}\frac{M_\theta^2}{mr^2} - G\frac{mM}{r}
\end{eqnarray}
である.このポテンシャルを実効ポテンシャルと言う.

これより自由度が削減したラウシアンからラグランジュ方程式を求めることによって$r$の時間発展を解析していく.この系のラグランジアンは時間$t$に陽に依存しないため,ハミルトニアンも第一積分である.よって,
\begin{eqnarray}
  H = \frac{\partial R}{\partial \dot{r}}\dot{r}-R = \frac{1}{2}m\dot{r}^2 + \frac{1}{2}\frac{M_\theta^2}{mr^2} - G\frac{mM}{r}
\end{eqnarray}
が保存する.ここで,これらのエネルギー保存と角運動量保存の式より,
\begin{eqnarray}
  \frac{d r}{dt} &&= \pm\sqrt{ \frac{2}{m}\left(H- \frac{1}{2}\frac{M_\theta^2}{mr^2} + G\frac{mM}{r}\right) }\\
  \frac{d\theta}{dt}&&= \frac{M_\theta}{mr^2}
\end{eqnarray}
が得られる.よってこれらを組み合わせて
\begin{eqnarray}
  \frac{dr}{d\theta} = \pm\frac{mr^2}{M_\theta }\sqrt{ \frac{2}{m}\left(H- \frac{1}{2}\frac{M_\theta^2}{mr^2} + G\frac{mM}{r}\right) }
\end{eqnarray}
と表現できる.なお,
\begin{eqnarray}
  \frac{d}{d\theta}\left( \frac{1}{r} \right) &&= \mp \frac{m}{M_\theta} \sqrt{ \frac{2}{m}\left(H- \frac{1}{2}\frac{M_\theta^2}{mr^2} + G\frac{mM}{r}\right) }\\
  &&=\mp \sqrt{ -\left( \frac{1}{r}- A \right)^2 + A^2 + \frac{2m}{M_\theta^2} H }, \quad A = \frac{Gm^2 M}{M_\theta^2}
\end{eqnarray}
とも表現できる.ここで,
\begin{eqnarray}
  \frac{1}{r}- A = \sqrt{ A^2 + \frac{2m}{M_\theta^2} H }\cos\eta
\end{eqnarray}
とすると,
\begin{eqnarray}
  \mp\sin\eta = \frac{d}{d\theta}\cos\eta = -\sin\eta \frac{d\eta}{d\theta},\quad\therefore \frac{d\eta}{d\theta} = \pm 1
\end{eqnarray}
と変形できるため,定数$\omega$を用いて
\begin{eqnarray}
  \eta = \pm(\theta +\omega)
\end{eqnarray}
となり,
\begin{eqnarray}
  \frac{1}{r} = A + \sqrt{ A^2 + \frac{2m}{M_\theta^2} H }\cos(\theta +\omega) = A\left( 1 + \sqrt{ 1 + \frac{2m}{M_\theta^2 A^2}H }\cos(\theta +\omega) \right)
\end{eqnarray}
であり,
\begin{eqnarray}
  \begin{cases}
    H<0 & 楕円\\
    H=0 & 放物線\\
    H>0 & 双曲線
  \end{cases}
\end{eqnarray}
であることが,離心率:
\begin{eqnarray}
  e = \sqrt{ 1 + \frac{2m}{M_\theta^2 A^2}H }
\end{eqnarray}
を使うことによって分かる.
\section*{\Large{第四問}}
\subsection*{(1)}
\begin{eqnarray}
  \Psi(\lambda) &&= \det\left( \lambda I-A \right)\\
  &&= \det
  \begin{bmatrix}
    \lambda-1 & -3 & -6\\
    0 & \lambda-2 & -2\\
    1 & -3 & \lambda-3
  \end{bmatrix} = (\lambda-1)(\lambda-2)(\lambda-3)+6 + 6(\lambda-2)-6(\lambda-1)\\
  &&= \lambda^3-5\lambda^2+8\lambda-4 = (\lambda-1)(\lambda-2)^2 =0
\end{eqnarray} 
である.よって,固有値は$\lambda=1,2$である.また,固有状態は
\begin{eqnarray}
  &&\lambda=2の時:\frac{1}{\sqrt{10}}[3,1,0]^{T}\\
  &&\lambda=1の時:\frac{1}{\sqrt{21}}[-4,-2,1]^{T}
\end{eqnarray}
\subsection*{(2)}
定義より,
\begin{eqnarray}
  \Psi(A) = (A-1I)(A-2I)^2 = 
  \begin{bmatrix}
    -6 & 18 & 12\\
    -2 & 6 & 4\\
    -1 & 3 & 2
  \end{bmatrix}
\end{eqnarray}
である.
\subsection*{(3)}
(2)より,
\begin{eqnarray}
  A^3 = 5A^2 -8A+4 + \Psi
\end{eqnarray}
より,これを使用して簡単にすると
\begin{eqnarray}
  240A^2-465A + 226 + \Psi\cdot(49+A)
\end{eqnarray}
である.よって,
\begin{eqnarray}
  f(A) =
  \begin{bmatrix}
    -6 & 48 & 48\\
    -4 & 22 & 16\\
    -6 & 18 & 18
  \end{bmatrix}
\end{eqnarray}
\subsection*{(4)}
$e^{it}$のラプラス変換を考え,実部と虚部でそれぞれ$\cos t$と$\sin t$を考えれば良い.よって,
\begin{eqnarray}
  \mathcal{L}[e^{it}]= \int_0^{\infty}\exp\left( (i-s)t \right)dt = -1\frac{1}{i-s}= \frac{s}{s^2+1} + i\frac{1}{s^2+1}
\end{eqnarray}
である.よって,
\begin{eqnarray}
  \mathcal{L}[\cos t] &&= \Re \mathcal{L}[e^{it}] =  \frac{s}{s^2+1}\\
  \mathcal{L}[\sin t] &&= \Im \mathcal{L}[e^{it}] =  \frac{1}{s^2+1}\\
\end{eqnarray}
また,
\begin{eqnarray}
  \mathcal{L}[e^{at}]= \int_0^{\infty}\exp\left( (a-s)t \right)dt = -1\frac{1}{a-s}= \frac{1}{s-a}
\end{eqnarray}
\subsection*{(5)}
\begin{eqnarray}
  G(s)= \int_0^{\infty}e^{-st}g(t) &&= \left[ \left( -\frac{1}{s}e^{-st} \right)g(t) \right]_0^{\infty} - \int_0^\infty  \left( -\frac{1}{s}e^{-st} \right)g(t) dt\\
  &&= \frac{1}{s}\int_0^{\infty} e^{-st}f(t) = \frac{1}{s} F(s)
\end{eqnarray}
である.
\subsection*{(6)}
ラプラス変換をすると
\begin{eqnarray}
  H(s)-\frac{1}{s}H(s) = 5 \frac{s}{s^2+1},\quad\therefore H(s) = \frac{1}{s-1}\left( 1-\frac{1}{s^2+1} \right)
\end{eqnarray}
よりこれを逆ラプラス変換して
\begin{eqnarray}
  h(t) = e^t ( 1-\sin t )
\end{eqnarray}
である.

\section*{\Large{第五問}}
なんか見たことない?ひょっとして期末テストと同じ問題?
\subsection*{(1)}
古典近似での分配関数は
\begin{eqnarray}
  Z &&= \frac{1}{(2\pi\hbar)^N}\int\int\cdots\int\exp\left( -\frac{1}{k_BT} \sum_{i=1}^{N}\left( \frac{p_i^2}{2m} + \frac{1}{2}m\omega^2 x_i^2 \right) \right)\Pi_{j=1}^{N}dp_j dx_j\\
  &&=z^N\\
  z&&= \frac{1}{2\pi\hbar}\int\int \exp\left( -\frac{1}{k_BT} \left( \frac{p^2}{2m} + \frac{1}{2}m\omega^2 x^2 \right) \right)dp dx\\
  &&= \frac{k_B T}{\hbar\omega}
\end{eqnarray}
よって,
\begin{eqnarray}
  Z = \left( \frac{k_B T}{\hbar\omega} \right)^{3N}
\end{eqnarray}
となる.
\subsection*{(2)}
$F,S$を求めて,$F = E-TS$を利用する.
\begin{eqnarray}
  F = - k_B T \log Z = - k_B T N \log\left( \frac{k_B T}{\hbar\omega} \right)\\
  S = -\frac{\partial F}{\partial T} = k_B N \log\left( \frac{k_B T}{\hbar\omega} \right) + k_BN
\end{eqnarray}
であるため,
\begin{eqnarray}
  E = Nk_B T
\end{eqnarray}
となる.もしくは
\begin{eqnarray}
  E = -\frac{d}{d\beta}\log Z
\end{eqnarray}
からも求まる.
よって比熱は
\begin{eqnarray}
  C = \frac{dE}{dT} = 3Nk_B
\end{eqnarray}
となる.
\subsection*{(3)}
分配関数は
\begin{eqnarray}
  z &&= \text{Tr} \exp\left( -\beta H \right) = \sum_i \exp\left[ -\frac{1}{k_BT} \left( n+ \frac{1}{2} \right)\hbar\omega \right]\\
  &&=\frac{e^{-\hbar\omega/2k_BT}}{1-e^{-\hbar\omega/k_BT}}
\end{eqnarray}
より,
\begin{eqnarray}
  Z = z^{3N} = \left( \frac{e^{-\hbar\omega/2k_BT}}{1-e^{-\hbar\omega/k_BT}} \right)^{3N}
\end{eqnarray}
となる.
\subsection*{(4)}
\begin{eqnarray}
  F = -3N k_BT \log \left( \frac{e^{-\hbar\omega/2k_BT}}{1-e^{-\hbar\omega/k_BT}} \right) = \frac{1}{2}3N\hbar\omega + Nk_BT \log \left( 1-e^{-\hbar\omega/k_BT} \right)
\end{eqnarray}
よって,
\begin{eqnarray}
  E = -T^2 \frac{d}{dT}\left( \frac{F}{T} \right) = \frac{1}{2}N\hbar\omega + \frac{N\hbar\omega}{e^{\hbar\omega/k_B T}-1}
\end{eqnarray}
よって比熱は
\begin{eqnarray}
  C = \frac{dE}{dT} = Nk_B\left( \frac{\hbar\omega}{k_BT} \right)^2 \frac{e^{\hbar\omega/k_BT}}{(e^{\hbar\omega/k_BT}-1)^2}
\end{eqnarray}
\subsection*{(5)}
低温による漸近では,$k_B T \ll \hbar\omega$より,
\begin{eqnarray}
  C \sim 3Nk_B\left( \frac{\hbar\omega}{k_BT} \right)^2 e^{-\hbar\omega/k_BT}
\end{eqnarray}
であり,高温による展開では$k_B T \gg \hbar\omega$より,
\begin{eqnarray}
  e^{\hbar\omega/k_BT} \sim 1 + \frac{\hbar\omega}{k_BT}
\end{eqnarray}
より,
\begin{eqnarray}
  C \sim 3N k_B
\end{eqnarray}
となる.
\subsection*{(6)}
高温では一致をして低温では異なる理由として,高温では振動子が励起して古典的な描像を同じになる一方,低温では基底状態に近く量子効果が大きくなるため.
\subsection*{(7)}
\section*{\Large{第六問}}
\subsection*{(1)}
1: -19, 2: -31, 3: -34
\subsection*{(2)}
4:X線,5:マクロ波,6:$\gamma$線,7:X線
\subsection*{(3)}
紫外線だから紫?
また,
\begin{eqnarray}
  E = \frac{\hbar c}{2\pi\lambda}, f = \frac{c}{\lambda}
\end{eqnarray}
また,$\hbar c\sim 197[\text{MeV}\cdot \text{fm}]$であるため,
\begin{eqnarray}
  E \sim \frac{197[\text{MeV}\cdot \text{fm}]}{2\pi 6.3\cdot 10^8 [\text{fm}]}
\end{eqnarray}
であるが小さい,,よって普通に計算する.
\begin{eqnarray}
  E = \frac{hc}{\lambda} \sim \frac{6.6\cdot 10^{-34}\cdot 3\cdot 10^8}{6.3\cdot 10^{-7}} \sim 3\cdot 10^{-33}
\end{eqnarray}
\subsection*{(4)}
シンチレータの中を荷電粒子が通過するとその経路にわたって分子が励起され,その励起状態のもつエネルギーの一部が可視光として放出し,その光量を計測することで荷電粒子のエネルギーを見積もることができる.
\subsection*{(5)}
名称:光電子増倍管,A:光電効果で光子を変換する.B:Aから放出された光子をより多く増倍させる.
\begin{eqnarray}
  \frac{10^{-2}[\text{m}]}{3.0\cdot 10^{8}[\text{m/s}]} + \frac{1}{2}1[\text{ns}] \sim 5.3\cdot 10^{-10}[\text{s}]
\end{eqnarray}
\subsection*{(6)}
\subsection*{(7)}
潮解性があるため,大気に触れるとボロボロになってしまうため.
\subsection*{(8)}
$\gamma$線では物質との反応によって$\gamma$線の数に比例するだけの電子が放出され,それらが$X$線となり検出できるため,$\gamma$線一つあたりのエネルギー分析が可能になる.
\end{document}