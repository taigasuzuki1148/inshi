%%%%%%%%%%%%%%%%%%%%%%%%%%%%%%%%%%%%%%%%%%%%%%%%%%%%%%%
%                   File: OSAmeetings.tex             %
%                  Date: 29 Novemver 2018              %
%                                                     %
%     For preparing LaTeX manuscripts for submission  %
%       submission to OSA meetings and conferences    %
%                                                     %
%       (c) 2018 Optical Society of America           %
%%%%%%%%%%%%%%%%%%%%%%%%%%%%%%%%%%%%%%%%%%%%%%%%%%%%%%%

\documentclass[12pt,dvipdfmx]{jsarticle}
%% if A4 paper needed, change letterpaper to A4
\usepackage[dvipdfmx]{graphicx}
\usepackage[dvipdfmx]{color}
\usepackage{osameet3} %% use version 3 for proper copyright statement
\usepackage{ascmac}
%% provide authormark
\newcommand\authormark[1]{\textsuperscript{#1}}

%% standard packages and arguments should be modified as needed
\usepackage{amsmath,amssymb}
\usepackage[colorlinks=true,bookmarks=false,citecolor=blue,urlcolor=blue]{hyperref} %pdflatex
%\usepackage[breaklinks,colorlinks=true,bookmarks=false,citecolor=blue,urlcolor=blue]{hyperref} %latex w/dvipdf
\usepackage{mathtools}
\usepackage{amsmath}
\usepackage{empheq}
\usepackage{physics}
\usepackage[scr=rsfs]{mathalpha}
\usepackage[svgnames]{xcolor}% tikzより前に読み込む必要あり
\usepackage{tikz}
\usepackage{bm}
\usepackage{here}
\usepackage{braket}
\usepackage{framed,color}
\usepackage{dcolumn}
\definecolor{shadecolor}{gray}{0.80}
\usetikzlibrary{perspective}
\tikzset
{%
  my ball/.style={draw,circle,minimum size=2*\r cm,inner sep=0,shading=ball,ball color=cyan!50!blue,opacity=#1},
  my ball/.default=1,
  hidden line/.style={black!60}
}
\begin{document}

\title{東工大理物 2021}

\author{21B00817 鈴木泰雅,\authormark{1}}

\email{\authormark{*}suzuki.t.ec@m.titech.ac.jp} %% email address is required

\section*{\Large{第一問}}
\subsection*{\large{(1)}}
今回の角運動量の成分は$z$軸しか持たないため,慣性モーメントは$z$軸を中心に考えれば良い.よって,
\begin{eqnarray}
  I_l = \int_0^{l}r^2 \frac{m}{l}dr = \frac{1}{3}ml^2
\end{eqnarray}
であり,$O$まわりは
\begin{eqnarray}
  I_a = \int_0^{a} r^2 \left( \frac{M}{\pi a^2} \right)2\pi r dr = \frac{Ma^2}{2}
\end{eqnarray}
\subsection*{\large{(2)}}
剛体は質量中心に関してのみ考えれば良いので,
\begin{eqnarray}
  V_l &&= -\int_{l/2}^{l/2\cos\theta}mg = mg \frac{l}{2}(1-\cos\theta), \\
  V_a &&= -\int_{l+a}^{l\cos\theta+a\cos\phi} Mg = Mg \left( l(1-\cos\theta) + a(1-\cos\phi) \right)
\end{eqnarray}
となる.
\subsection*{\large{(3)}}
剛体の問題では回転の運動エネルギーと固定点の運動エネルギーに分解できる.また,回転の運動エネルギ-は系によらず
\begin{eqnarray}
  T = \frac{1}{2}\bm{\omega}^{T}\hat{I}\bm{\omega}
\end{eqnarray}
と表現することができる.よって,
\begin{eqnarray}
  T_l = \underbrace{\frac{1}{2}\left( \frac{1}{3}ml^2 \right)\dot{\theta}^2}_{回転の運動エネルギー} + \underbrace{0}_{固定点Pの速度は0}
\end{eqnarray}
となり,
\begin{eqnarray}
  T_a = \underbrace{\frac{1}{2}\left( \frac{Ma^2}{2} \right)\dot{\phi}^2}_{回転の運動エネルギー} + \underbrace{\frac{1}{2}M\left\{ l^2\dot{\theta}^2 + a^2\dot{\phi}^2 + 2al\dot{\theta}\dot{\phi}\cos(\theta-\phi) \right\}}_{固定点の運動エネルギー}
\end{eqnarray}
となる.よって,全体の運動エネルギーは
\begin{eqnarray}
  T = T_l + T_a = \frac{1}{6}ml\dot{\theta}^2 + \frac{3}{4}Ma^2\dot{\phi}^2 + \frac{1}{2}Ml^2\dot{\theta}^2 + Mal\dot{\theta}\dot{\phi}\cos(\theta-\phi)
\end{eqnarray}

\subsection*{\large{(4)}}
ラグランジアンは
\begin{eqnarray}
  L &&= T-V \\
  &&= \frac{1}{6}ml\dot{\theta}^2 + \frac{3}{4}Ma^2\dot{\phi}^2 + \frac{1}{2}Ml^2\dot{\theta}^2 + Mal\dot{\theta}\dot{\phi}\cos(\theta-\phi) \\
  &&\quad-mg \frac{l}{2}(1-\cos\theta)-Mg \left( l(1-\cos\theta) + a(1-\cos\phi) \right)
\end{eqnarray}
となるためラグランジュ方程式は
\begin{eqnarray}
  &&\ddot{\theta}\left( \frac{1}{3}ml^2 + Ml^2 \right)+ \left( Mal\cos(\theta-\phi)\ddot{\phi}-Mal\sin(\theta-\phi)(\dot{\theta}-\dot{\phi})\dot{\phi} \right)\\
  &&\quad + Mal\sin(\theta-\phi)\dot{\theta}\dot{\phi}+mg\frac{l}{2}\sin\theta +Mgl\sin\theta =0\\
  &&\frac{3}{2}Ma^2 \ddot{\phi} + \left( Mal \cos(\theta-\phi)\ddot{\theta}-Mal\sin(\theta-\phi)(\dot{\theta}-\dot{\phi})\dot{\theta} \right) + Mga\sin\phi =0
\end{eqnarray}
であり,微小項を無視することによって,
\begin{eqnarray}
  &&\ddot{\theta}l^2\left( \frac{1}{3}m + M \right) + \ddot{\phi}al(M) + \theta gl \left( \frac{1}{2}m +M \right) =0\\
  &&\ddot{\phi}\frac{3}{2}a^2 (M) + \ddot{\theta}al(M) + \phi ga (M) =0
\end{eqnarray}
が得られる.
\subsection*{\large{(5)}}
$m/M\to 0$のラグランジュ方程式は
\begin{eqnarray}
  &&\ddot{\theta}l^2M + \ddot{\phi}al(M) + \theta gl M =0\\
  &&\ddot{\phi}\frac{3}{2}a^2 M + \ddot{\theta}alM + \phi ga M =0
\end{eqnarray}
となり,$\ddot{\theta},\ddot{\phi}$について解くと
\begin{eqnarray}
  \begin{bmatrix}
    \ddot{\theta}\\
    \ddot{\phi}
  \end{bmatrix}
  =-\frac{2g}{al}
  \begin{bmatrix}
    3/2a & -a \\
    -l & l    
  \end{bmatrix}
  \begin{bmatrix}
    \theta \\
    \phi
  \end{bmatrix}
\end{eqnarray}
となる.この時,基準振動数はこの係数行列の固有値を$-1$倍して平方を取った値に等しいので,
(詳しくは\href{http://ishi-lab.mp.es.osaka-u.ac.jp/lecture/Mech2/%E9%80%A3%E6%88%90%E6%8C%AF%E5%8B%95%E3%83%86%E3%82%AD%E3%82%B9%E3%83%88.PDF}{こちら})
\begin{eqnarray}
  \omega_{\pm}^2 = \frac{g}{al}\left( l+\frac{3a}{2} \pm \sqrt{ \left( l+\frac{3a}{2} \right)^2 +4al } \right)
\end{eqnarray}
\subsection*{\large{(6)}}

\begin{eqnarray}
  \omega_{\pm}^2 \approx \frac{g}{a} \left( 1+ \frac{3}{2}\frac{a}{l} \pm \sqrt{ 1+7\frac{a}{l} } \right) \approx \frac{g}{a} \left( 1+ \frac{3}{2}\frac{a}{l} \pm\left( 1 + \frac{7}{2}\frac{a}{l} \right) \right)
\end{eqnarray}
となるため,
\begin{eqnarray}
  \omega_+^2 \approx \frac{2g}{a}, \quad\omega_-^2 \approx -\frac{2g}{l}
\end{eqnarray}
であるが,なぜマイナスになってしまうのか...

\newpage
\section*{\Large{第二問}}
\subsection*{\large{(1)}}
ローレンツ力より,
\begin{eqnarray}
  IB_0
\end{eqnarray}
である.
\subsection*{\large{(2)}}
線素$dr$が受ける力$dF$は
\begin{eqnarray}
  dF = IB_0 dr \quad (半時計周り)
\end{eqnarray}
より,力のモーメント(torque)は
\begin{eqnarray}
  \int_0^{a} [ \bm{r}\times d\bm{F} ]_z dr= \frac{1}{2}a^2 IB_0
\end{eqnarray}
である.
\subsection*{\large{(3)}}
ファラデーの電磁誘導の法則より,
\begin{eqnarray}
  V_{\text{emf}} = \frac{d}{dt}\left( B_0 \cdot \frac{1}{2}a^2 \int_0^{t}\omega(t')dt' \right) = \frac{1}{2}a^2B_0 \omega(t)
\end{eqnarray}
となる.
\subsection*{\large{(4)}}
レールから棒に流れる向きを正とすると,回路の方程式より
\begin{eqnarray}
  V-\frac{1}{2}a^2B_0 \omega(t)  =  IR , \quad\therefore I =\frac{1}{R}\left( V-\frac{1}{2}a^2B_0 \omega(t) \right)
\end{eqnarray}
\subsection*{\large{(5)}}
回転の運動方程式は,慣性モーメントが$\frac{1}{3}\lambda a^3$より,
\begin{eqnarray}
  \frac{1}{3}\lambda a^3\frac{d\omega(t)}{dt} =  \frac{1}{2}a^2 IB_0 =  \frac{1}{2}a^2 B_0\frac{1}{R}\left( V-\frac{1}{2}a^2B_0 \omega(t) \right)
\end{eqnarray}
であり,終端では$\omega(t)$が時間依存しなくなるため,
\begin{eqnarray}
  \frac{1}{2}a^2 B_0\frac{1}{R}\left( V-\frac{1}{2}a^2B_0 \omega(t) \right)=0,\quad\therefore \omega = \frac{2V}{a^2 B_0}
\end{eqnarray}
\subsection*{\large{(6)}}
問題文に書き込んだような問題設定で考える.
回路の方程式は
\begin{eqnarray}
  \frac{Q}{C} + R \frac{dQ}{dt} -\frac{1}{2}B_0a^2 \omega(t) =0
\end{eqnarray}
であり,回転の運動方程式は
\begin{eqnarray}
  \frac{1}{3}\lambda a^3\frac{d\omega(t)}{dt} = -\frac{1}{2}IB_0a^2 = -\frac{1}{2}\frac{dQ}{dt}B_0 a^2
\end{eqnarray}
である.よって,
\begin{eqnarray}
  \frac{d}{dt}\left( \frac{1}{3}\lambda a^3 \omega(t) + \frac{1}{2}Q(t)B_0a^2 \right) =0
\end{eqnarray}
より,
\begin{eqnarray}
  \omega(t) + \left( \frac{3B_0}{2\lambda a} \right)Q(t) =\text{Const}
\end{eqnarray}
である.また,$t=0$で$\omega=0$とすると,
\begin{eqnarray}
  \omega(t) + \left( \frac{3B_0}{2\lambda a} \right)Q(t) =\left( \frac{3B_0}{2\lambda a} \right)Q_0
\end{eqnarray}
である.また,回路の方程式より,終端では$dQ=0$であるため,
\begin{eqnarray}
  \frac{Q_{\infty}}{C} - \frac{1}{2}B_0a^2 \omega_{\infty}=0
\end{eqnarray}
であるため,
\begin{eqnarray}
  Q_{\infty} = \frac{1}{2}CB_0a^2 \omega_{\infty}
\end{eqnarray}
ここで,
\begin{eqnarray}
  \omega_{\infty} +  \left( \frac{3B_0}{2\lambda a} \right)\frac{1}{2}CB_0a^2 \omega_{\infty} = Q_0
\end{eqnarray}
であるため,
\begin{eqnarray}
  \omega_{\infty} = \left( 1 + \frac{3B_0^2 C a}{4\lambda} \right)^{-1}Q_0
\end{eqnarray}
である.
\subsection*{\large{(7)}}
そもそも起電力が
\begin{eqnarray}
  V_{\text{emf}} = \frac{d}{dt} \left( B(t) \frac{1}{2}a^2 \int_0^t \omega(t')dt' \right)
\end{eqnarray}
とすると方程式が複雑になりすぎて求まらなくない?
\newpage
\section*{\Large{第三問}}
\subsection*{\large{(1)}}
\begin{eqnarray}
  \nabla\ln |\bm{r}| = \frac{\bm{r}}{r}
\end{eqnarray}
である.
\subsection*{\large{(2)}}
\begin{eqnarray}
  \bm{r} = r\cos\theta\bm{e}_x + r\sin\theta\bm{e}_y
\end{eqnarray}
とすると,
\begin{eqnarray}
  \frac{\partial}{\partial r} &&= \frac{\partial x}{\partial r}\frac{\partial }{\partial x} + \frac{\partial y}{\partial r}\frac{\partial }{\partial y} =\frac{\partial\bm{r}}{\partial r}\cdot\nabla =  \bm{e}_r \cdot\nabla\\
  \frac{\partial}{\partial \theta} &&= \frac{\partial x}{\partial \theta}\frac{\partial }{\partial x} + \frac{\partial y}{\partial \theta}\frac{\partial }{\partial y}=\frac{\partial\bm{r}}{\partial \theta}\cdot\nabla = r\bm{e}_\theta \cdot\nabla
\end{eqnarray}
であるめ,$\nabla$というベクトルはそれぞれの基底の線形結合より,
\begin{eqnarray}
  \nabla = \bm{e}_r\frac{\partial}{\partial r} +  \bm{e}_\theta\frac{1}{r}\frac{\partial}{\partial \theta}
\end{eqnarray}
である.よって,
\begin{eqnarray}
  \nabla\cdot\nabla &&= \left( \bm{e}_r\frac{\partial}{\partial r} +  \bm{e}_\theta\frac{1}{r}\frac{\partial}{\partial \theta} \right)\left(\bm{e}_r\frac{\partial}{\partial r} +  \bm{e}_\theta\frac{1}{r}\frac{\partial}{\partial \theta}\right)\\
  &&= \frac{\partial^2}{\partial r^2} + \bm{e}_r \left( \frac{\partial}{\partial r} \bm{e}_\theta \right) \frac{1}{r}\frac{\partial}{\partial\theta} + \bm{e}_\theta \frac{1}{r}\left( \frac{\partial}{\partial\theta}\bm{e}_r \right)\frac{\partial}{\partial r} + \frac{1}{r^2}\frac{\partial^2}{\partial \theta^2}\\
  &&= \frac{\partial^2}{\partial r^2} + \frac{1}{r}\frac{\partial}{\partial r} + \frac{1}{r^2}\frac{\partial^2}{\partial\theta^2}
\end{eqnarray}
となる.

\subsection*{\large{(3)}}
\begin{eqnarray}
  \Delta\ln|\bm{r}| = -\frac{1}{r^2} + \frac{1}{r^2} = 0
\end{eqnarray}
である.
\subsection*{\large{(4)}}
$Q = \partial_x \ln r, P = -\partial_y \ln r$とすると,グリーンの定理より,
\begin{eqnarray}
  I = \oint_C \left( -\partial_y \ln r dx + \partial_y \ln r dy \right)
\end{eqnarray}
であり,
\begin{eqnarray}
  dx = \cos\theta dr  - r\sin\theta d\theta, \quad dy = \sin\theta dr + r\cos\theta d\theta
\end{eqnarray}
であることと,
\begin{eqnarray}
  \partial_x \ln r = \frac{x}{r^2} = \frac{\cos\theta}{r}, \quad \partial_y \ln r = \frac{y}{r^2} = \frac{\sin\theta}{r}
\end{eqnarray}
であるため,
\begin{eqnarray}
  I &&= \oint_C \left\{ -\frac{\sin\theta}{r}\left(cos\theta dr  - r\sin\theta d\theta \right) + \frac{\cos\theta}{r}\left( \sin\theta dr + r\cos\theta d\theta \right)  \right\}\\
  &&= \oint_C dr = 2\pi
\end{eqnarray}
である.
\subsection*{\large{(5)}}
\begin{eqnarray}
  \Delta\ln|r| = 2\pi\delta(\bm{r})
\end{eqnarray}
である.
\subsection*{\large{(7)}}
極の値を$r$とすると,
\begin{eqnarray}
  p = \pm\sqrt{ k^2 + i\delta } \approx \pm k \left( 1 + \frac{1}{2k^2}i\delta \right)
\end{eqnarray}
である.ここで,問題文に記載したような半円を考えた時
\begin{eqnarray}
  J = \int_{-r}^{r} + \int_{r=r}
\end{eqnarray}
を考えた時,(おそらくxの場合分けをしてジョルダン不等式で評価する必要がある.)
\begin{eqnarray}
  \left| \int_{\theta} \frac{e^{ipx}}{p^2-k^2-i\delta} dp \right| \sim\mathcal{O}\left( \frac{1}{r} \right) \to 0
\end{eqnarray}
であるため,最初の項のみを考える.よって,留数定理より,
\begin{eqnarray}
  J = 2\pi i \frac{e^{i\alpha x}}{2\alpha} \to 2\pi i \frac{e^{ikx}}{2k},\quad\delta\to 0
\end{eqnarray}
となる.

\end{document}