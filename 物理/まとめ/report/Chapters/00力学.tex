\documentclass[../Main.tex]{subfiles}
\begin{document}

\subsection{運動の原理}
\subsubsection{運動方程式}
物事の運動は次の運動方程式に基づいている:
\begin{eqnarray}
  m\frac{d^2\bm{r}}{dt^2} = \bm{F} + \bm{F}' \label{equation_motion}
\end{eqnarray}
ただし,$\bm{F}'$は物体をある曲面に拘束する力であり,$\bm{F}$は外力である.
そして,\textbf{拘束力は仕事をしない}という\textbf{仮想仕事の原理}を要請する.拘束を破らないような変化:
\begin{eqnarray}
  \delta\bm{r} = \dot{\bm{r}}dt
\end{eqnarray}
加えられた力がする仕事は
\begin{eqnarray}
  \delta W &&= \bm{F}\cdot \delta\bm{r} = \left( \bm{F} + \bm{F}' \right)\dot{\bm{r}}dt = m\frac{d^2\bm{r}}{dt^2}\dot{\bm{r}}dt\\
  &&=\frac{d}{dt} \left( \frac{1}{2}m\dot{\bm{r}}^2 \right)>0
\end{eqnarray}
となる.そして,
\begin{eqnarray}
  \mathcal{F} := \bm{F}\cdot\frac{\partial \bm{r}}{\partial q^i}\delta q^i  
\end{eqnarray}
とすると,釣り合いの条件は
\begin{eqnarray}
  \mathcal{F}= 0
\end{eqnarray}
である.また,
\begin{eqnarray}
  \bm{F} = -\frac{\partial U}{\partial \bm{r}}
\end{eqnarray}
の時は
\begin{eqnarray}
  \delta W = \bm{F}\cdot \delta\bm{r} =  -\frac{\partial U}{\partial \bm{r}}\delta\bm{r} = -\delta U =0
\end{eqnarray}
となり,ポテンシャルの極値が安定点となる.
\subsubsection{配位空間上の運動方程式}
運動方程式の中に拘束力があると少々扱いが難しい.そこで拘束力を除去した運動方程式を構築したい.そこで,運動方程式\eqref{equation_motion}の両辺に$\partial_i\bm{r}$をかけて,
\begin{eqnarray}
  m\left( \frac{d^2\bm{r}}{dt^2} \cdot \frac{\partial \bm{r}}{\partial q^i} \right) = \left( \bm{F} + \bm{F}' \right)\partial_i\bm{r} = \mathcal{F}
\end{eqnarray}
となる.ここで,左辺を一般化座標のみで表現することを試みる.
\begin{eqnarray}
  \frac{d \bm{r}}{dt} = \frac{\partial \bm{r}}{\partial q^j} \frac{d q^j}{dt}
\end{eqnarray}
であるため
\begin{eqnarray}
  \frac{d^2\bm{r}}{dt^2} = \frac{\partial \bm{r}}{\partial q^j} \frac{d^2 q^j}{dt^2} + \frac{\partial^2 \bm{r}}{\partial q^j \partial q^k}\frac{d q^j}{dt}\frac{d q^k}{dt}
\end{eqnarray}
となるため,
\begin{eqnarray}
  m \left\{ \left(\frac{\partial \bm{r}}{\partial q^j}\cdot  \frac{\partial \bm{r}}{\partial q^i} \right)\frac{d^2 q^j}{dt^2} + \left(\frac{\partial^2 \bm{r}}{\partial q^j \partial q^k}\cdot \frac{\partial \bm{r}}{\partial q^i} \right)\frac{d q^j}{dt}\frac{d q^k}{dt} \right\} = \mathcal{F}
\end{eqnarray}
ここで,$m\left(\frac{\partial \bm{r}}{\partial q^j}\cdot  \frac{\partial \bm{r}}{\partial q^i} \right)$を計量テンソル$m_{ij}$を使い,$ m\left(\frac{\partial^2 \bm{r}}{\partial q^j \partial q^k}\cdot \frac{\partial \bm{r}}{\partial q^i} \right)$を$C_{ijk}$という第一種クリストッフェル記号を用いて表現すると
\begin{eqnarray}
  m_{ij}\frac{d^2 q^j}{dt^2} + C_{ijk}\frac{d q^j}{dt}\frac{d q^k}{dt}  = \mathcal{F}_i
\end{eqnarray}
になる.
なお,
\begin{eqnarray}
  \frac{\partial m_{ij}}{\partial q^{l}} = C_{ijk} + C_{jik}
\end{eqnarray}
であり,他も同様に成立するので
\begin{eqnarray}
  C_{ijk} = \frac{1}{2}\left( \frac{\partial m_{ki}}{\partial q^j} + \frac{\partial m_{ji}}{\partial q^k} - \frac{\partial m_{jk}}{\partial q^i} \right)
\end{eqnarray}
となる.よって拘束力が含まれてない,一般化座標を使った方程式を作成できた.
\subsection{ラグランジュ方程式}
以上で求めた方程式は拘束力が含まれていない方程式ではあるものの,すべての計量テンソルとクリストッフェル記号を求めるのは些か大変である.そこで,ラグランジュ方程式を導出する.
系の時間発展に伴う時間変化:
\begin{eqnarray}
  d\bm{r} = \frac{\partial \bm{r}}{\partial q^i}dq^i + \frac{\partial\bm{r}}{\partial t}dt
\end{eqnarray}
となり,系の時間発展によって超曲面に垂直な成分を持ってしまう可能性があり.しかし,この場合も仮想変位:
\begin{eqnarray}
  \delta\bm{r} = \frac{\partial \bm{r}}{\partial q^i}\delta q^i
\end{eqnarray}
を考えることができ,
\begin{eqnarray}
  &&m_{ij}\frac{d^2 q^j}{dt^2} + C_{ijk}\frac{d q^j}{dt}\frac{d q^k}{dt}\\
  \quad&&=m_{ij}\frac{d^2 q^j}{dt^2} + \frac{1}{2}\left( \frac{\partial m_{ki}}{\partial q^j}+\frac{\partial m_{ji}}{\partial q^k}-\frac{\partial m_{jk}}{\partial q^i} \right)\frac{d q^j}{dt}\frac{d q^k}{dt}\\
  \quad&&=\frac{d}{dt}\left\{ \frac{\partial}{\partial \dot{q}^i}\left( \frac{1}{2}m_{jk}\dot{q}^j\dot{q}^k \right) \right\} -\frac{\partial}{\partial q^i}\left( \frac{1}{2}m_{jk}\dot{q}^j\dot{q}^k  \right)
\end{eqnarray}
となることが分かる.ここで,
\begin{eqnarray}
  T = \frac{1}{2}m_{jk}\dot{q}^j\dot{q}^k, \mathcal{F}_i = -\frac{\partial U}{\partial q^i}
\end{eqnarray}
の時,
\begin{eqnarray}
  L = T-U
\end{eqnarray}
として,
\begin{eqnarray}
  \frac{d}{dt}\left( \frac{\partial L}{\partial \dot{q}^i} \right) - \frac{\partial L}{\partial q^i} =0
\end{eqnarray}
となる.これを\textbf{ラグランジュ方程式}という.

\subsection{剛体の運動}
剛体をラグランジュ方程式を用いて解析していく.剛体も質点からなる集合であるため,全質量を$M$,重心を$\bm{r}_c$とすると,
\begin{eqnarray}
  M\frac{d^2\bm{r}_c}{dt^2} = \sum_i \bm{F}_i
\end{eqnarray}
となる.角運動量は$\bm{L} = \sum_i \bm{r}_i\times m_i \bm{r}_i$
\begin{eqnarray}
  \frac{d\bm{L}}{dt} = \sum_i ( \bm{r}_i \times \bm{F}_i )
\end{eqnarray}
である.例えば,剛体が角速度$\omega$で回転しているときの$x$成分は
\begin{eqnarray}
  L_x &&= \sum_i \left\{ \bm{r}_i \times (m_i \bm{v}_i) \right\}_x = \sum_i m_i\left\{ \bm{r}_i \times ( \bm{\omega}\times \bm{r}_i) \right\}_x \\
  &&= \omega_x \sum_i m_i \left( y_i^2 + z_i^2 \right) -\omega_y\sum_i m_i x_iy_i -\omega_z \sum_i m_i z_ix_i\\
  &&= I_x \omega_x - I_{xy}\omega_y - I_{xz}\omega_z
\end{eqnarray}
となる.この式に関してより詳しく見ていく.
\begin{eqnarray}
  \bm{L} = \sum_i \bm{r}\times\dot{\bm{r}}, \left( 連続質量分布では\bm{M} = \int\bm{r}\times\dot{\bm{r}}dm(\bm{r}) \right)
\end{eqnarray}
よって,
\begin{eqnarray}
  \frac{d\bm{L}}{dt} &&= \sum_i m_{i}\bm{r}\times\ddot{\bm{r}} = \sum_i \bm{r}_i \times\left( \bm{F}_i + \sum_{i\neq j}f_{ij} \right)\\
  &&= \sum_i \bm{r}_i \times\left( \bm{F}_i \right)
\end{eqnarray}
となる.ここで,空間に固定した座標系で見た時の剛体の重心の位置ベクトルを$\bm{R}$,重心から$m_i$までのベクトルを$\bm{r}_i$とすると,
\begin{eqnarray}
  \sum_i m_i \bm{r}_i= 0
\end{eqnarray}
となる.そして全運動量とその時間変化は
\begin{eqnarray}
  &&\bm{P}= \sum_i m_i \left( \dot{\bm{R}} + \dot{\bm{r}}_i \right) =  \sum_i m_i\dot{\bm{R}}\\
  &&\dot{\bm{P}} = \sum_i m_i\ddot{\bm{R}} = \sum_i \left( \bm{F}_i + \sum_{i\neq j}f_{ij} \right) =  \sum_i\bm{F}_i 
\end{eqnarray}
となり,全運動量は全外力が重心にかかったものとみなすことができる.なおこの式は重心が非慣性系でも成立する.このように着目すると重心の運動と重心まわりの回転運動を別に扱える.
そして,
\begin{eqnarray}
  T = \frac{1}{2}\sum_i m_i ( \dot{\bm{R}} + \dot{\bm{r}}_i )^2 = \frac{1}{2}\sum_i m_i \left( \dot{\bm{R}}^2 + \dot{\bm{r}}_i^2 \right)
\end{eqnarray}
となる,これは
\begin{eqnarray}
  \sum_i m_i \dot{\bm{R}}\cdot\dot{\bm{r}}_i = \dot{\bm{R}}\cdot \left( \sum_i m_i \dot{\bm{r}}_i \right) =0
\end{eqnarray}
であることを利用した.また,すべての質点は固定されているため,角速度$\bm{\omega}$で回転した時,
\begin{eqnarray}
  \bm{v}_i = \bm{\omega}\times\bm{r}_i
\end{eqnarray}
となる.ここで,角運動量は
\begin{eqnarray}
  \bm{L} &&= \sum_i m_i\bm{r}_i \times \dot{\bm{r}}_i= \sum_i m_i\bm{r}_i \times \left( \bm{\omega}\times\bm{r}_i \right)\\
  &&= \sum_i m_i\bm{r}_i \times \dot{\bm{r}}_i= \sum_i m_i\left( r_i^2 \bm{\omega}-( \bm{\omega}\cdot\bm{r}_i )\bm{r}_i \right)
\end{eqnarray}


\biblio % Needed for referencing to working when compiling individual subfiles - Do not remove
\end{document}