%%%%%%%%%%%%%%%%%%%%%%%%%%%%%%%%%%%%%%%%%%%%%%%%%%%%%%%
%                   File: OSAmeetings.tex             %
%                  Date: 29 Novemver 2018              %
%                                                     %
%     For preparing LaTeX manuscripts for submission  %
%       submission to OSA meetings and conferences    %
%                                                     %
%       (c) 2018 Optical Society of America           %
%%%%%%%%%%%%%%%%%%%%%%%%%%%%%%%%%%%%%%%%%%%%%%%%%%%%%%%

\documentclass[12pt,dvipdfmx]{jsarticle}
%% if A4 paper needed, change letterpaper to A4
\usepackage[dvipdfmx]{graphicx}
\usepackage[dvipdfmx]{color}
\usepackage{osameet3} %% use version 3 for proper copyright statement
\usepackage{ascmac}
%% provide authormark
\newcommand\authormark[1]{\textsuperscript{#1}}

%% standard packages and arguments should be modified as needed
\usepackage{amsmath,amssymb}
\usepackage[colorlinks=true,bookmarks=false,citecolor=blue,urlcolor=blue]{hyperref} %pdflatex
%\usepackage[breaklinks,colorlinks=true,bookmarks=false,citecolor=blue,urlcolor=blue]{hyperref} %latex w/dvipdf
\usepackage{mathtools}
\usepackage{amsmath}
\usepackage{empheq}
\usepackage{physics}
\usepackage[scr=rsfs]{mathalpha}
\usepackage[svgnames]{xcolor}% tikzより前に読み込む必要あり
\usepackage{tikz}
\usepackage{bm}
\usepackage{here}
\usepackage{braket}
\usepackage{framed,color}
\usepackage{dcolumn}
\definecolor{shadecolor}{gray}{0.80}
\usetikzlibrary{perspective}
\tikzset
{%
  my ball/.style={draw,circle,minimum size=2*\r cm,inner sep=0,shading=ball,ball color=cyan!50!blue,opacity=#1},
  my ball/.default=1,
  hidden line/.style={black!60}
}
\begin{document}
\title{実験 カンニングシート}

\author{21B00817 鈴木泰雅,\authormark{1}}
\section*{\Large{計測と解析}}

\subsection*{誤差の伝播公式}
誤差$\Delta f$は
\begin{eqnarray}
  \Delta f= \sqrt{ \sum_i \left(\frac{\partial f}{\partial x_i}\right)^2(\Delta_i)^2 }
\end{eqnarray}
となる.ないしは
\begin{eqnarray}
  f(x,y+\delta) = f(x,y) + \frac{\partial f}{\partial y}(\delta)
\end{eqnarray}
として計算しても同じである.
\subsection*{最小二乗法の公式}
\begin{eqnarray}
  f= ax_{data} + b
\end{eqnarray}
として近似したいとき,最小二乗法によると
\begin{eqnarray}
  a &&= \frac{\overline{xy} - (\overline{x})(\overline{y})}{\overline{x^2} - (\overline{x})^2} = \frac{\mathrm{Cov}[\mathcal{X},\mathcal{Y}]}{\mathrm{Cov}[\mathcal{X}, \mathcal{X}]} = \frac{\mathrm{Cov}[\mathcal{X},\mathcal{Y}]}{\mathrm{Var}[\mathcal{X}]}\\
  b &&= \overline{y} - a\overline{x}
\end{eqnarray}
\section*{\Large{単位と暗記事項}}
\begin{eqnarray}
  &&G= 10^{9},\quad M=10^{6},\quad k = 10^{3} \\
  &&\mu= 10^{-6},\quad n=10^{-9},\quad p = 10^{-12},\quad f = 10^{-15} \\ 
\end{eqnarray}
光子のときのエネルギと運動量は
\begin{eqnarray}
  E = h\nu = 2\pi\hbar\frac{c}{\lambda}\\
  P = E/c = h\nu/c
\end{eqnarray}
なお,
\begin{eqnarray}
  1[\text{eV}] &&= 1.602\cdot 10^{-19}[\text{J}]\\
  \hbar c &&\approx 200[\text{MeV}\cdot\text{fm}]\\
   c &&\approx 3.0\cdot 10^{8}[\text{m}/\text{s}]\\
   h &&\approx 6.62\cdot 10^{-34}[\text{Js}]\\
   k_{\text{B}}&&\approx 1.38\cdot 10^{-23}[\text{J}/\text{K}]
\end{eqnarray}
\begin{eqnarray}
  [\text{W}]= [\text{J}/\text{s}],\quad [\text{J}]= [\text{m}\cdot\text{N}]
\end{eqnarray}
\section*{\Large{光}}
\subsection*{時間分解能}
接近したパルス同士が重なってしまい分離できなくなる.つまりパルスの数を数え落とすようになってしまい数え落としたパルスの数を補正する必要がある.嬉しいことに装置には不感時間$\tau$が付与されているため
計算できる.
\subsection*{光増倍管}
増幅率は以下で定義される:
\begin{eqnarray}
  G = \lambda^n,\quad \lambda:1回のダイノードへの衝突で叩き出される電子の数,n:ダイノードの数
\end{eqnarray}
\subsection*{単位時間あたりのレーザーから放出される光子数}
全体のレーザーのエネルギ/1光子のエネルギーが光子の数であるため
\begin{eqnarray}
  N = \frac{W}{h\nu} = \frac{W}{2\pi\hbar c/\lambda}
\end{eqnarray}
\subsection*{レンズの式}
\begin{eqnarray}
  \frac{1}{f}= \frac{1}{a} + \frac{1}{b}
\end{eqnarray}
なお,一般に凸レンズよりも凹レンズの方が収差が小さい.
\subsection*{偏光板}
光軸方向のみの光が出力されその強度は$\cos(\theta-\phi)$である.ちょうど$\theta=\phi+\pi/2$などのように光軸と振動方向が直交するときは$0$になる.

\subsection*{1/2偏光板}
偏光の位相を$\pi$だけずらしたものである.円偏光であれば反対方向に回転,直線偏光であれば透過軸を対称な直線偏光になる.光の強度は角度依存性なし
\subsection*{1/4偏光板}
位相を$\pi/2$だけずらしたものである.直線偏光が入ると楕円変更になり,ちょうど円偏光になる箇所も存在する.振動方向と光軸
との角度を$\pi/4$にすることで円偏光が得られる.光の強度は角度依存性なし
\subsection*{反射と偏光}
$p$偏光の向きは入射面に対して垂直,$s$偏光は入射面と平行である.
\subsection*{屈折の法則}
\begin{eqnarray}
  n_1 \sin\theta_1 = n_2\sin\theta_2
\end{eqnarray}
となる.ちょうど入射角$\theta_1$と透過角$\theta_2$の関係が
\begin{eqnarray}
  \theta_1 + \theta_2 =\pi/2
\end{eqnarray}
のときブリュースター角は
\begin{eqnarray}
  \theta = \tan^{-1}\left(\frac{n_2}{n_1}\right)
\end{eqnarray}
となる.なお,これは$p$偏光のみに関して起こり$s$偏光では起こらない.
\section*{\Large{真空}}
真空を実現できる装置の性能の順番として
\begin{eqnarray}
  油回転ポンプ(10^{-1}\sim 6.5\cdot 10^{-2}\text{Pa}) < 油拡散ポンプ(\sim 10^{-5}\text{Pa})
\end{eqnarray}
であるが,先に油回転ポンプで油拡散ポンプないの大気をあらびきしておき,油拡散ポンプを使用するのがベストである.
また,計測器としてはガイスラー菅,ピラニゲージが$1atm\sim 10^{-1}\text{Pa}$ほどの圧力を測定することができるが,電離真空計では$10^{-1}\sim 10^{-5}Pa$の圧力を得られる.
\subsection*{気体吸着}
低温で吸着しておりそれを温度を上げていったときに容器内にある分子がそれぞれ脱離する順番は沸点が低い順番だと考えれば良い.
\begin{eqnarray}
  窒素 < 酸素 < 二酸化窒素 \ll 水
\end{eqnarray}
つまり窒素,酸素などの順番で脱離すると考えられる.
\subsection*{コンダクタンス}

配管コンダクタンスは以下で定義されている:
\begin{eqnarray}
  C = \frac{Q}{p_1-p_2}
\end{eqnarray}
なお,$Q$は配管を流れる気体流量,$p_1,p_2$は配管のそれぞれの圧力を表す.直列に接続された場合は
\begin{eqnarray}
  \frac{1}{C_{直列}}= \sum_i \frac{1}{C_i}, \quad C_{並列} = \sum_i C_i
\end{eqnarray}
となる.それぞれの排気速度は
\begin{eqnarray}
  -V_1 \frac{d p_1}{dt} = Q-Q_1,\quad -V_2 \frac{d p_2}{dt}= Q'-Q-Q_2
\end{eqnarray}
となる.
また,コンダクタンスの定義より
\begin{eqnarray}
  C_1 = \frac{Q'}{p_2-p_3}
\end{eqnarray}
などが成立する.

\section*{\Large{エレクトロニクス}}
それぞれの回路は教科書と実験レポートを参照にすること
なお,ローパスフィルタにおいては
\begin{eqnarray}
  f_c = \frac{1}{2\pi R_1 C_1}
\end{eqnarray}
以上の周波数を通しにくく,これ以下の周波数の電流を通すような回路である.

\section*{\Large{放射線}}
厚さ$dx$に入射した$I$個の光子が$-dI$個だけ減少したとすると
\begin{eqnarray}
  -dI = \mu I dx,\quad \mu は吸収係数[\text{cm}{}^{-1}]
\end{eqnarray}
であり,
\begin{eqnarray}
  I(x) = I_0 e^{-\mu x} = I_0 e^{-\sigma nx},\quad \mu= \sigma n
\end{eqnarray}
$n$は単位体積あたりの原子の数$[\text{cm}{}^{-3}]$であり,$\sigma$は面積の次元を持つ1原子あたりの吸収および散乱の全断面積である.
\subsection*{$\gamma$線との相互作用}
$\gamma$線では光電効果,コンプトン効果,電子対生成がある.
\begin{eqnarray}
  &&\sigma_{\text{photo}} \propto Z^5\\
  &&\sigma_{\text{comp}} \propto Z\\
  &&\sigma_{\text{pair}} \propto Z^2
\end{eqnarray}
であり,電子対生成に関しては$1.02\text{MeV}$以上の$\gamma$線だけで相互作用が起きる.
\subsection*{シンチレーターの種類}
密度$\rho$の物質に対して
\textbf{1cm進むごとに$\mathcal{O}(2\rho)[\text{MeV}]$だけエネルギーを失う.}
無機シンチレーターは潮解性があり,大気に晒すとぼろぼろになってしまう.プラスチックシンチレータは安価であり潮解性がない.
\section*{\Large{同軸ケーブル}}
伝送方程式は
\begin{eqnarray}
  \frac{\partial^2 V}{\partial z^2} = LC\frac{\partial^2 V}{\partial t^2}
\end{eqnarray}
であり,ここから速度は
\begin{eqnarray}
  \frac{1}{v^2} = LC,\quad\therefore v =\sqrt{\frac{1}{LC}}
\end{eqnarray}
となる.また,
\begin{eqnarray}
  L = \frac{\mu}{2\pi}\ln\left(\frac{b}{a}\right),\quad C = \frac{2\pi\epsilon}{\ln(b/a)}
\end{eqnarray}
となる.また,インピーダンスは
\begin{eqnarray}
  Z_0 = \sqrt{\frac{L}{C}}
\end{eqnarray}
となる.
\subsection*{インピーダンス整合}
終端で反射しないためには終端のインピーダンス$Z_L$を全体のインピーダンス$Z_0$と同じにする必要がある.

\end{document}