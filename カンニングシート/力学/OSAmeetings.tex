%%%%%%%%%%%%%%%%%%%%%%%%%%%%%%%%%%%%%%%%%%%%%%%%%%%%%%%
%                   File: OSAmeetings.tex             %
%                  Date: 29 Novemver 2018              %
%                                                     %
%     For preparing LaTeX manuscripts for submission  %
%       submission to OSA meetings and conferences    %
%                                                     %
%       (c) 2018 Optical Society of America           %
%%%%%%%%%%%%%%%%%%%%%%%%%%%%%%%%%%%%%%%%%%%%%%%%%%%%%%%

\documentclass[12pt,dvipdfmx]{jsarticle}
%% if A4 paper needed, change letterpaper to A4
\usepackage[dvipdfmx]{graphicx}
\usepackage[dvipdfmx]{color}
\usepackage{osameet3} %% use version 3 for proper copyright statement
\usepackage{ascmac}
%% provide authormark
\newcommand\authormark[1]{\textsuperscript{#1}}

%% standard packages and arguments should be modified as needed
\usepackage{amsmath,amssymb}
\usepackage[colorlinks=true,bookmarks=false,citecolor=blue,urlcolor=blue]{hyperref} %pdflatex
%\usepackage[breaklinks,colorlinks=true,bookmarks=false,citecolor=blue,urlcolor=blue]{hyperref} %latex w/dvipdf
\usepackage{mathtools}
\usepackage{amsmath}
\usepackage{empheq}
\usepackage{physics}
\usepackage[scr=rsfs]{mathalpha}
\usepackage[svgnames]{xcolor}% tikzより前に読み込む必要あり
\usepackage{tikz}
\usepackage{bm}
\usepackage{here}
\usepackage{braket}
\usepackage{framed,color}
\usepackage{dcolumn}
\definecolor{shadecolor}{gray}{0.80}
\usetikzlibrary{perspective}
\tikzset
{%
  my ball/.style={draw,circle,minimum size=2*\r cm,inner sep=0,shading=ball,ball color=cyan!50!blue,opacity=#1},
  my ball/.default=1,
  hidden line/.style={black!60}
}
\begin{document}
\title{力学 カンニングシート}

\author{21B00817 鈴木泰雅,\authormark{1}}
\section*{\Large{極座標系での運動方程式}}
二次元系における極座標の運動方程式
\begin{eqnarray}
  &&m(\ddot{r}-r\dot{\theta}^2) = F_r\\
  &&m\frac{1}{r}\frac{d}{dt}(r^2\dot{\theta})= F_\theta
\end{eqnarray}
となる.
\section*{\Large{剛体}}
\subsection*{慣性モーメント}
\begin{eqnarray}
  I = \int r^2 dm
\end{eqnarray}
\subsection*{運動方程式}
\begin{eqnarray}
  I\frac{d\omega}{dt} = \bm{r}\times\bm{F}
\end{eqnarray}
\subsection*{原理}
剛体の運動では重心方向の運動と,回転方向の運動に分離して考えることもできる.(場合によっては相互作用を削除したい場合はラグランジュ方程式を使う)
また,回転方向は非慣性系でも,慣性力の効果が入らないため同滑車等の議論でもこれを多用する.
\subsection*{運動量,エネルギー}
剛体全体が何も支点に関して固定されていない場合は以下のように書かれる:
\begin{eqnarray}
  \bm{P}&&= M\bm{v}_G, \quad \bm{L}= \bm{r}_G \times M\bm{v}_G + I\omega\\
  T &&= \frac{1}{2}M\bm{v}_G^2 + \frac{1}{2}I\omega^2
\end{eqnarray}

ただし,ただし,剛体がどこかの点で固定されており回転している場合には

\begin{eqnarray}
  \bm{L}&&= I'\omega'\\
  T &&= \frac{1}{2}I'\omega'^2
\end{eqnarray}
とした方が解きやすい.なお慣性モーメントが重心からの点ではなく,また,$\omega'$も支点からの回転であることには注意したい.


\end{document}