%%%%%%%%%%%%%%%%%%%%%%%%%%%%%%%%%%%%%%%%%%%%%%%%%%%%%%%
%                   File: OSAmeetings.tex             %
%                  Date: 29 Novemver 2018              %
%                                                     %
%     For preparing LaTeX manuscripts for submission  %
%       submission to OSA meetings and conferences    %
%                                                     %
%       (c) 2018 Optical Society of America           %
%%%%%%%%%%%%%%%%%%%%%%%%%%%%%%%%%%%%%%%%%%%%%%%%%%%%%%%

\documentclass[12pt,dvipdfmx]{jsarticle}
%% if A4 paper needed, change letterpaper to A4
\usepackage[dvipdfmx]{graphicx}
\usepackage[dvipdfmx]{color}
\usepackage{osameet3} %% use version 3 for proper copyright statement
\usepackage{ascmac}
%% provide authormark
\newcommand\authormark[1]{\textsuperscript{#1}}

%% standard packages and arguments should be modified as needed
\usepackage{amsmath,amssymb}
\usepackage[colorlinks=true,bookmarks=false,citecolor=blue,urlcolor=blue]{hyperref} %pdflatex
%\usepackage[breaklinks,colorlinks=true,bookmarks=false,citecolor=blue,urlcolor=blue]{hyperref} %latex w/dvipdf
\usepackage{mathtools}
\usepackage{amsmath}
\usepackage{empheq}
\usepackage{physics}
\usepackage[scr=rsfs]{mathalpha}
\usepackage[svgnames]{xcolor}% tikzより前に読み込む必要あり
\usepackage{tikz}
\usepackage{bm}
\usepackage{here}
\usepackage{braket}
\usepackage{framed,color}
\usepackage{dcolumn}
\definecolor{shadecolor}{gray}{0.80}
\usetikzlibrary{perspective}
\tikzset
{%
  my ball/.style={draw,circle,minimum size=2*\r cm,inner sep=0,shading=ball,ball color=cyan!50!blue,opacity=#1},
  my ball/.default=1,
  hidden line/.style={black!60}
}
\begin{document}
\title{量子力学 カンニングシート}

\author{21B00817 鈴木泰雅,\authormark{1}}
\section*{\Large{各種公式}}
\subsection*{交換関係}
\begin{eqnarray}
  [AB,C] = [A,C]B + A[B,C],\quad [A+B,C] = [A,C] + [B,C]
\end{eqnarray}
\subsection*{行列の関係式}
\begin{eqnarray}
  &&\exp(iaA) = \cos(a)I +i\sin(a)A,\quad A^2 =I を満たす行列\\
  &&\exp(aA) = \cosh(a)I +\sinh(a)A,\quad A^2 =I を満たす行列
\end{eqnarray}
\subsection*{不確定性原理}
\begin{eqnarray}
  \Delta p\Delta x \geq \frac{\hbar}{2},\quad \Delta E\Delta t \geq\frac{\hbar}{2}
\end{eqnarray}
\subsection*{連続の式}
\begin{eqnarray}
  \frac{\partial\rho}{\partial t} = -\nabla\cdot\bm{j},\quad \bm{j} = \frac{i\hbar}{2m}\left( \Psi\Delta\Psi^{*}-\Psi^{*}\Delta\Psi \right)
\end{eqnarray}
ただし$\rho=|\Psi|^2$を満たしている.
\subsection*{ハイゼンベルク方程式}
\begin{eqnarray}
  \frac{d\hat{X}}{dt}  = \frac{i}{\hbar}\left[ \hat{H}, \hat{X} \right]
\end{eqnarray}
\section*{\Large{摂動論}}
\begin{eqnarray}
  E_0 = E_0^{(0)}+ \lambda\langle 0|\hat{V}|0\rangle + \lambda^2 \sum_{j\neq 0}\frac{|\langle j|\hat{V}|0\rangle|^2}{E_0^{(0)}-E_j^{(0)}}
\end{eqnarray}
状態は
\begin{eqnarray}
  |\psi\rangle = |0\rangle + \lambda\sum_{i\neq j}|j\rangle \frac{\langle j|\hat{V}|0\rangle}{E_0^{(0)}-E_j^{(0)}}
\end{eqnarray}
\section*{\Large{スピン}}
\begin{eqnarray}
  j_+|j,m\rangle= \sqrt{ j(j+1) }
\end{eqnarray}

\end{document}