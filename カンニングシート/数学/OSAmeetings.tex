%%%%%%%%%%%%%%%%%%%%%%%%%%%%%%%%%%%%%%%%%%%%%%%%%%%%%%%
%                   File: OSAmeetings.tex             %
%                  Date: 29 Novemver 2018              %
%                                                     %
%     For preparing LaTeX manuscripts for submission  %
%       submission to OSA meetings and conferences    %
%                                                     %
%       (c) 2018 Optical Society of America           %
%%%%%%%%%%%%%%%%%%%%%%%%%%%%%%%%%%%%%%%%%%%%%%%%%%%%%%%

\documentclass[12pt,dvipdfmx]{jsarticle}
%% if A4 paper needed, change letterpaper to A4
\usepackage[dvipdfmx]{graphicx}
\usepackage[dvipdfmx]{color}
\usepackage{osameet3} %% use version 3 for proper copyright statement
\usepackage{ascmac}
%% provide authormark
\newcommand\authormark[1]{\textsuperscript{#1}}

%% standard packages and arguments should be modified as needed
\usepackage{amsmath,amssymb}
\usepackage[colorlinks=true,bookmarks=false,citecolor=blue,urlcolor=blue]{hyperref} %pdflatex
%\usepackage[breaklinks,colorlinks=true,bookmarks=false,citecolor=blue,urlcolor=blue]{hyperref} %latex w/dvipdf
\usepackage{mathtools}
\usepackage{amsmath}
\usepackage{empheq}
\usepackage{physics}
\usepackage[scr=rsfs]{mathalpha}
\usepackage[svgnames]{xcolor}% tikzより前に読み込む必要あり
\usepackage{tikz}
\usepackage{bm}
\usepackage{here}
\usepackage{braket}
\usepackage{framed,color}
\usepackage{dcolumn}
\definecolor{shadecolor}{gray}{0.80}
\usetikzlibrary{perspective}
\tikzset
{%
  my ball/.style={draw,circle,minimum size=2*\r cm,inner sep=0,shading=ball,ball color=cyan!50!blue,opacity=#1},
  my ball/.default=1,
  hidden line/.style={black!60}
}
\begin{document}
\title{数学 カンニングシート}

\author{21B00817 鈴木泰雅,\authormark{1}}
\section*{\Large{極座標}}
\subsection*{3次元}
\begin{eqnarray}
  \nabla &&= \bm{e}_r \frac{\partial}{\partial r} + \bm{e}_{\theta}\frac{1}{r} \frac{\partial}{\partial\theta} + \bm{e}_{\phi}\frac{1}{r\sin\theta}\frac{\partial}{\partial \phi}\\
  \nabla^2 &&= \frac{1}{r^2}\frac{\partial}{\partial r}\left( r^2\frac{\partial}{\partial r} \right) + \frac{1}{r^2\sin\theta}\frac{\partial}{\partial\theta}\left( \sin\theta\frac{\partial}{\partial\theta} \right) + \frac{1}{r^2\sin^2\theta}\frac{\partial^2}{\partial \phi^2}
\end{eqnarray}
ヤコビアンは$r^2\sin\theta$
\subsection*{2次元}
\begin{eqnarray}
  \nabla &&= \bm{e}_r \frac{\partial}{\partial r} + \bm{e}_{\theta}\frac{1}{r} \frac{\partial}{\partial\theta}\\
  \nabla^2 &&= \frac{1}{r}\frac{\partial}{\partial r}\left( r \frac{\partial}{\partial r} \right) + \frac{1}{r^2}\frac{\partial}{\partial\theta^2}
\end{eqnarray}
ヤコビアンは$r$
\section*{\Large{円柱座標}}
\begin{eqnarray}
  \nabla &&= \bm{e}_r \frac{\partial}{\partial r} + \bm{e}_{\theta}\frac{1}{r}\frac{\partial}{\partial \theta} +\bm{e}_z \frac{\partial}{\partial z}\\
  \nabla^2 &&= \frac{1}{r}\frac{\partial}{\partial r}\left( r\frac{\partial}{\partial r} \right) + \frac{1}{r^2}\frac{\partial^2}{\partial\theta^2} + \frac{\partial^2}{\partial z^2}
\end{eqnarray}
ヤコビアンは$r$
\section*{\Large{デルタ関数}}
\begin{eqnarray}
  \delta(ax) &&= \frac{1}{|a|}\delta(x)\\
  \delta(f(x)) &&= \sum_i \frac{1}{|f(a_i)|}\delta(x-a_i)\\
  \delta(x) &&= \frac{1}{2\pi}\int_{-\infty}^{\infty} e^{ikx}dk\\
  \Delta\left( \frac{1}{r} \right) &&= -4\pi\delta(r)\\
  \nabla\left( \frac{1}{r} \right) &&= -\frac{\bm{e}_r}{r^2} = -\frac{\bm{r}}{r^3},\quad\therefore \nabla\cdot\left( \frac{\bm{r}}{r^3} \right) = 4\pi\delta(r)
\end{eqnarray}
また、クロネッカーのデルタに関しては次が有名である:
\begin{eqnarray}
  \delta_{n,m} = \frac{1}{2\pi}\int_{-\pi}^{\pi}e^{i(n-m)\theta}d\theta
\end{eqnarray}
\section*{\Large{三角関数・双曲線関数}}
\subsection*{展開}
\begin{eqnarray}
  &&\sin(x) \sim x-\frac{x^3}{3!}+\cdots\\
  &&\cos(x) \sim 1-\frac{x^2}{2!}+\cdots\\
  &&\tan(x) \sim 1+\frac{x^3}{3!}+\cdots
\end{eqnarray}

\begin{eqnarray}
  &&\sinh(x) \sim x+\frac{x^3}{3!}+\cdots\\
  &&\cosh(x) \sim 1+\frac{x^2}{2!}+\cdots\\
  &&\tanh(x) \sim 1-\frac{x^3}{3!}+\cdots
\end{eqnarray}
\subsection*{微分とかの性質}
\begin{eqnarray}
  &&\cosh^2(x)-\sinh^2(x)=1\\
  &&1-\tanh^2(x) =\frac{1}{\cosh^2(x)}
\end{eqnarray}
微分は
\begin{eqnarray}
  &&\left(\cosh(x)\right)' = \sinh(x)\\
  &&\left(\sinh(x)\right)' = \cosh(x)\\
  &&\left(\tanh(x)\right)' = \frac{1}{\cosh^2(x)}
\end{eqnarray}
\section*{\Large{ベクトル解析}}
\begin{eqnarray}
  &&A\cdot (B\times C) = (A\times B)\cdot C = C\cdot (A\times B) = (C\times A)\cdot B = B\cdot (A\times B)\\
  &&A\times (B\times C) = B(A\cdot C) - C(A\cdot B)\\
  &&(A\times B)\cdot(C\times D)= (A\cdot C)(B\cdot D) - (A\cdot D)(B\cdot C)
\end{eqnarray}
また,
\begin{eqnarray}
  \nabla\cdot(\nabla\times\bm{A})= \nabla\times(\nabla f)=0
\end{eqnarray}
となる.

\section*{\Large{積分}}
\subsection*{ジョルダン不等式}
\begin{eqnarray}
  \int_0^\pi e^{-r\sin\theta}d\theta < \frac{\pi}{r}
\end{eqnarray}
\subsection*{留数定理}
\begin{eqnarray}
  \int_C f(z)dz= 2\pi i \sum \Res_{z=z_0}f(z)
\end{eqnarray}
なお,留数は以下で求められる
\begin{eqnarray}
  \Res_{z=z_0}f(z) = \frac{1}{(m-1)!}\lim_{z\to z_0}\frac{d^{m-1}}{dz^{m-1}}\left[ (z-z_0)^m f(z) \right]
\end{eqnarray}
\subsection*{$\log$の入った積分}
これは分枝載線に気を付ける.
\subsection*{$\sin$の入った積分}
\begin{eqnarray}
  I = \int_0^{2\pi}f(\cos\theta,\sin\theta)d\theta
\end{eqnarray}
の時は$z=e^{i\theta}$と置いて,原点を中心とする単位円上で積分を行えばよい.
$\infty$などが積分に入っている場合は$\Re$などを最終的に取る方が実は良かったりする.
\section*{\Large{フーリエ}}
\subsection*{フーリエ級数展開}
$(-\pi,\pi)$において定義された関数$f(x)$に関して
\begin{eqnarray}
  f(x) &&\sim \frac{a_0}{2} + \sum_{n=1}^{\infty}\left( a_n \cos(nx) + b_n\sin(nx) \right),\\
  && a_n := \frac{1}{\pi}\int_{-\pi}^{\pi} f(x)\cos(nx) dx\\
  && b_n := \frac{1}{\pi}\int_{-\pi}^{\pi} f(x)\sin(nx)dx
\end{eqnarray}
\subsection*{フーリエ変換}
\begin{eqnarray}
  \tilde{f}(k)= \frac{1}{\sqrt{2\pi}} \int_{-\infty}^{\infty}f(x)e^{-ikx}dx
\end{eqnarray}

\section*{\Large{ラプラス変換}}
\begin{eqnarray}
  F(p) = \int_0^\infty e^{-pt}f(t)dt
\end{eqnarray}
で与えられる.微分方程式を解く際に覚えなければいけない公式は
\begin{eqnarray}
  \mathcal{L}[f^{(n)}(t)] = p^nF(p) -p^{n-1}f(0) - p^{n-2} f(0) -\cdots
\end{eqnarray}
であり,
\begin{eqnarray}
  \mathcal{L}[e^{at}] = \frac{1}{p-a}, \quad \mathcal{L}[1]= \frac{1}{p},\quad \mathcal{L}[\cos(at)] = \frac{p}{p^2+a^2}, \quad \mathcal{L}[\sin(at)] = \frac{a}{p^2+a^2}
\end{eqnarray}
\section*{\Large{特殊関数}}
\subsection*{\large{ガンマ関数}}
\begin{eqnarray}
  \Gamma(x) := \int_0^\infty e^{-t}t^{x-1}dx
\end{eqnarray}
部分積分より
\begin{eqnarray}
  \Gamma(x+1) = x \Gamma(x)
\end{eqnarray}
が成立し,
\begin{eqnarray}
  \Gamma(1)=1,\quad \Gamma(n+1) = n\Gamma(n),\quad \Gamma\left(\frac{1}{2}\right)= \sqrt{\pi},\quad \Gamma(n+1) = n!
\end{eqnarray}
が成立する.
\subsection*{\large{ベータ関数}}
\begin{eqnarray}
  B(z,\xi) := \int_0^1 t^{z-1}(1-t)^{\xi-1}dt
\end{eqnarray}
また,
\begin{eqnarray}
  B(z,\xi) = \frac{\Gamma(z)\Gamma(\xi)}{\Gamma(z+\xi)}
\end{eqnarray}
\subsection*{\large{ルジャンドル関数}}
ルジャンドルの微分方程式は
\begin{eqnarray}
  (1-x^2)y'' -2xy' + \lambda y=0
\end{eqnarray}
この時$\lambda=n(n+1)$の値を取る時,解はルジャンドル多項式になる$P_n(x)$なお,これはロドリゲスの公式により
\begin{eqnarray}
  P_n(x) = \frac{1}{2^n n!}\frac{d^n}{dx^n}(x^2-1)^n
\end{eqnarray}
\subsection*{\large{ベッセル関数}}
ベッセルの微分方程式は
\begin{eqnarray}
  x^2y'' + xy' + (x^2-\nu^2)y=0
\end{eqnarray}
\subsection*{\large{エルミート多項式}}
\begin{eqnarray}
  e^{-z^2+2zx} = \sum_{n=0}^{\infty} H_n(x)\frac{z^n}{n!}
\end{eqnarray}
である.各種性質はレポートを見る.なお,エルミート多項式は次のエルミート微分方程式を満たす:
\begin{eqnarray}
  y''-2xy' + 2ny=0
\end{eqnarray}
あとはレポートを見直せば問題ない
\end{document}