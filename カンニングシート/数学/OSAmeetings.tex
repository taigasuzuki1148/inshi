%%%%%%%%%%%%%%%%%%%%%%%%%%%%%%%%%%%%%%%%%%%%%%%%%%%%%%%
%                   File: OSAmeetings.tex             %
%                  Date: 29 Novemver 2018              %
%                                                     %
%     For preparing LaTeX manuscripts for submission  %
%       submission to OSA meetings and conferences    %
%                                                     %
%       (c) 2018 Optical Society of America           %
%%%%%%%%%%%%%%%%%%%%%%%%%%%%%%%%%%%%%%%%%%%%%%%%%%%%%%%

\documentclass[12pt,dvipdfmx]{jsarticle}
%% if A4 paper needed, change letterpaper to A4
\usepackage[dvipdfmx]{graphicx}
\usepackage[dvipdfmx]{color}
\usepackage{osameet3} %% use version 3 for proper copyright statement
\usepackage{ascmac}
%% provide authormark
\newcommand\authormark[1]{\textsuperscript{#1}}

%% standard packages and arguments should be modified as needed
\usepackage{amsmath,amssymb}
\usepackage[colorlinks=true,bookmarks=false,citecolor=blue,urlcolor=blue]{hyperref} %pdflatex
%\usepackage[breaklinks,colorlinks=true,bookmarks=false,citecolor=blue,urlcolor=blue]{hyperref} %latex w/dvipdf
\usepackage{mathtools}
\usepackage{amsmath}
\usepackage{empheq}
\usepackage{physics}
\usepackage[scr=rsfs]{mathalpha}
\usepackage[svgnames]{xcolor}% tikzより前に読み込む必要あり
\usepackage{tikz}
\usepackage{bm}
\usepackage{here}
\usepackage{braket}
\usepackage{framed,color}
\usepackage{dcolumn}
\definecolor{shadecolor}{gray}{0.80}
\usetikzlibrary{perspective}
\tikzset
{%
  my ball/.style={draw,circle,minimum size=2*\r cm,inner sep=0,shading=ball,ball color=cyan!50!blue,opacity=#1},
  my ball/.default=1,
  hidden line/.style={black!60}
}
\begin{document}
\title{数学 カンニングシート}

\author{21B00817 鈴木泰雅,\authormark{1}}
\section*{\Large{極座標}}
\subsection*{3次元}
\begin{eqnarray}
  \nabla &&= \bm{e}_r \frac{\partial}{\partial r} + \bm{e}_{\theta}\frac{1}{r} \frac{\partial}{\partial\theta} + \bm{e}_{\phi}\frac{1}{r\sin\theta}\frac{\partial}{\partial \phi}\\
  \nabla^2 &&= \frac{1}{r^2}\frac{\partial}{\partial r}\left( r^2\frac{\partial}{\partial r} \right) + \frac{1}{r^2\sin\theta}\frac{\partial}{\partial\theta}\left( \sin\theta\frac{\partial}{\partial\theta} \right) + \frac{1}{r^2\sin^2\theta}\frac{\partial^2}{\partial \phi^2}
\end{eqnarray}
ヤコビアンは$r^2\sin\theta$
\subsection*{2次元}
\begin{eqnarray}
  \nabla &&= \bm{e}_r \frac{\partial}{\partial r} + \bm{e}_{\theta}\frac{1}{r} \frac{\partial}{\partial\theta}\\
  \nabla^2 &&= \frac{1}{r}\frac{\partial}{\partial r}\left( r \frac{\partial}{\partial r} \right) + \frac{1}{r^2}\frac{\partial}{\partial\theta^2}
\end{eqnarray}
ヤコビアンは$r$
\section*{\Large{円柱座標}}
\begin{eqnarray}
  \nabla &&= \bm{e}_r \frac{\partial}{\partial r} + \bm{e}_{\theta}\frac{1}{r}\frac{\partial}{\partial \theta} +\bm{e}_z \frac{\partial}{\partial z}\\
  \nabla^2 &&= \frac{1}{r}\frac{\partial}{\partial r}\left( r\frac{\partial}{\partial r} \right) + \frac{1}{r^2}\frac{\partial^2}{\partial\theta^2} + \frac{\partial^2}{\partial z^2}
\end{eqnarray}
ヤコビアンは$r$
\section*{\Large{デルタ関数}}
\begin{eqnarray}
  \delta(ax) &&= \frac{1}{|a|}\delta(x)\\
  \delta(f(x)) &&= \sum_i \frac{1}{|f(a_i)|}\delta(x-a_i)\\
  \delta(x) &&= \frac{1}{2\pi}\int_{-\infty}^{\infty} e^{ikx}dk\\
  \Delta\left( \frac{1}{r} \right) &&= -4\pi\delta(r)\\
  \nabla\left( \frac{1}{r} \right) &&= -\frac{\bm{e}_r}{r^2} = -\frac{\bm{r}}{r^3},\quad\therefore \nabla\cdot\left( \frac{\bm{r}}{r^3} \right) = 4\pi\delta(r)
\end{eqnarray}
\section*{\Large{三角関数・双曲線関数}}
\subsection*{展開}
\begin{eqnarray}
  &&\sin(x) \sim x-\frac{x^3}{3!}+\cdots\\
  &&\cos(x) \sim 1-\frac{x^2}{2!}+\cdots\\
  &&\tan(x) \sim 1+\frac{x^3}{3!}+\cdots
\end{eqnarray}

\begin{eqnarray}
  &&\sinh(x) \sim x+\frac{x^3}{3!}+\cdots\\
  &&\cosh(x) \sim 1+\frac{x^2}{2!}+\cdots\\
  &&\tanh(x) \sim 1-\frac{x^3}{3!}+\cdots
\end{eqnarray}
\subsection*{微分とかの性質}
\begin{eqnarray}
  &&\cosh^2(x)-\sinh^2(x)=1\\
  &&1-\tanh^2(x) =\frac{1}{\cosh^2(x)}
\end{eqnarray}
微分は
\begin{eqnarray}
  &&\left(\cosh(x)\right)' = \sinh(x)\\
  &&\left(\sinh(x)\right)' = \cosh(x)\\
  &&\left(\tanh(x)\right)' = \frac{1}{\cosh^2(x)}
\end{eqnarray}

\end{document}