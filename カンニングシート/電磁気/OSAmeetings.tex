%%%%%%%%%%%%%%%%%%%%%%%%%%%%%%%%%%%%%%%%%%%%%%%%%%%%%%%
%                   File: OSAmeetings.tex             %
%                  Date: 29 Novemver 2018              %
%                                                     %
%     For preparing LaTeX manuscripts for submission  %
%       submission to OSA meetings and conferences    %
%                                                     %
%       (c) 2018 Optical Society of America           %
%%%%%%%%%%%%%%%%%%%%%%%%%%%%%%%%%%%%%%%%%%%%%%%%%%%%%%%

\documentclass[12pt,dvipdfmx]{jsarticle}
%% if A4 paper needed, change letterpaper to A4
\usepackage[dvipdfmx]{graphicx}
\usepackage[dvipdfmx]{color}
\usepackage{osameet3} %% use version 3 for proper copyright statement
\usepackage{ascmac}
%% provide authormark
\newcommand\authormark[1]{\textsuperscript{#1}}

%% standard packages and arguments should be modified as needed
\usepackage{amsmath,amssymb}
\usepackage[colorlinks=true,bookmarks=false,citecolor=blue,urlcolor=blue]{hyperref} %pdflatex
%\usepackage[breaklinks,colorlinks=true,bookmarks=false,citecolor=blue,urlcolor=blue]{hyperref} %latex w/dvipdf
\usepackage{mathtools}
\usepackage{amsmath}
\usepackage{empheq}
\usepackage{physics}
\usepackage[scr=rsfs]{mathalpha}
\usepackage[svgnames]{xcolor}% tikzより前に読み込む必要あり
\usepackage{tikz}
\usepackage{bm}
\usepackage{here}
\usepackage{braket}
\usepackage{framed,color}
\usepackage{dcolumn}
\definecolor{shadecolor}{gray}{0.80}
\usetikzlibrary{perspective}
\tikzset
{%
  my ball/.style={draw,circle,minimum size=2*\r cm,inner sep=0,shading=ball,ball color=cyan!50!blue,opacity=#1},
  my ball/.default=1,
  hidden line/.style={black!60}
}
\begin{document}
\title{電磁気 カンニングシート}

\author{21B00817 鈴木泰雅,\authormark{1}}
\section*{\Large{静電場(誘電体を含む)}}
\subsection*{\large{Maxwell方程式}}
\begin{eqnarray}
  &&\nabla\cdot \bm{D} = \rho\\
  &&\nabla\cdot\bm{B}=0\\
  &&\nabla\times \bm{E} = -\frac{\partial \bm{B}}{\partial t}\\
  &&\nabla\times \bm{H} = \bm{j} + \frac{\partial\bm{E}}{\partial t}
\end{eqnarray}
\subsection*{\large{ビオ・ザバールの法則}}
\begin{eqnarray}
  d\bm{B} = \mu_0 \frac{I}{4\pi}\frac{d\bm{s}\times \bm{r}}{r^3}
\end{eqnarray}
\subsection*{\large{アンペールの法則}}
\begin{eqnarray}
  \oint_C \bm{B}\cdot d\bm{s} = \sum \mu_0 I
\end{eqnarray}
\subsection*{\large{磁界が電流に及ぼす力}}
単位長さあたり
\begin{eqnarray}
  d\bm{F}=\bm{I}\times\bm{B}
\end{eqnarray}
が成立する.ただし,この式は二つの電流無限に長いときであり,有限の時は成立しない.


\subsection*{\large{コンデンサー}}
\begin{eqnarray}
  &&C= \frac{Q}{V_1-V_2}\\
  &&C_{並列} = \sum_i C_i,\quad \frac{1}{C_{直列}} = \sum_i \frac{1}{C_i}
\end{eqnarray}
\subsection*{\large{誘電体中の電界}}
\begin{eqnarray}
  \bm{E}_{全体} &&= \bm{E}_{作用している電場} + \bm{E}'_{誘導される電場}\\
  &&=\bm{E}_{作用している電場} - \frac{\bm{P}}{3\epsilon_0}|_{球体のとき}\\
  &&=\bm{E}_{作用している電場} - \frac{\bm{P}}{\epsilon_0}|_{平面板のとき, 法線方向}\\
  &&=\bm{E}_{作用している電場} - \frac{\bm{P}}{2\epsilon_0}|_{棒のとき, 垂直方向}
\end{eqnarray}
なお,
\begin{eqnarray}
  \bm{D} = \epsilon \bm{E}_{全体}= \epsilon_0\left(1+\chi  \right)\bm{E}_{全体} = \epsilon_0 \bm{E}_{全体} + \bm{P}  
\end{eqnarray}
また,誘電体の表面にある面密度:
\begin{eqnarray}
  \sigma_P = \bm{n}\cdot\bm{P}=P_n
\end{eqnarray}
内部にある体積密度:
\begin{eqnarray}
  \rho_P = -\nabla\bm{P}(\bm{r})
\end{eqnarray}
\subsection*{\large{磁性体中の磁界}}
誘電体と同様にして
\begin{eqnarray}
  \bm{H}_{全体} &&= \bm{H}_{作用している磁場}+\bm{H}'_{誘導される磁場}\\
  &&=\bm{H}_{作用している磁場}-\frac{N_{\alpha}\bm{M}}{\mu_0}
\end{eqnarray}
といった形で誘導電場と同じように反磁場が働く.
\begin{eqnarray}
  \bm{B} = \mu\bm{H}_{全体}= \mu_0(1+\chi)\bm{H}_{全体}= \mu_0 \bm{H}_{全体} + \bm{M}
\end{eqnarray}
なお,反磁性体と常磁性体の大きな違いは$\mu,\mu_0$の大きさの違いであり,法則自体には違いは全くない.ただし,自発磁化$M_0$などが働く場合は
\begin{eqnarray}
  \bm{B} = \mu\bm{H}_{全体} + \bm{M}_0
\end{eqnarray}
になるため上記の公式は使えない.特に自発磁化がない反磁性体などに有効な公式である.
\subsection*{一般の磁化電流}
磁性体の内部の空間に分布している磁化電流は
\begin{eqnarray}
  \bm{i}_M = \frac{1}{\mu_0}\nabla\times\bm{M}
\end{eqnarray}
磁性体の表面に流れる磁化電流は
\begin{eqnarray}
  \bm{j}_M = \frac{-1}{\mu_0}\bm{n}\times\bm{M}
\end{eqnarray}

\subsection*{\large{ポテンシャル}}
\begin{eqnarray}
  \bm{B} = \nabla\times \bm{A},\quad \bm{E} = -\frac{\partial\bm{A}}{\partial t} -\nabla\phi
\end{eqnarray}
\subsection*{\large{Maxwellの応力}}
\begin{eqnarray}
  T = 
  \begin{bmatrix}
    E_x D_x -\frac{1}{2}\bm{E}\bm{D} & E_x D_y & E_x D_z\\
    E_y D_x  & E_y D_y -\frac{1}{2}\bm{E}\bm{D}  & E_y D_z\\
    E_z D_x  & E_z D_y & E_z D_z -\frac{1}{2}\bm{E}\bm{D}
  \end{bmatrix}
\end{eqnarray}
となる.
\section*{\Large{電磁波}}
\subsection*{\large{空間上の電磁波}}
平面電磁波の関係:
\begin{eqnarray}
  \frac{H}{E}= \sqrt{\frac{\epsilon}{\mu}},\quad \frac{1}{2}\epsilon E^2 = \frac{1}{2}\mu H^2
\end{eqnarray}
となり、エネルギーは電波と磁場に半分づつ分かれている。
\subsection*{\large{エネルギーの流れ}}
ポンティングベクトルは以下で定義される:
\begin{eqnarray}
  S = E\times H
\end{eqnarray}
ただし注意として
\begin{eqnarray}
  S = \frac{1}{\mu_0}\Re\bm{E}\times\Re\bm{B}
\end{eqnarray}
である。
運動量の流れは
\begin{eqnarray}
  \frac{1}{c^2}\langle S \rangle
\end{eqnarray}
\subsection*{\large{導体中の電磁波}}
この時は$\sigma,\mu$
\begin{eqnarray}
  \bm{E} = \sigma \bm{J}
\end{eqnarray}
を使って次の電信方程式を立てられる:
\begin{eqnarray}
  \frac{\partial E_x}{\partial z^2} = \mu\epsilon\frac{\partial E_x}{\partial t^2} + \sigma\mu\frac{\partial E_x}{\partial t}
\end{eqnarray}
完全導体では$\sigma=\infty$の導体のことであり、超伝導体に似ているが、超伝導体はこれ以外にも様々な性質を持っている。
\subsection*{\large{Maxwell方程式から得られるもの}}
Maxwell方程式から
\begin{eqnarray}
  -\bm{j}\cdot\bm{E} = \nabla\cdot\left(\bm{E}\times\frac{1}{\mu}\bm{B}\right) + \frac{\partial}{\partial t}\left[ \frac{1}{2}\left(\frac{1}{\mu}\bm{B}^2 + \epsilon\bm{E}^2\right) \right]
\end{eqnarray}
が得られる。

\end{document}